\input tex/epsf.tex
\font\sixteen=cmbx16
\font\twelve=cmr12
\font\fonteautor=cmbx12
\font\fonteemail=cmtt10
\font\twelvenegit=cmbxti12
\font\twelvebold=cmbx12
\font\trezebold=cmbx13
\font\twelveit=cmsl12
\font\monodoze=cmtt12
\font\it=cmti12
\voffset=0,959994cm % 3,5cm de margem superior e 2,5cm inferior
\parskip=6pt

\def\titulo#1{{\noindent\sixteen\hbox to\hsize{\hfill#1\hfill}}}
\def\autor#1{{\noindent\fonteautor\hbox to\hsize{\hfill#1\hfill}}}
\def\email#1{{\noindent\fonteemail\hbox to\hsize{\hfill#1\hfill}}}
\def\negrito#1{{\twelvebold#1}}
\def\italico#1{{\twelveit#1}}
\def\monoespaco#1{{\monodoze#1}}
\def\iniciocodigo{\lineskip=0pt\parskip=0pt}
\def\fimcodigo{\twelve\parskip=0pt plus 1pt\lineskip=1pt}

\long\def\abstract#1{\parshape 10 0.8cm 13.4cm 0.8cm 13.4cm
0.8cm 13.4cm 0.8cm 13.4cm 0.8cm 13.4cm 0.8cm 13.4cm 0.8cm 13.4cm
0.8cm 13.4cm 0.8cm 13.4cm 0.8cm 13.4cm
\noindent{{\twelvenegit Abstract: }\twelveit #1}}

\def\resumo#1{\parshape  10 0.8cm 13.4cm 0.8cm 13.4cm
0.8cm 13.4cm 0.8cm 13.4cm 0.8cm 13.4cm 0.8cm 13.4cm 0.8cm 13.4cm
0.8cm 13.4cm 0.8cm 13.4cm 0.8cm 13.4cm
\noindent{{\twelvenegit Resumo: }\twelveit #1}}

\def\secao#1{\vskip12pt\noindent{\trezebold#1}\parshape 1 0cm 15cm}
\def\subsecao#1{\vskip12pt\noindent{\twelvebold#1}}
\def\referencia#1{\vskip6pt\parshape 5 0cm 15cm 0.5cm 14.5cm 0.5cm 14.5cm
0.5cm 14.5cm 0.5cm 14.5cm {\twelve\noindent#1}}

%@* .

\twelve
\vskip12pt
\titulo{Weaver Random Number Generator}
\vskip12pt
\autor{Thiago Leucz Astrizi}
\vskip6pt
\email{thiago@@bitbitbit.com.br}
\vskip6pt

\abstract{This article contains the implementation of several random number
generator algorithms in literary programming and is intended to be
used with Weaver Game Engine. However, it can also be used as an
independent API in other projects. All different algorithms are
encapsulated in the same API letting programmers to easily change the
algorithm using macro definitions. We also describe here how the
algorithms perform in some statistical tests and in benchmarks running
in Linux, OpenBSD, Windows and Web Assembly.}

%\vskip 0.5cm plus 3pt minus 3pt

\secao{1. Introduction}

\subsecao{1.1. Random Number Generators in Games}

A random number generator is useful in the following scenarios in a
game:

a) Simulation: This applies to natural phenomenon that can be random
 but also applies to social phenomenon. This is commonly used in games
 that are simulators. But also can simulate changes in the weather,
 wind or the mood of characters in practically any kind of game.

b) Sampling: A game can have different number of opponents that could
be generated. By the generation of different characteristics chosen
randomly we are sampling one between all possible characters that
could be generated. For example in a Pokémon game each species of the
creatures can have lots of variation in different attributes depending
of the genetics and also the attacks that the creature knows can
change. Generating a random creature, we are sampling one individual
between other possible ones. We could also have a game with a
procedurally generated world, where the entire world is sampled from
the set of all possible worlds.

c) Programming: During the game development, the programmer can use a
random number generator to simulate different choices that a player
could do just to test if he can find some combinations that bring
problems. And also can generate random levels to test if some bug
involving the game engine can be spotted.

d) Aesthetics: Adding some randomness in image patterns that otherwise
would be regular images can increase the aesthetic appeal for
them. Some randomness also can make screen transitions more
interesting.

e) Entertainment: The fun of a game could be entirely in the
randomness. This is the case for games involving dices, roulette,
shuffling cards.

With so many different use cases for random number generators in
games, we really need a general purpose random number generator. Never
a generator specialized only in some of the above use cases.

Usually we have a greater tolerance when a random number generator
produces predictable results in a game. Lots of old games used in the
past generators that consisted in a sequence of fixed numbers, and the
bias in such games was not apparent. But there are cases when the
consequence of a bad generator are very severe. For example, a
simulator that aims for realism can produce a bad training for players
that use the game to trains for real activities. A bad generator can
be exploited in games involving bets or serious competition.

Our objective here will be define a random number generator API and
different options for random number generators to instantiate it. A
user can choose which algorithm will be used, but if no choice is
presented, the defined library will make its own choice about the
algorithm.

\subsecao{1.2. Literary Programming and Notation}

This article uses the technique of ``Literary Programming'' to develop
our random number generator API. This technique was presented at
[Knuth, 1984] and have as objective develop software in a way that a
computer program to be compiled is exactly equal a document written
for human beings detailing and explaining the code. This document is
not independent of the source code, it is the project source
code. Automated tools are used to extract the code from this document,
sort it in the right order and produce the code that is passed to a
compiler.

For example, in this article we will define two different
files: \monoespaco{random.c} and \monoespaco{random.h}. Both of them
can be inserted statically in any project or compiled as a shared
library. The content of \monoespaco{random.h} is:

\iniciocodigo
@(src/random.h@>=
#ifndef WEAVER_RANDOM
#define WEAVER_RANDOM
#ifdef __cplusplus
extern "C" {
#endif
#include <stdint.h>
#include <stdbool.h>
#if defined(__unix__) || defined(__APPLE__)
#include <pthread.h>
#endif
@<Choose Default RNG Algorithm@>
@<RNG Structs@>
@<RNG Declarations@>
#ifdef __cplusplus
}
#endif
#endif
@
\fimcodigo

The two first lines and the last one are macros that ensure that the
function declaration and variables from this file will always be
included at most once in a compiling unit. We also put macros to check
if we are compiling this as C or C++. If we are in C++, we assure the
compiler that all our functions will be in C-style. We never will
modify them with operator overload, for example. This makes the code
became more compact.

The red part in the above code shows that some code will be inserted
there in the future. In ``RNG Declarations'', for example, we will put
there function declarations.

Each piece of code have a title, that in the above example is
\monoespaco{random.h}. The title shows where the code will be placed.
In the case above, the code will directly to a file. In future pieces
of code, we will have a title ``RNG Declarations'' with the code that
will be inserted in the place of the red text in the code above.

\subsecao{1.3. API functions that will be defined}

Our RNG API will have three different functions.

The first one generates and initializes a random number generator
returning a pointer for it. To initialize it, we will require a memory
allocation function (can be \monoespaco{malloc} from standard library,
or a custom memory allocator). Next we need a seed represented as an
array size and an array of 64-bit integers that we will assume being
chosen uniformly at random.

This way, our initialization function is generic enough that the user
can use a personalized amount of entropy for the generator. In a
shuffling card game, for example, if we want to shuffle 54 cards, we
would want in the initialization a number of $n$ random bits such that
$2^n$ is not much smaller than $54!$. Otherwise, we surely will not be
able to generate most of the possible shuffling. However, in some
machines and environment, getting randomness could be difficult and we
should be able to pass a smaller number of random bits during
initialization.

\iniciocodigo
@<RNG Declarations@>=
struct _Wrng *_Wcreate_rng(void *(*alloc)(size_t), size_t size, uint64_t *seed);
@
\fimcodigo

However, some algorithms can ignore additional randomness passed if
the number is greater than recommended. And some algorithms can give
no guarantee of quality if you pass less randomness that
recommended. The API should inform the programmer the minimum
recommended size and maximum recommended size for the array in the
initialization. We will define these values during the description of
different algorithms:

\iniciocodigo
@<RNG Declarations@>=
@<Warn About Ideal Seed Size@>
@
\fimcodigo

The second function in our API is the one that returns a new 64-bit
random number after each invocation:

\iniciocodigo
@<RNG Declarations@>+=
uint64_t _Wrand(struct _Wrng *);
@
\fimcodigo

And the last function will finalize the random number generator. It
receives a pointer for the dis-allocation function or NULL. If the
function gets NULL as first argument, the random number generator will
not be dis-allocated (some memory managers could use some kind of
garbage collection instead of a \monoespaco{free} function). But even
if not dis-allocated, the random number generator should not be used
after being finalized with this function:

\iniciocodigo
@<RNG Declarations@>+=
bool _Wdestroy_rng(void (*free)(void *), struct _Wrng *);
@
\fimcodigo

\subsecao{1.4. Supporting Threads}

The code in this section will ensure that more than one thread can use
our random number generator after its initialization. First we need
the headers for thread support where this is supported:

\iniciocodigo
@<Including Headers@>+=
#if defined(__unix__) || defined(__APPLE__)
#include <pthread.h>
#endif
#if defined(_WIN32)
#include <windows.h>
#endif
@
\fimcodigo

Each random number struct will have a mutex
(inside \monoespaco{struct \_Wrng}):


\iniciocodigo
@<Mutex Declaration@>=
#if defined(__unix__) || defined(__APPLE__)
pthread_mutex_t mutex;
#endif
#if defined(_WIN32)
CRITICAL_SECTION mutex;
#endif
@
\fimcodigo

If we have a pointer for \monoespaco{struct \_Wrng}
 called \monoespaco{rng}, we can initialize the mutex with:

\iniciocodigo
@<Initializing Mutex@>=
#if defined(__unix__) || defined(__APPLE__)
pthread_mutex_init(&(rng -> mutex), NULL);
#endif
#if defined(_WIN32)
InitializeCriticalSection(&(rng -> mutex));
#endif
@
\fimcodigo

Before entering the critical section in code with this mutex, we do
the following:

\iniciocodigo
@<Mutex:WAIT@>=
#if defined(__unix__) || defined(__APPLE__)
pthread_mutex_lock(&(rng -> mutex));
#endif
#if defined(_WIN32)
EnterCriticalSection(&(rng -> mutex));
#endif
@
\fimcodigo

And after the critical section, we release the mutex with:

\iniciocodigo
@<Mutex:SIGNAL@>=
#if defined(__unix__) || defined(__APPLE__)
pthread_mutex_unlock(&(rng -> mutex));
#endif
#if defined(_WIN32)
LeaveCriticalSection(&(rng -> mutex));
#endif
@
\fimcodigo

The main reason for a function to finalize the generator, in our case,
the function \monoespaco{\_Wdestroy\_rng}, is the need to finalize the
mutex. And free allocated memory if needed. As there is no other
reason, here we already know the necessary to describe the generator
finalization. And we can define the finalization function:

\iniciocodigo
@<Definition: \_Wdestroy\_rng@>=
bool _Wdestroy_rng(void (*free)(void *), struct _Wrng *rng){
  bool ret;
#if defined(__unix__) || defined(__APPLE__)
  ret = pthread_mutex_destroy(&(rng -> mutex));
#elif defined(_WIN32)
  DeleteCriticalSection(&(rng -> mutex));
  ret = true; // According with the documentation, this always succeeds
#endif
  if(free != NULL)
    free(rng);
  return ret;
}
@
\fimcodigo

\secao{2. Random Number Generator Algorithms}

\subsecao{2.1. Linear Congruent Generator (LGC)}

A Linear Congruent Generator (LCG) is a generator of sequences $(x_0,
x_1, \ldots))$ such that for given integer values $a$, $c$ and $m$:

$$
x_{i+1}=ax_i+c\; (mod\; m)
$$

The choice for the parameters $a$, $c$ and $m$ determines the quality
of the generator. As we want a generator for 64-bit numbers, we will
choose $m=2^{64}$. This way, we do not need to use explicitly the
modulo operation.

About the multiplier $a$, we will use the value 0xfa346cbfd5890825
because this is one of the values suggested in the article [Steele,
2021] that examined the quality of different parameters for this kind
of generator.

About the value $c$, we need to ensure that it will be an odd integer
(because it need to be prime in relation to $m$). Other than this, we
can choose any possible odd number. The initial value also need to be
an odd integer.

To ensure this, we can use the first 64 bits of our seed to initialize
$x_0$ after force the least significant bit to 1. If we have more
64-bit numbers in the seed, we can use it as our $c$, also setting to
1 the least significance bit. Otherwise, we just use $c=1$.


This means that the recommended size for our seed is 1 or 2 numbers
with 64 bits:

\iniciocodigo
@<Warn About Ideal Seed Size@>=
#ifdef W_RNG_LCG
#define _W_RNG_MINIMUM_RECOMMENDED_SEED_SIZE  1
#define _W_RNG_MAXIMUM_RECOMMENDED_SEED_SIZE  2
#endif
@
\fimcodigo

The struct for our generator is very simple. We need to store only the
last generated value and the constant $c$:

\iniciocodigo
@<RNG Structs@>=
#ifdef W_RNG_LCG
struct _Wrng{
  uint64_t last_value, c;
  @<Mutex Declaration@>
};
#endif
@
\fimcodigo

And we initialize this with the following code:

\iniciocodigo
@<Definition: \_Wcreate\_rng@>=
#ifdef W_RNG_LCG
struct _Wrng *_Wcreate_rng(void *(*allocator)(size_t), size_t size,
                           uint64_t *seed){
  struct _Wrng *rng = (struct _Wrng *) allocator(sizeof(struct _Wrng));
  if(rng != NULL){
    // If have a seed, use it, otherwise pick this constant chosen at random:
    rng -> last_value = ((size > 0)?(seed[0]):(0x1c3b9d10b1d41adc));
    rng -> c = ((size > 1)?(seed[1]):(1));
    rng -> last_value |= (1u);
    rng -> c |= (1u);
    @<Initializing Mutex@>
  }
  return rng;
}
#endif
@
\fimcodigo

And the function that returns new random values is:

\iniciocodigo
@<Definition: \_Wrand@>=
#ifdef W_RNG_LCG
uint64_t _Wrand(struct _Wrng *rng){
  uint64_t ret;
  @<Mutex:WAIT@>
  rng -> last_value = 0xfa346cbfd5890825 * rng -> last_value + rng -> c;
  ret = rng -> last_value;
  @<Mutex:SIGNAL@>
  return ret;
}
#endif
@
\fimcodigo

The expected period for this generator is $2^{64}$. After this, the
values will repeat themselves.

\subsecao{2.2. SFMT}

The generator SFMT is the ``SIMD-Oriented Fast Mersenne Twister'',
proposed first at [Saito, 2006]. Its name is because this generator
period are very big mersenne primes (with the form $2^n-1$). The most
used implementation of this algorithm have periods of $2^{199937}-1$
different values without repetitions. This is more than enough (and
even an exaggeration) for any use case. And ``SIMD'' is an instruction
set for vectorizing 128-bit integers that more recent CPUs
support. These instructions are expected to make the algorithm run
faster, if the compiler is smart enough to use it, or if we use them
explicitly in assembly.

The generator works using the following recursion where each value
$x_i$ have 128 bits:

$$
X_{i+n}=g(X_i, \ldots, X_{i+n-1})
$$

In our case, we use $n=156$. This means that we always must store and
keep in the memory the last 156 values of the sequence, each one with
128 bits. This explains why this generator needs much more memory when
compared to others.

The function $g$ works in this case concatenating all the input values
in a binary vector with $156 \times 128$ values. Then, it multiplies a
binary matrix by this binary vector. The binary matrix has $156 \times
128$ columns and 128 lines. Except that instead of working with the
usual field of rational or real numbers, we are working with the
finite field GF(2) composed only by two numbers: 0 and 1 (the vector
and the matrix are binaries). What we means by ``multiplication'' in
this context is the logical AND and the addition is the logical
XOR. And we choose a convenient matrix such that we do not need to
store and multiply it explicitly. We can represent the multiplication
of the matrix by a vector using a compact number of operations.

In the struct for our generator, we need to store all the previous
necessary values and also an offset, that stores the position of the
next 64-bit value to be returned. If the offset is even, we assume
that we need to generate a new value. In the even case, we generate a
new 128 bit value and return the first 64 bits. In the odd case, we
still have the remaining 64 bits of the last value and do not need to
generate a new one.

\iniciocodigo
@<RNG Structs@>=
#ifdef W_RNG_MERSENNE_TWISTER
struct _Wrng{
  char w[128 * 156 / 8]; // Todos N valores gerados, cada um com _W bits
  int offset;          // Index for the next 64-bit value to be returned
  @<Mutex Declaration@>
};
#endif
@
\fimcodigo

The language C, following the ISO standards, do not support variables
that always have 128 bits. However, some compilers like GCC and Clang
support this as an extension. Using the extension, we can compute the
function $g$ to obtain the next element from our random sequence using
the following operations:

\iniciocodigo
@<SFMT: Compute next element@>=
#ifdef __SIZEOF_INT128__
unsigned __int128 result, tmp;
uint32_t aux[4];
int i, index = rng -> offset / 2;
result = ((unsigned __int128 *) (rng -> w))[index];
result = result << 8;
result = result ^ ((unsigned __int128 *) (rng -> w))[index];
i = (index + 122) % 156;
aux[0] = *(((uint32_t *) &(((unsigned __int128 *) (rng -> w))[i])) + 0);
aux[1] = *(((uint32_t *) &(((unsigned __int128 *) (rng -> w))[i])) + 1);
aux[2] = *(((uint32_t *) &(((unsigned __int128 *) (rng -> w))[i])) + 2);
aux[3] = *(((uint32_t *) &(((unsigned __int128 *) (rng -> w))[i])) + 3);
aux[0] = (aux[0] >> 11) & 0xDFFFFFEF; // 0xBFFFFFF6
aux[1] = (aux[1] >> 11) & 0xDDFECB7F; // 0xBFFAFFFF
aux[2] = (aux[2] >> 11) & 0xBFFAFFFF; // 0xDDFECB7F
aux[3] = (aux[3] >> 11) & 0xBFFFFFF6; // 0xDFFFFFEF
memcpy(&tmp, aux, 16);
result = result ^ tmp;
i = (index + 156 - 2) % 156;
result = result ^ (((unsigned __int128 *) (rng -> w))[i] >> 8);
i = (index + 156 - 1) % 156;
aux[0] = *(((uint32_t *) &(((unsigned __int128 *) (rng -> w))[i])) + 0);
aux[1] = *(((uint32_t *) &(((unsigned __int128 *) (rng -> w))[i])) + 1);
aux[2] = *(((uint32_t *) &(((unsigned __int128 *) (rng -> w))[i])) + 2);
aux[3] = *(((uint32_t *) &(((unsigned __int128 *) (rng -> w))[i])) + 3);
aux[0] = (aux[0] << 18);
aux[1] = (aux[1] << 18);
aux[2] = (aux[2] << 18);
aux[3] = (aux[3] << 18);
memcpy(&tmp, aux, 16);
result = result ^ tmp;
((unsigned __int128 *) (rng -> w))[index] = result;
#else
#error "Mersenne Twister unsupported without 128 bit integer support." 
#endif
@
\fimcodigo

Notice that if we are using a compiler without support for fixed-size
variables with 128 bits, we return an error and refuse to compile.

The code above represents the multiplication between a matrix and a
vector. Notice that the represented matrix is very sparse. If we are
generating value $x_i$, we compute it using only the previous values
$x_{i-1}$, $x_{i-2}$, $x_{i-156}$ and $x_{i-34}$.

Knowing how to generate the next $x_i$ with 128 bits, the code of our
function that returns the next 64-bit pseudo-random value is:

\iniciocodigo
@<Definition: \_Wrand@>=
#ifdef W_RNG_MERSENNE_TWISTER
uint64_t _Wrand(struct _Wrng *rng){
  uint64_t ret;
  @<Mutex:WAIT@>
  if(rng -> offset % 2 == 0){
    @<SFMT: Compute next element@>
  }
  ret = ((uint64_t *) (rng -> w))[rng -> offset];
  rng -> offset = (rng -> offset + 1) % (128 * 156 / 64);
  @<Mutex:SIGNAL@>
  return ret;
}
#endif
@
\fimcodigo

To initialize this generator we need to fill our previous 156 values
obtaining it from our seed. Even with a smaller seed, we stretch it
with a mini-random number generator just to fill the initial vector of
previous values. The first two components of this mini-pseudo-random
number generator are these functions:

\iniciocodigo
@<Definition: \_Wcreate\_rng@>=
#ifdef W_RNG_MERSENNE_TWISTER
static uint32_t f1(uint32_t x){
  return (x ^ (x >> 27)) * (uint32_t) 1664525UL;
}
static uint32_t f2(uint32_t x){
  return (x ^ (x >> 27)) * (uint32_t) 1566083941UL;
}
#endif
@
\fimcodigo

How we will initialize the struct is defined by the reference
implementation. Therefore, like in the reference, we use a
mini-generator based on 32-bit numbers to fill our initial state:

\iniciocodigo
@<Definition: \_Wcreate\_rng@>=
#ifdef W_RNG_MERSENNE_TWISTER
struct _Wrng *_Wcreate_rng(void *(*allocator)(size_t), size_t size,
                           uint64_t *seed){
  struct _Wrng *rng = (struct _Wrng *) allocator(sizeof(struct _Wrng));
  if(rng != NULL){
    uint32_t *dst = (uint32_t *) (rng -> w), *origin = (uint32_t *) seed;
    size_t size_dst = 128 * 156 / 32, size_origin = size * 2;
    int count, r, i, j, mid = 306, lag = 11;
    // Preenchemos inicialmente tudo com 0x8b:
    memset(rng -> w, 0x8b, 128 * 156 / 8);
    count = ((size_origin + 1 >= size_dst)?(size_origin + 1):(size_dst));
    r = f1(dst[0] ^ dst[mid] ^ dst[size_dst - 1]);
    dst[mid] += r;
    r += size * 2;
    dst[mid + 11] += r;
    dst[0] = r;
    count --;
    for(i = 1, j = 0; j < count && j  < size_origin; j ++){
      r = f1(dst[i] ^  dst[(i + mid) % size_dst] ^
             dst[(size_dst - 1 + i) % size_dst]);
      dst[(i + mid) % size_dst] += r;
      r += origin[j] + i;
      dst[(i + mid + 11) % size_dst] += r;
      dst[i] = r;
      i = (i + 1) % size_dst;
    }
    for(; j < count; j++){
      r = f1(dst[i] ^ dst[(i + mid) % size_dst] ^
             dst[(i + size_dst - 1) % size_dst]);
      dst[(i + mid) % size_dst] += r;
      r += i;
      dst[(i + mid + 11) % size_dst] += r;
      dst[i] = r;
      i = (i + 1) % size_dst;
    }
    for (j = 0; j < size_dst; j++) {
      r = f2(dst[i] + dst[(i + mid) % size_dst] +
             dst[(i + size_dst - 1) % size_dst]);
      dst[(i + mid) % size_dst] ^= r;
      r -= i;
      dst[(i + mid + lag) % size_dst] ^= r;
      dst[i] = r;
      i = (i + 1) % size_dst;
    }
    rng -> offset = 0;
    @<SFMT: Ensure Period@>
    @<Initializing Mutex@>
  }
  return rng;
}
#endif
@
\fimcodigo

Notice how before returning (and initializing the mutex) we check if
the generator will have the desired period of $2^{19937}-1$. We ensure
this with a parity check. If parity is wrong, we change a single bit
to ensure the correct period:

\iniciocodigo
@<SFMT: Ensure Period@>=
{
  // Bit mask to be checked
  uint32_t parity = (dst[0] & 0x00000001U) ^ (dst[3] & 0xc98e126aU);
  // Parity check:
  for (i = 16; i > 0; i >>= 1)
    parity ^= parity >> i;
  parity = parity & 1;
  if(parity != 1)
    dst[0] = dst[0] ^ 1;
}
@
\fimcodigo

Finally, we need to document the recommended seed size for this
algorithm. We can perfectly deal with a seed of just 64 bits. In fact,
in the reference implementation, it is possible to initialize with
just 32 bits. If we pass a bigger seed, in the initialization we use
the additional bits to compute the initial state. However, as we need
to fill a vector with 19968 bits, providing more than 312 numbers with
64 bits will be redundant:

\iniciocodigo
@<Warn About Ideal Seed Size@>=
#ifdef W_RNG_MERSENNE_TWISTER
#define _W_RNG_MINIMUM_RECOMMENDED_SEED_SIZE  1
#define _W_RNG_MAXIMUM_RECOMMENDED_SEED_SIZE  312
#endif
@
\fimcodigo

\subsecao{2.3. SplitMix64}

SplitMix is a random number generator designed to be splittable. This
means that exist an algorithm, that given the RNG structure, produce a
new RNG structure such that both structures will generate unrelated
random numbers with the same quality and without needing to provide a
new seed.

But the real reason for providing the SplitMix algorithm here is not
this property. We are not planning to support in our API splittable
generators. Our real interest is in the different nature of this
generator compared with others and the fact that it can be initialized
with only 64-bit seeds.

Recall that in the end of Mersenne Twister section, we needed to
consider the case of how to fill our initial stat with 312 bits if we
get fewer random bits in the seed. For Mersenne Twister we used
exactly the method suggested in the reference implementation, which
consists in using another specific RNG just to fill the initial
state. In these cases, we need a RNG that works with shorter seeds and
papers like [Matsumoto, 2007] shows the importance of choosing
radically different RNGs for this purpose, to avoid correlation of
sequences generated with similar seeds.

Ignoring the splitting mechanism, SplitMix is very simple. It was
created from the idea of having a state with 64 bits, and each time we
need a new value, we update the state to a new value and return the
result of a hash function over the RNG state.

Except that for performance reasons, instead of using a full hash
function, we use a bit mixer: a component that is part of some hash
functions and is responsible for avoid correlations between similar
input values. The bit mixer used for SplitMix64 is the one defined
originally for a hash function called MurmurHash3:

\iniciocodigo
@<SplitMix: Bit Mixer@>=
{
  uint64_t tmp = *state;
  tmp = (tmp ^ (tmp >> 33)) * 0xff51afd7ed558ccdl;
  tmp = (tmp ^ (tmp >> 33)) * 0xc4ceb9fe1a85ec53l;
  ret = tmp ^ (tmp >> 33);
} 
@
\fimcodigo

And before computing the above bit mixer, we also need to update the
RNG state with the following line of code where \monoespaco{gamma} is
an odd constant:

\iniciocodigo
@<SplitMix: Update State@>=
{
  *state +=  gamma;
} 
@
\fimcodigo

The final construction of the function that returns a new
pseudo-random number is:

\iniciocodigo
@<Auxiliary Functions@>=
#if defined(W_RNG_SPLITMIX) || defined(W_RNG_XOSHIRO) || defined(W_RNG_PCG) || \
  defined(W_RNG_CHACHA20)
static uint64_t splitmix_next(uint64_t *state, uint64_t gamma){
  uint64_t ret;
  @<SplitMix: Bit Mixer@>
  @<SplitMix: Update State@>
  return ret;
}
#endif
@
\fimcodigo

If we are using SplitMix not to initialize another RNG state, but as
our real RNG, we will store this state and gamma value in the RNG
struct:

\iniciocodigo
@<RNG Structs@>=
#ifdef W_RNG_SPLITMIX
struct _Wrng{
  uint64_t state, gamma;
  @<Mutex Declaration@>
};
#endif
@
\fimcodigo

And the API function to generate a new pseudo-random number is:

\iniciocodigo
@<Definition: \_Wrand@>=
#ifdef W_RNG_SPLITMIX
uint64_t _Wrand(struct _Wrng *rng){
  uint64_t ret;
  @<Mutex:WAIT@>
  ret = splitmix_next(&(rng -> state), rng -> gamma);
  @<Mutex:SIGNAL@>
  return ret;
}
#endif
@
\fimcodigo

We can initialize the RNG structure using directly the seed to choose
the initial state and, if possible, the gamma value. We just need to
ensure that the gamma value is odd to have the expected period of
$2^{64}$:

\iniciocodigo
@<Definition: \_Wcreate\_rng@>=
#ifdef W_RNG_SPLITMIX
struct _Wrng *_Wcreate_rng(void *(*allocator)(size_t), size_t size,
                           uint64_t *seed){
  int i;
  struct _Wrng *rng = (struct _Wrng *) allocator(sizeof(struct _Wrng));
  if(rng != NULL){
    if(size < 1)
      rng -> state = 0x32147198b5436569;
    else
      rng -> state = seed[0];
    if(size < 2)
      rng -> gamma = 0x9e3779b97f4a7c15;
    else
      rng -> gamma = (seed[1] | 1);
    @<Initializing Mutex@>
  }
  return rng;
}
#endif
@
\fimcodigo

This means that when using the SplitMix generator, we need a seed with
1 or 2 64-bit numbers:

\iniciocodigo
@<Warn About Ideal Seed Size@>=
#ifdef W_RNG_SPLITMIX
#define _W_RNG_MINIMUM_RECOMMENDED_SEED_SIZE  1
#define _W_RNG_MAXIMUM_RECOMMENDED_SEED_SIZE  2
#endif
@
\fimcodigo


\subsecao{2.4. Xoshiro256**}

Xoshiro and Mersenne Twister have some similarities. Both use past
outputs directly to produce new inputs in a matrix multiplication. But
contrary to Mersenne Twister, Xoshiro computes its sequences as
256-bit numbers, not 128-bit numbers. And instead of generating the
next input storing all previous 156 outputs, Xoshiro stores only the
last 256-number generated, requiring much less memory.

In the RNG struct we store the 256-number as a sequence of four 64-bit
number:

\iniciocodigo
@<RNG Structs@>=
#ifdef W_RNG_XOSHIRO
struct _Wrng{
  uint64_t w[4];   // Valores de estado
  @<Mutex Declaration@>
};
#endif
@
\fimcodigo

Like Mersenne Twister, to produce the next value, we multiply it by a
matrix and use operations in GF(2): we use the operator AND in the
bits instead of multiplication and operator XOR instead of sum. And we
also use a matrix that at the same time produce good results and is
convenient enough to not be explicitly stored. Instead, the matrix
multiplication can be written using only a few fast operations, like
combining AND, XOR and SHIFT.

In fact, the matrix multiplication to update the state is represented
using only this code with 7 lines:

\iniciocodigo
@<Xoshiro: Matrix Multiplication@>=
{
  uint64_t t = rng -> w[1] << 17;
  rng -> w[2] ^= rng -> w[0];
  rng -> w[3] ^= rng -> w[1];
  rng -> w[1] ^= rng -> w[2];
  rng -> w[0] ^= rng -> w[3];
  rng -> w[2] ^= t;
  rng -> w[3] = ((rng -> w[3] << 45) | (rng -> w[3] >> 19));
}
@
\fimcodigo

But as we generate pseudo-random numbers using only a linear
transformation and as our state is very small (contrary to Mersenne
Twister), we need to use an additional trick to avoid failing in
statistical tests. Instead of returning the real 256-bit number
produced by our generator, we pass part of this value in a scrambler
to produce the value that will be returned, masking the relationship
between this returned value and the previous one:

\iniciocodigo
@<Xoshiro: Scrambler@>=
{
  uint64_t tmp = rng -> w[1] * 5;
  ret = ((tmp << 7) | (tmp >> 57)) * 9;
} 
@
\fimcodigo

And finally, our function that returns the next pseudo-random value
first uses the scrambler to compute the returned
value \monoespaco{ret} using the current state, and then generates the
next state using the matrix multiplication:

\iniciocodigo
@<Definition: \_Wrand@>=
#ifdef W_RNG_XOSHIRO
uint64_t _Wrand(struct _Wrng *rng){
  uint64_t ret;
  @<Mutex:WAIT@>
  @<Xoshiro: Scrambler@>
  @<Xoshiro: Matrix Multiplication@>
  @<Mutex:SIGNAL@>
  return ret;
}
#endif
@
\fimcodigo

Finally, we need to initialize Xoshiro**. If the seed have 256 bits or
more, we copy the bits from the seed to the RNG state. If we have
less, we copy what we have and use the last 64 bits to initialize
SplitMix64 state using the default gamma value and produce the
remaining bits using the SplitMix generator:

\iniciocodigo
@<Definition: \_Wcreate\_rng@>=
#ifdef W_RNG_XOSHIRO
struct _Wrng *_Wcreate_rng(void *(*allocator)(size_t), size_t size,
                           uint64_t *seed){
  int i;
  struct _Wrng *rng = (struct _Wrng *) allocator(sizeof(struct _Wrng));
  if(rng != NULL){
    if(size >= 4){
      for(i = 0; i < 4; i ++)
        rng -> w[i] = seed[i];
    }
    else{
      uint64_t state = 0x32147198b5436569, gamma = 0x9e3779b97f4a7c15;
      for(i = 0; i < size - 1; i ++)
        rng -> w[i] = seed[i];
      if(size > 1)
        state = seed[i];
      for(; i < 4; i ++)
        rng -> w[i] = splitmix_next(&state, gamma);
    }                           
    @<Initializing Mutex@>
  }
  return rng;
}
#endif
@
\fimcodigo

Because of this, we recommend that Xoshiro gets a seed with between
one and four 64-bit elements:

\iniciocodigo
@<Warn About Ideal Seed Size@>=
#ifdef W_RNG_XOSHIRO
#define _W_RNG_MINIMUM_RECOMMENDED_SEED_SIZE  1
#define _W_RNG_MAXIMUM_RECOMMENDED_SEED_SIZE  4
#endif
@
\fimcodigo


\subsecao{2.5. PCG (\italico{Permuted Congruential Generator})}

The PCG random number generator is similar to SplitMix. It uses a
simple operation to change the internal state. But instead of
returning the internal state as the next number in the sequence, it
applies a more complicated function to the internal state to produce
the output.

The simple transformation in the state, in SplitMix means adding a
constant to the internal state. But in PCG, we update the internal
state applying a linear congruent generator (LCG) like the one from
subsection 2.1.

However, that linear congruent generator was defined using 64
bits. The internal state for PCG have 128 bits and therefore, we use a
128-bit  LCG generator to update the internal state.

Our RNG struct is defined as below:

\iniciocodigo
@<RNG Structs@>=
#ifdef W_RNG_PCG
#ifdef __SIZEOF_INT128__
struct _Wrng{
  unsigned __int128 state;
  unsigned __int128 increment; // Almost an odd number, like in any LCG
  @<Mutex Declaration@>
};
#else
#error "PCG unsupported without 128 bit integer support."
#endif
#endif
@
\fimcodigo

We update the internal state performing the LCG operation of
multiplying by a constant multiplier carefully chosen and adding our
odd increment to the result:

\iniciocodigo
@<PCG: Update State@>=
{
  unsigned __int128 multiplier;
  multiplier = 2549297995355413924ULL;
  multiplier = multiplier << 64;
  multiplier += 4865540595714422341ULL;
  rng -> state = rng -> state * multiplier + rng -> increment;
}
@
\fimcodigo

But like in SplitMix, instead of returning this state as the next
number in the sequence, we pass this to some sort of hash function or
bit scrambler. But in PCG case, what we do is combine the 128 bits in
64 bits using a XOR operation an then we apply a permutation over the
result:

\iniciocodigo
@<PCG: Permutation@>=
{
  uint64_t xorshifted, rot;
  xorshifted = (((uint64_t)(rng -> state >> 64u)) ^ ((uint64_t) rng -> state));
  rot = rng -> state >> 122u;
  ret = (xorshifted >> rot) | (xorshifted << ((-rot) & 63));
}
@
\fimcodigo

And finally, the complete function to generate the next random number
combines these operations as below:

\iniciocodigo
@<Definition: \_Wrand@>=
#ifdef W_RNG_PCG
uint64_t _Wrand(struct _Wrng *rng){
  uint64_t ret;
  @<Mutex:WAIT@>
  @<PCG: Update State@>
  @<PCG: Permutation@>
  @<Mutex:SIGNAL@>
  return ret;
}
#endif
@
\fimcodigo

What remains to be done is initialize the initial state. If we pass a
big seed, with 256 bits or more, we could use it directly to fill the
internal state. Instead, we mix them in a more complicated way to
match the reference implementation that tries to deal with seeds that
are not much random.

If we pass a seed with at least 128 bits, we use it directly to
initialize the initial state and we set the increment of our internal
LCG as 1. If we have 192 bits, we can use them to initialize to
different values in our increment. And if we have less than 128 bits,
we use SplitMix to stretch our initial random bits to 128. The
initialization function is defined as below:

\iniciocodigo
@<Definition: \_Wcreate\_rng@>=
#ifdef W_RNG_PCG
struct _Wrng *_Wcreate_rng(void *(*allocator)(size_t), size_t size,
                           uint64_t *seed){
  struct _Wrng *rng = (struct _Wrng *) allocator(sizeof(struct _Wrng));
  if(rng != NULL){
    if(size >= 4){
      unsigned __int128 multiplier;
      multiplier = 2549297995355413924ULL;
      multiplier = multiplier << 64;
      multiplier += 4865540595714422341ULL;
      unsigned __int128 initstate = seed[0], initseq;
      initstate = initstate << 64;
      initstate += seed[1];
      initseq = seed[2];
      initseq = initseq << 64;
      initseq += seed[3];
      rng->state = 0U;
      rng -> increment = (initseq << 1) | 1;
      rng->state = rng->state * multiplier + rng -> increment;
      rng -> state += initstate;
      rng->state = rng->state * multiplier + rng -> increment;
    }
    else if(size >= 2){
      rng -> state = seed[0];
      rng -> state = (rng -> state << 64);
      rng -> state += seed[1];
      if(size == 3){
        rng -> increment = seed[2];
        rng -> increment = (rng -> increment) | 1;
      }
      else
        rng -> increment = 1;
    }
    else{
      uint64_t state, gamma = 0x9e3779b97f4a7c15;
      if(size > 0)
        state = seed[0];
      else
        state = 0x32147198b5436569;
      rng -> state = splitmix_next(&state, gamma);
      rng -> state = (rng -> state << 64);
      rng -> state += splitmix_next(&state, gamma);
      rng -> increment = 1;
    }                           
    @<Initializing Mutex@>
  }
  return rng;
}
#endif
@
\fimcodigo

This means that our recommended seed size is a number between 2 and 4
to avoid needing to stretch the initial randomness with SplitMix:

\iniciocodigo
@<Warn About Ideal Seed Size@>=
#ifdef W_RNG_PCG
#define _W_RNG_MINIMUM_RECOMMENDED_SEED_SIZE  2
#define _W_RNG_MAXIMUM_RECOMMENDED_SEED_SIZE  4
#endif
@
\fimcodigo


\subsecao{2.6. ChaCha20}

ChaCha20 is a cryptographic secure random number generator, but
this affirmation is only true if it is used correctly: the seed must
have a sufficiently large size, it should be chosen in a completely
random and uniform way and it never should be utilized again after
the use. Our API does not guarantee these things, here we are not
concerned with cryptographic ally secure random number generators. For
us, ChaCha20 will be only another RNG like the previous ones. The
difference is that it is expected to have a greater quality at the
cost of being slower.

This algorithm has as first component a padding function that gets as
input 384 bits that will be treated as a sequence of 64-bit
numbers. The padding function outputs 512 bits that should be
interpreted as a sequence of 32-bit numbers that form a $4\times 4$
matrix. Even if we do not store them as a matrix.

The padding function is:

@<ChaCha20: Padding@>=
#ifdef W_RNG_CHACHA20
static void chacha_padding(uint64_t input[6], uint32_t output[16]){
  int i, j;
  output[0] = ('e' << 24) + ('x' << 16) + ('p' << 8) + 'a';
  output[1] = ('n' << 24) + ('d' << 16) + (' ' << 8) + '3';
  output[2] = ('2' << 24) + ('-' << 16) + ('b' << 8) + 'y';
  output[3] = ('t' << 24) + ('e' << 16) + (' ' << 8) + 'k';
  for(j=4, i = 0; i < 6; i ++, j += 2){
    output[j] = (input[i] / 4294967296llu);
    output[j+1] = input[i] % 4294967296llu;
  }
}
#endif
@

In the input for this padding function, the first 4 values are taken
from the initial seed, The next value is a counter initialized as 0
and incremented each time the generator produces a new set of
numbers. The last value is a nonce, a number that should be used only
once.

The next component is the function below that gets 4 values as 32-bit
numbers and modify them doing a sequence of operations:

@<ChaCha20: QuarterRound@>=
#ifdef W_RNG_CHACHA20
void quarter_round(uint32_t *a, uint32_t *b, uint32_t *c, uint32_t *d){
  *a = *a + *b;
  *d = *d ^ *a;
  *d = (*d << 16) | (*d >> 16);
  *c = *c + *d;
  *b = *b ^ *c;
  *b = (*b << 12) | (*b >> 20);
  *a = *a + *b;
  *d = *d ^ *a;
  *d = (*d << 8) | (*d >> 24);
  *c = *c + *d;
  *b = *b ^ *c;
  *b = (*b << 7) | (*b >> 25);
}
#endif
@

The previous operation is used in the next permutation function that
gets as input 16 numbers with 32 bits and transform them. Notice that
in each iteration, the entire input is completely changed twice. As
there are 10 iterations, we make 20 transformations in our input.

\iniciocodigo
@<ChaCha20: Permutação@>=
#ifdef W_RNG_CHACHA20
void chacha_permutation(uint32_t elements[16]){
  int i;
  for(i = 0; i < 10; i ++){
    quarter_round(&elements[0], &elements[4], &elements[8], &elements[12]);
    quarter_round(&elements[1], &elements[5], &elements[9], &elements[13]);
    quarter_round(&elements[2], &elements[6], &elements[10], &elements[14]);
    quarter_round(&elements[3], &elements[7], &elements[11], &elements[15]);
    quarter_round(&elements[0], &elements[5], &elements[10], &elements[15]);
    quarter_round(&elements[1], &elements[6], &elements[11], &elements[12]);
    quarter_round(&elements[2], &elements[7], &elements[8], &elements[13]);
    quarter_round(&elements[3], &elements[4], &elements[9], &elements[14]);
  }
}
#endif
@
\fimcodigo

With these prerequisites we can begin the definition of our RNG struct
and functions for our API. Our RNG should store a vector with 6
elements, each one a 64-bit number (that will be used in the padding
function to produce the next numbers), the last 512 bits generated by
the RNG (if they exist) and an index to know which previously
generated value should be returned. Our struct is:

\iniciocodigo
@<RNG Structs@>=
#ifdef W_RNG_CHACHA20
struct _Wrng{
  uint64_t array[6];
  uint32_t generated_values[16];
  int index;
  @<Mutex Declaration@>
};
#endif
@
\fimcodigo

When this struct is allocated, the first 4 values in our 6-element
array are copied from the seed, the next one is filled with 0 and the
last one also is taken from the seed, but if it do not exist, we can
use 0 (this is the nonce that identify the produced sequence for a
given initial seed). This means that our seed in our API should have
between 4 and 5 values:

\iniciocodigo
@<Warn About Ideal Seed Size@>=
#ifdef W_RNG_CHACHA20
#define _W_RNG_MINIMUM_RECOMMENDED_SEED_SIZE  4
#define _W_RNG_MAXIMUM_RECOMMENDED_SEED_SIZE  5
#endif
@
\fimcodigo

If we have less values we will use the SplitMix algorithm to produce
the remaining values, except for the nonce that we can keep with
0. Our initialization function is given below.

\iniciocodigo
@<Definition: \_Wcreate\_rng@>=
#ifdef W_RNG_CHACHA20
struct _Wrng *_Wcreate_rng(void *(*allocator)(size_t), size_t size,
                           uint64_t *seed){
  int i, j;
  struct _Wrng *rng = (struct _Wrng *) allocator(sizeof(struct _Wrng));
  if(rng != NULL){
    for(i = 0; i < 4; i ++){
      if(i < size - 1 || size >= 4)
        rng -> array[i] = seed[i];
      else{
        // Using SplitMix to fill the remaining values
        uint64_t state, gamma = 0x9e3779b97f4a7c15;
        if(size > 0)
          state = seed[i];
        else
          state = 0x32147198b5436569;
        for(j = i; j < 4; j ++)
          rng -> array[j] = splitmix_next(&state, gamma);
        break;        
      }
    }
    rng -> array[4] = 0;
    if(size > 4)
      rng -> array[5] = seed[4];
    else
      rng -> array[5] = 0;
    rng -> index = 0;
    @<Initializing Mutex@>
  }
  return rng;
}
#endif
@
\fimcodigo

To produce new values, first we need to check if we still have
generated bits that still were not returned. Each time ChaCha20
produces new values, it produces 16 numbers with 32 bits. This means
that we can use it as 8 numbers with 64 bits. When we do not need to
produce new values, we proceed as below:

\iniciocodigo
@<Definition: \_Wrand@>=
#ifdef W_RNG_CHACHA20
@<ChaCha20: Padding@>
@<ChaCha20: QuarterRound@>
@<ChaCha20: Permutação@>
uint64_t _Wrand(struct _Wrng *rng){
  uint64_t ret;
  @<Mutex:WAIT@>
  if(rng -> index % 16 == 0){
    rng -> index = 0;
    @<ChaCha20: Generating new values@>
    rng -> array[4] ++;
  }
  ret = rng -> generated_values[rng -> index];
  ret = ret << 32;
  ret += rng -> generated_values[rng -> index + 1];
  rng -> index += 2; 
  @<Mutex:SIGNAL@>
  return ret;
}
#endif
@
\fimcodigo

And finally, when we really need to produce new values, we use the
following code:

@<ChaCha20: Generating new values@>=
{
  int i;
  uint32_t padded_array[16], copied_array[16];
  chacha_padding(rng -> array, padded_array);
  for(i = 0; i < 16; i ++)
    copied_array[i] = padded_array[i];
  chacha_permutation(padded_array);
  for(i = 0; i < 16; i ++)
    rng -> generated_values[i] = padded_array[i] + copied_array[i];
}
@

\subsecao{2.6. Final Structure in File}

The file with the source code of our functions will have the following
structure:

\iniciocodigo
@(src/random.c@>=
@<Including Headers@>
#include "random.h"
#include <string.h> // memcpy
@<Auxiliary Functions@>
@<Definition: \_Wrand@>
@<Definition: \_Wcreate\_rng@>
@<Definition: \_Wdestroy\_rng@>
@
\fimcodigo

\secao{3. Quality Tests}

To test the quality of all algorithms defined in this work and compare
them, we implemented the empirical tests proposed in [Knuth,
1998]. They are our initial set of tests, but more tests could be
implemented and added to this work later.

A caveat is that lots of tests suggested by Knuth assume that we are
generating random numbers between 0 and $n$ or real numbers between 0
and 1. But for many tests this gives more weight to the more
significant bits of a number and we want to give the same weight for
all bits. Because of this, we treat the sequence generated by our RNG
as a stream of bits, choosing different lengths of bits in each test
to generate our numbers.

In most tests we apply the chi-square test as suggested in [Knuth,
1998] to test the quality and randomness of our results. We repeat
each chi-square test three times and consider the test a failure if in
any of them we obtain a very improbable result with a chance lesser
than 1\% of happening. If we get two times a slightly improbable
result (probability of happening: 5\%) we also consider the test a
failure. We repeat this triple test 1000 times. The expected result is
a success of about 92.34\% in all tests.

For all tests we use exactly the same seed with randomly chosen
numbers. The numbers are:
(\monoespaco{32147198b5436569}, \monoespaco{260287febfeb34e9},
\monoespaco{0b6cc94a91a265e4},
\monoespaco{c6a109c50dd52f1b}, \monoespaco{8298497f3992d73a}).

\subsecao{3.1. Equidistribution Test}

In the most basic test we check the equidistribution of bits 0 and
1. The success for each algorithm is:

\vbox{%A forma mais gseral de remover espaçamento entre linhas:
\baselineskip-1000pt\lineskip0pt\lineskiplimit16383.99999pt\tabskip0pt
\def\linha{\noalign{\hrule}}
\def\hidewidth{\hskip-1000pt plus 1fill}
\def\col{\hbox{\vrule height12pt depth3.5pt width0pt}}
\halign to15cm{\col#& \vrule#\tabskip=1em plus2em&
\hfil#& \vrule#& \hfil#\hfil& \vrule#&
\hfil#& \vrule#&\hfil#& \vrule#\tabskip=0pt\cr\linha
&&\omit\hidewidth Gerador\hidewidth&&\omit\hidewidth
Sucesso\hidewidth&&
\omit\hidewidth Gerador\hidewidth&&Sucesso&\cr\linha
&&SFMT&&92\%&&Xoshiro**&&91\%&\cr\linha
&&PCG&&93\%&&\monoespaco{LCG}&&93\%&\cr\linha
&&ChaCha20&&91\%&&&&&\cr\linha}}

\subsecao{3.2. Serial Test}

In the serial test we generate sequence of $d$ bits and check if each
possible combination happens in a proportion of $1/2^n$. It is
analogous to the equidistribution, but using more bits. Here we chosen
$d=15$. Our result is:

\vbox{%A forma mais gseral de remover espaçamento entre linhas:
\baselineskip-1000pt\lineskip0pt\lineskiplimit16383.99999pt\tabskip0pt
\def\linha{\noalign{\hrule}}
\def\hidewidth{\hskip-1000pt plus 1fill}
\def\col{\hbox{\vrule height12pt depth3.5pt width0pt}}
\halign to15cm{\col#& \vrule#\tabskip=1em plus2em&
\hfil#& \vrule#& \hfil#\hfil& \vrule#&
\hfil#& \vrule#&\hfil#& \vrule#\tabskip=0pt\cr\linha
&&\omit\hidewidth Gerador\hidewidth&&\omit\hidewidth
Sucesso\hidewidth&&
\omit\hidewidth Gerador\hidewidth&&Sucesso&\cr\linha
&&SFMT&&92\%&&Xoshiro**&&92\%&\cr\linha
&&PCG&&93\%&&\monoespaco{LCG}&&91\%&\cr\linha
&&ChaCha20&&93\%&&&&&\cr\linha}}

\subsecao{3.3. Gap Test}

This is a gap test for bits. We interpret the sequence generated by
the RNG as a sequence of bits and count the number of sequences of bits
0 followed by a 1. We treat them as ``gaps''.

For example, in the sequence ``100110010100001'' we get the following
gaps: ``1'' (size 0), ``001'' (size 2), ``1'' (size 0), ``001'' (size
2), ``01'' (size 1) e ``00001'' (size 4). For each RNG we obtain a
sample of $5\cdot2^{20}$ gaps and analyze the number of gaps for each
size. We compare this with the expected result using the chi-square
test. Our result is:

\vbox{%A forma mais gseral de remover espaçamento entre linhas:
\baselineskip-1000pt\lineskip0pt\lineskiplimit16383.99999pt\tabskip0pt
\def\linha{\noalign{\hrule}}
\def\hidewidth{\hskip-1000pt plus 1fill}
\def\col{\hbox{\vrule height12pt depth3.5pt width0pt}}
\halign to15cm{\col#& \vrule#\tabskip=1em plus2em&
\hfil#& \vrule#& \hfil#\hfil& \vrule#&
\hfil#& \vrule#&\hfil#& \vrule#\tabskip=0pt\cr\linha
&&\omit\hidewidth Generator\hidewidth&&\omit\hidewidth
Sucesso\hidewidth&&
\omit\hidewidth Generator\hidewidth&&Sucesso&\cr\linha
&&SFMT&&92\%&&Xoshiro**&&92\%&\cr\linha
&&PCG&&91\%&&\monoespaco{LCG}&&86\%&\cr\linha
&&ChaCha20&&91\%&&&&&\cr\linha}}

\subsecao{3.4. Poker Test}

Here we interpret our RNG sequence as sequences of numbers between 0
and 15. We generate a lot of tuples with 5 elements. Then, we count
the occurrence of the following events: all values in the tuple are
different, we found 4 equal values, we found 3 equal values, we found
a triple and a pair of equal values, we found two pairs of equal
values, we found a single pair of equal values. We measure how the
number of each of these events compare with the expected numbers.

Our result is:

\vbox{%A forma mais gseral de remover espaçamento entre linhas:
\baselineskip-1000pt\lineskip0pt\lineskiplimit16383.99999pt\tabskip0pt
\def\linha{\noalign{\hrule}}
\def\hidewidth{\hskip-1000pt plus 1fill}
\def\col{\hbox{\vrule height12pt depth3.5pt width0pt}}
\halign to15cm{\col#& \vrule#\tabskip=1em plus2em&
\hfil#& \vrule#& \hfil#\hfil& \vrule#&
\hfil#& \vrule#&\hfil#& \vrule#\tabskip=0pt\cr\linha
&&\omit\hidewidth Generator\hidewidth&&\omit\hidewidth
Sucesso\hidewidth&&
\omit\hidewidth Generator\hidewidth&&Sucesso&\cr\linha
&&SFMT&&92\%&&Xoshiro**&&93\%&\cr\linha
&&PCG&&91\%&&\monoespaco{LCG}&&92\%&\cr\linha
&&ChaCha20&&92\%&&&&&\cr\linha}}

\subsecao{3.5. Collector Test}

Like in the previous test we measure the equidistributivity of a hand
of poker, here we measure how random is the order in which the cards
appear. We interpret our RNG sequence as a sequence of numbers between
0 and 15. We keep generating values until we get all 16 different
values finishing the collection. Next we count the number of generated
values that were necessary to finish the collection. We repeat this
4408394 times. And check in how many of these tests we managed to
finish the collection generating only 16 numbers, only 17 numbers, and
so on until 116+ generated values. We compare these results with what
is expected in a real random sequence.

The result is:

\vbox{%A forma mais gseral de remover espaçamento entre linhas:
\baselineskip-1000pt\lineskip0pt\lineskiplimit16383.99999pt\tabskip0pt
\def\linha{\noalign{\hrule}}
\def\hidewidth{\hskip-1000pt plus 1fill}
\def\col{\hbox{\vrule height12pt depth3.5pt width0pt}}
\halign to15cm{\col#& \vrule#\tabskip=1em plus2em&
\hfil#& \vrule#& \hfil#\hfil& \vrule#&
\hfil#& \vrule#&\hfil#& \vrule#\tabskip=0pt\cr\linha
&&\omit\hidewidth Generator\hidewidth&&\omit\hidewidth
Sucesso\hidewidth&&
\omit\hidewidth Generator\hidewidth&&Sucesso&\cr\linha
&&SFMT&&91\%&&Xoshiro**&&92\%&\cr\linha
&&PCG&&92\%&&\monoespaco{LCG}&&00\%&\cr\linha
&&ChaCha20&&92\%&&&&&\cr\linha}}

This is the first test in which one of the algorithms fail. The LCG
produces results that are too idealized to be considered random.

\subsecao{3.6. Permutation Test}

Here we generate numbers between 0 and 7, discarding repeated
values. In the end we produce a list with 8 values in some order. We
repeat the test and count the number of times that each permutation of
8 numbers is produced, comparing the result with what is expected from
a random sequence.


The result is:

\vbox{%A forma mais gseral de remover espaçamento entre linhas:
\baselineskip-1000pt\lineskip0pt\lineskiplimit16383.99999pt\tabskip0pt
\def\linha{\noalign{\hrule}}
\def\hidewidth{\hskip-1000pt plus 1fill}
\def\col{\hbox{\vrule height12pt depth3.5pt width0pt}}
\halign to15cm{\col#& \vrule#\tabskip=1em plus2em&
\hfil#& \vrule#& \hfil#\hfil& \vrule#&
\hfil#& \vrule#&\hfil#& \vrule#\tabskip=0pt\cr\linha
&&\omit\hidewidth Generator\hidewidth&&\omit\hidewidth
Sucesso\hidewidth&&
\omit\hidewidth Generator\hidewidth&&Sucesso&\cr\linha
&&SFMT&&93\%&&Xoshiro**&&92\%&\cr\linha
&&PCG&&93\%&&\monoespaco{LCG}&&02\%&\cr\linha
&&ChaCha20&&92\%&&&&&\cr\linha}}

\subsecao{3.7. Runs Up Test}

Here, like in the previous test, we generate a permutation of
elements. But instead of counting each possible permutation, we check
the number of ascending sequences in which each element is greater
than the previous one.

For this, we generated permutations with numbers between 0 and
8192. And we counted the number of ascending sequences with size 1, 2,
3, 4, 5, 6+ elements.

The result is:

\vbox{%A forma mais gseral de remover espaçamento entre linhas:
\baselineskip-1000pt\lineskip0pt\lineskiplimit16383.99999pt\tabskip0pt
\def\linha{\noalign{\hrule}}
\def\hidewidth{\hskip-1000pt plus 1fill}
\def\col{\hbox{\vrule height12pt depth3.5pt width0pt}}
\halign to15cm{\col#& \vrule#\tabskip=1em plus2em&
\hfil#& \vrule#& \hfil#\hfil& \vrule#&
\hfil#& \vrule#&\hfil#& \vrule#\tabskip=0pt\cr\linha
&&\omit\hidewidth Generator\hidewidth&&\omit\hidewidth
Sucesso\hidewidth&&
\omit\hidewidth Generator\hidewidth&&Sucesso&\cr\linha
&&SFMT&&92\%&&Xoshiro**&&91\%&\cr\linha
&&PCG&&90\%&&\monoespaco{LCG}&&90\%&\cr\linha
&&ChaCha20&&90\%&&&&&\cr\linha}}

\subsecao{3.8. Maximum of $t$ Test}

Here we produce 3 numbers between 0 and 63 and choose always the
greatest value. Next we count how many times each number is chosen
when we repeat the test and compare the result with what is expected.

The result is:

\vbox{%A forma mais gseral de remover espaçamento entre linhas:
\baselineskip-1000pt\lineskip0pt\lineskiplimit16383.99999pt\tabskip0pt
\def\linha{\noalign{\hrule}}
\def\hidewidth{\hskip-1000pt plus 1fill}
\def\col{\hbox{\vrule height12pt depth3.5pt width0pt}}
\halign to15cm{\col#& \vrule#\tabskip=1em plus2em&
\hfil#& \vrule#& \hfil#\hfil& \vrule#&
\hfil#& \vrule#&\hfil#& \vrule#\tabskip=0pt\cr\linha
&&\omit\hidewidth Generator\hidewidth&&\omit\hidewidth
Sucesso\hidewidth&&
\omit\hidewidth Generator\hidewidth&&Sucesso&\cr\linha
&&SFMT&&92\%&&Xoshiro**&&92\%&\cr\linha
&&PCG&&91\%&&\monoespaco{LCG}&&86\%&\cr\linha
&&ChaCha20&&92\%&&&&&\cr\linha}}

\subsecao{3.9. Collision Test}

Here we do not use the chi-square test because we are searching with
elements that have a probability too small to be measured by this
test. We measure the probability of finding a collision if we treat
our RNG as a hash function.

For this test me assumed a 20-bit output for the RNG and produced
$2^{14}$ different outputs. Next, we counted the number of collisions
checking if this value is too greater or too smaller than expected. We
repeat this test 3000 times. The number of expected collisions was
chosen to produce a success rate near to 92\% like in the previous
tests.

The result is:

\vbox{%A forma mais gseral de remover espaçamento entre linhas:
\baselineskip-1000pt\lineskip0pt\lineskiplimit16383.99999pt\tabskip0pt
\def\linha{\noalign{\hrule}}
\def\hidewidth{\hskip-1000pt plus 1fill}
\def\col{\hbox{\vrule height12pt depth3.5pt width0pt}}
\halign to15cm{\col#& \vrule#\tabskip=1em plus2em&
\hfil#& \vrule#& \hfil#\hfil& \vrule#&
\hfil#& \vrule#&\hfil#& \vrule#\tabskip=0pt\cr\linha
&&\omit\hidewidth Generator\hidewidth&&\omit\hidewidth
Sucesso\hidewidth&&
\omit\hidewidth Generator\hidewidth&&Sucesso&\cr\linha
&&SFMT&&94\%&&Xoshiro**&&93\%&\cr\linha
&&PCG&&94\%&&\monoespaco{LCG}&&92\%&\cr\linha
&&ChaCha20&&93\%&&&&&\cr\linha}}

\subsecao{3.10. Birthday Spacing Test}

Here we generate 512 different values, each one with 25 bits. Next we
order them and measure the difference between successive numbers
counting the number of equal differences found. The result is compared
with what is expected from random sequences.

This test was originally suggested by George Marsaglia and at the time
found problems in many generators that passed in other tests.

Our results are:

\vbox{%A forma mais gseral de remover espaçamento entre linhas:
\baselineskip-1000pt\lineskip0pt\lineskiplimit16383.99999pt\tabskip0pt
\def\linha{\noalign{\hrule}}
\def\hidewidth{\hskip-1000pt plus 1fill}
\def\col{\hbox{\vrule height12pt depth3.5pt width0pt}}
\halign to15cm{\col#& \vrule#\tabskip=1em plus2em&
\hfil#& \vrule#& \hfil#\hfil& \vrule#&
\hfil#& \vrule#&\hfil#& \vrule#\tabskip=0pt\cr\linha
&&\omit\hidewidth Generator\hidewidth&&\omit\hidewidth
Sucesso\hidewidth&&
\omit\hidewidth Generator\hidewidth&&Sucesso&\cr\linha
&&SFMT&&92\%&&Xoshiro**&&92\%&\cr\linha
&&PCG&&92\%&&\monoespaco{LCG}&92\%&\cr\linha
&&ChaCha20&&91\%&&&&&\cr\linha}}

\subsecao{3.11. Serial Correlation Test}

In this test we produce 1000 numbers with 64 bits that we convert to
numbers between 0 and 1 dividing each number by $2^{64}-1$ casting to
floating points numbers with double precision. We compute the serial
correlation coefficient between each number compared to the previous
one, and measure if the result is in an acceptable range.

Next we repeat the same test not between each number and its
successor, but between each number and the number two positions ahead
in the sequence. Next with 3 positions ahead, and so on. We try to
find the worst serial correlation coefficient.

We choose an acceptable range as one in which we expect to succeed in
this test with probability about 93\% given a random sequence.

Our result is:

\vbox{%A forma mais gseral de remover espaçamento entre linhas:
\baselineskip-1000pt\lineskip0pt\lineskiplimit16383.99999pt\tabskip0pt
\def\linha{\noalign{\hrule}}
\def\hidewidth{\hskip-1000pt plus 1fill}
\def\col{\hbox{\vrule height12pt depth3.5pt width0pt}}
\halign to15cm{\col#& \vrule#\tabskip=1em plus2em&
\hfil#& \vrule#& \hfil#\hfil& \vrule#&
\hfil#& \vrule#&\hfil#& \vrule#\tabskip=0pt\cr\linha
&&\omit\hidewidth Generator\hidewidth&&\omit\hidewidth
Success\hidewidth&&
\omit\hidewidth Generator\hidewidth&&Success&\cr\linha
&&SFMT&&93\%&&Xoshiro**&&93\%&\cr\linha
&&PCG&&93\%&&\monoespaco{LCG}&93\%&\cr\linha
&&ChaCha20&&93\%&&&&&\cr\linha}}

\subsecao{3.12. Generating all 32-bit values}

This is not a test like the previous ones. Instead of a percentage, we
are interested if our random number generator can really produce all
possible 32-bit numbers when executed for sufficient time. And if so,
how many 32-bit numbers need to be generated. We assume that a
generator that do not produce all the numbers fail.

Our result is:

\vbox{%A forma mais gseral de remover espaçamento entre linhas:
\baselineskip-1000pt\lineskip0pt\lineskiplimit16383.99999pt\tabskip0pt
\def\linha{\noalign{\hrule}}
\def\hidewidth{\hskip-1000pt plus 1fill}
\def\col{\hbox{\vrule height12pt depth3.5pt width0pt}}
\halign to15cm{\col#& \vrule#\tabskip=1em plus2em&
\hfil#& \vrule#& \hfil#\hfil& \vrule#&
\hfil#& \vrule#&\hfil#& \vrule#\tabskip=0pt\cr\linha
&&\omit\hidewidth Generator\hidewidth&&\omit\hidewidth
Numbers generated\hidewidth&&
\omit\hidewidth Generator\hidewidth&&Success&\cr\linha
&&SFMT&&98852849277&&Xoshiro**&&92842427748&\cr\linha
&&PCG&&100979563727&&\monoespaco{LCG}&8589934581&\cr\linha
&&ChaCha20&&99686167785&&&&&\cr\linha}}

All RNG algorithms were able to produce all possible 32-bit numbers
and the number of tries were similar, except for LCG. The number of
generated numbers for LCG was too low, half the produced values were
new numbers, not repeated values. Like the collector test, this shows
the known bias in LCG generators about not repeating previously
generated values.

\subsecao{3.13. Test Conclusion}

We managed to produce tests that find problems for the LCG
algorithm. The result indicate a bias in that RNG against producing
repeated values.

\secao{4. Benchmarks}

To measure the performance for each algorithm, we produced 100
millions 64-bit numbers. To avoid compiler optimizations, we used each
result in a sum. We tested the time spent in 4 different environments
and 2 different computers.

The following test (with result in seconds) is the result of running
the test in OpenBSD 6.7 using computer A (Intel Pentium B980 dual
core, 2,40 GHz, 4 GB RAM) and compiler Clang 8.0.1:

\vbox{%A forma mais gseral de remover espaçamento entre linhas:
\baselineskip-1000pt\lineskip0pt\lineskiplimit16383.99999pt\tabskip0pt
\def\linha{\noalign{\hrule}}
\def\hidewidth{\hskip-1000pt plus 1fill}
\def\col{\hbox{\vrule height12pt depth3.5pt width0pt}}
\halign to15cm{\col#& \vrule#\tabskip=1em plus2em&
\hfil#& \vrule#& \hfil#\hfil& \vrule#&
\hfil#& \vrule#&\hfil#& \vrule#\tabskip=0pt\cr\linha
&&\omit\hidewidth Generator\hidewidth&&\omit\hidewidth
Time (s)\hidewidth&&
\omit\hidewidth Generator\hidewidth&&Time (s)&\cr\linha
&&SFMT&&9.571183&&Xoshiro**&&7.811493&\cr\linha
&&PCG&&8.133946&&\monoespaco{LCG}&7.411282&\cr\linha
&&ChaCha20&&10.498551&&SplitMix64&7.529885&&\cr\linha}}

Repeating these tests using the same computer, but Windows 10 and
compiling with Visual Studio (which does not support 128-bit
variables, and cannot compile our implementation for PCG and Mersenne
Twister), we get the following result:

\vbox{%A forma mais gseral de remover espaçamento entre linhas:
\baselineskip-1000pt\lineskip0pt\lineskiplimit16383.99999pt\tabskip0pt
\def\linha{\noalign{\hrule}}
\def\hidewidth{\hskip-1000pt plus 1fill}
\def\col{\hbox{\vrule height12pt depth3.5pt width0pt}}
\halign to15cm{\col#& \vrule#\tabskip=1em plus2em&
\hfil#& \vrule#& \hfil#\hfil& \vrule#&
\hfil#& \vrule#&\hfil#& \vrule#\tabskip=0pt\cr\linha
&&\omit\hidewidth Generator\hidewidth&&\omit\hidewidth
Time (s)\hidewidth&&
\omit\hidewidth Generator\hidewidth&&Time (s)&\cr\linha
&&SFMT&&-&&Xoshiro**&&4.570623&\cr\linha
&&PCG&&-&&\monoespaco{LCG}&3.615481&\cr\linha
&&ChaCha20&&8.019865&&SplitMix64&4.340963&&\cr\linha}}

Repeating the test in a second newer computer, a Intel i5-3210M quad
core with 2,50GHz and 4 GB RAM, running Ubuntu 20.04.2 and compiling
with GCC 9.3.0, our result is:

\vbox{%A forma mais gseral de remover espaçamento entre linhas:
\baselineskip-1000pt\lineskip0pt\lineskiplimit16383.99999pt\tabskip0pt
\def\linha{\noalign{\hrule}}
\def\hidewidth{\hskip-1000pt plus 1fill}
\def\col{\hbox{\vrule height12pt depth3.5pt width0pt}}
\halign to15cm{\col#& \vrule#\tabskip=1em plus2em&
\hfil#& \vrule#& \hfil#\hfil& \vrule#&
\hfil#& \vrule#&\hfil#& \vrule#\tabskip=0pt\cr\linha
&&\omit\hidewidth Generator\hidewidth&&\omit\hidewidth
Time (s)\hidewidth&&
\omit\hidewidth Generator\hidewidth&&Time (s)&\cr\linha
&&SFMT&&2.848333&&Xoshiro**&&2.565788&\cr\linha
&&PCG&&2.292101&&\monoespaco{LCG}&2.378609&\cr\linha
&&ChaCha20&&7.085438&&SplitMix64&2.437079&&\cr\linha}}

The most surprising result is PCG being faster than a linear
congruent generator. This is very unexpected, as PCG uses
internally a LCG. This is also the only case in which PCG run faster
than Xoshiro** and SplitMix64. The result is not much different when
compiling with Clang 10.0 in the same system:

\vbox{%A forma mais gseral de remover espaçamento entre linhas:
\baselineskip-1000pt\lineskip0pt\lineskiplimit16383.99999pt\tabskip0pt
\def\linha{\noalign{\hrule}}
\def\hidewidth{\hskip-1000pt plus 1fill}
\def\col{\hbox{\vrule height12pt depth3.5pt width0pt}}
\halign to15cm{\col#& \vrule#\tabskip=1em plus2em&
\hfil#& \vrule#& \hfil#\hfil& \vrule#&
\hfil#& \vrule#&\hfil#& \vrule#\tabskip=0pt\cr\linha
&&\omit\hidewidth Generator\hidewidth&&\omit\hidewidth
Time (s)\hidewidth&&
\omit\hidewidth Generator\hidewidth&&Time (s)&\cr\linha
&&SFMT&&2.601277&&Xoshiro**&&2.449881&\cr\linha
&&PCG&&2.222556&&\monoespaco{LCG}&2.348797&\cr\linha
&&ChaCha20&&4.620730&&SplitMix64&2.346077&&\cr\linha}}

Using the same computer, but compiling using Emscripten 2.0.14 to
produce Web Assembly and and executing the code in the browser 
Firefox 86.0, the result is:

\vbox{%A forma mais gseral de remover espaçamento entre linhas:
\baselineskip-1000pt\lineskip0pt\lineskiplimit16383.99999pt\tabskip0pt
\def\linha{\noalign{\hrule}}
\def\hidewidth{\hskip-1000pt plus 1fill}
\def\col{\hbox{\vrule height12pt depth3.5pt width0pt}}
\halign to15cm{\col#& \vrule#\tabskip=1em plus2em&
\hfil#& \vrule#& \hfil#\hfil& \vrule#&
\hfil#& \vrule#&\hfil#& \vrule#\tabskip=0pt\cr\linha
&&\omit\hidewidth Generator\hidewidth&&\omit\hidewidth
Time (s)\hidewidth&&
\omit\hidewidth Generator\hidewidth&&Time (s)&\cr\linha
&&SFMT&&1.318&&Xoshiro**&&0.460&\cr\linha
&&PCG&&1.324&&\monoespaco{LCG}&0.349&\cr\linha
&&ChaCha20&&3.687&&SplitMix64&0.393&&\cr\linha}}

The surprising result is how faster this runs when compared with
native code. Other than that, implementations which use 128-bit
numbers appear to have a worse performance in this environment.


\secao{5. Conclusion: Choosing the Default Algorithm}

When the user do not choose which algorithm our API will use, we
should make for him an informed choice.

Our first criterion, of course, is if the algorithm is supported in
the current architecture and environment. Algorithms PCG and Mersenne
Twister will not compile if the chosen compiler do not support
128-bit numbers.

The second criterion is if the algorithm passed in our tests. Here only
LCG is excluded, all other algorithms passed. Therefore, we will not
consider it in our third criteria.

The third criterion is the speed. Here in all our tests SplitMix64 was
the faster one, except when we tested in a Linux environment. It is
possible that this happened because the default compilers were more
recent and optimized better an implementation with 128-bit variables.

Combining these criteria, we end using SplitMix64 as the default
random number generator, except if we are in Linux and we have a
compiler that supports 128-bit variables:

\iniciocodigo
@<Choose Default RNG Algorithm@>=
#if !defined(W_RNG_MERSENNE_TWISTER) && !defined(W_RNG_XOSHIRO) && \
!defined(W_RNG_PCG) && !defined(W_RNG_LCG) && !defined(W_RNG_CHACHA20)
#if defined(__SIZEOF_INT128__) && defined(__linux__)
#define W_RNG_PCG
#else
#define W_RNG_SPLITMIX
#endif
#endif
@
\fimcodigo

\secao{References}

\referencia{Knuth, D. E. (1984) ``Literate Programming'', The Computer Journal,
volume 27, second edition, p. 97--111}

\referencia{Knuth, D. (1998) ``The Art of Computer Programming, v. 2:
Seminumerical Algorithms'', Addison-Wesley Professional, third
edition.}

\referencia{Saito, M.; Matsumoto M. (2006) ``SIMD-oriented fast Mersenne Twister: a 128-bit pseudorandom number generator'', Monte Carlo and Quasi-Monte Carlo Methods, Springer, p. 607--622}.

\referencia{Steele, G.; Vigna S. (2021) ``Computationally Easy, Spectrally Good Multipliers for Congruential Pseudorandom Number Generators'', arXiv preprint arXiv:\-2001.\-05304}.

\referencia{Matsumoto, M; Wada I.; Kuramoto A.; Ashihara, H. (2007) ``Common Defects in the Initialization of Pseudrandom Number Generators'', ACM Trans. Model. Comput. Simul, 17, 4.}

\fim
