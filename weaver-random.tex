\font\sixteen=cmbx16
\font\twelve=cmr12
\font\fonteautor=cmbx12
\font\fonteemail=cmtt10
\font\twelvenegit=cmbxti12
\font\twelvebold=cmbx12
\font\trezebold=cmbx13
\font\twelveit=cmsl12
\font\monodoze=cmtt12
\font\it=cmti12
\voffset=0,959994cm % 3,5cm de margem superior e 2,5cm inferior
\parskip=6pt

\def\titulo#1{{\noindent\sixteen\hbox to\hsize{\hfill#1\hfill}}}
\def\autor#1{{\noindent\fonteautor\hbox to\hsize{\hfill#1\hfill}}}
\def\email#1{{\noindent\fonteemail\hbox to\hsize{\hfill#1\hfill}}}
\def\negrito#1{{\twelvebold#1}}
\def\italico#1{{\twelveit#1}}
\def\monoespaco#1{{\monodoze#1}}
\def\iniciocodigo{\lineskip=0pt\parskip=0pt}
\def\fimcodigo{\twelve\parskip=0pt plus 1pt\lineskip=1pt}

\long\def\abstract#1{\parshape 10 0.8cm 13.4cm 0.8cm 13.4cm
0.8cm 13.4cm 0.8cm 13.4cm 0.8cm 13.4cm 0.8cm 13.4cm 0.8cm 13.4cm
0.8cm 13.4cm 0.8cm 13.4cm 0.8cm 13.4cm
\noindent{{\twelvenegit Abstract: }\twelveit #1}}

\def\resumo#1{\parshape  10 0.8cm 13.4cm 0.8cm 13.4cm
0.8cm 13.4cm 0.8cm 13.4cm 0.8cm 13.4cm 0.8cm 13.4cm 0.8cm 13.4cm
0.8cm 13.4cm 0.8cm 13.4cm 0.8cm 13.4cm
\noindent{{\twelvenegit Resumo: }\twelveit #1}}

\def\secao#1{\vskip12pt\noindent{\trezebold#1}\parshape 1 0cm 15cm}
\def\subsecao#1{\vskip12pt\noindent{\twelvebold#1}}
\def\referencia#1{\vskip6pt\parshape 5 0cm 15cm 0.5cm 14.5cm 0.5cm 14.5cm
0.5cm 14.5cm 0.5cm 14.5cm {\twelve\noindent#1}}

%@* .

\twelve
\vskip12pt
\titulo{Gerador de Números Aleatórios Weaver}
\vskip12pt
\autor{Thiago Leucz Astrizi}
\vskip6pt
\email{thiago@@bitbitbit.com.br}
\vskip6pt

\abstract{This article contains the implementation of several random number
generator algorithms in literary programming and is intended to be
used with Weaver Game Engine. However, it can also be used as an
independent API in other projects. All different algorithms are
encapsulated in the same API letting programmers to easily change the
algorithm using macro definitions. We also describe here how the
algorithms perform in some statistical tests and in benchmarks running
in Linux, OpenBSD, Windows and Web Assembly.}

\vskip 0.5cm plus 3pt minus 3pt

\resumo{Este artigo contém a implementação de vários algoritmos geradores
de números pseudo-randômicos em programação literária e ele foi feito
para ser usado junto com o Motor de Jogos Weaver. Contudo, ele também
pode ser usado como uma API independente em outros projetos. Todos os
diferentes algoritmos são encapsulados usando a mesma API de modo que
é possível mudar o algoritmo usado apenas por meio de definição de
macros. Também descrevemos aqui como os algoritmos se saem diante de
diferentes testes estatísticos e em medidores de desempenho rodando em
Linux, OpenBSD, Windows e Web Assembly.}

\secao{1. Introdução}

\subsecao{1.1. Geradores de Números Aleatórios em Jogos Eletrônicos}

Um gerador de números aleatórios é útil para os seguintes cenários em
um jogo eletrônico:

a) Simulação: Isso se aplica tanto à simulação de fenômenos naturais
que podem ser aleatórios ou mesmo fenômonos sociais. Isso é usado
exaustivamente em jogos que são simuladores, como simuladores de
cidades como SimCity. Mas pode ser usado para simular variações no
clima, direção do vento ou no humor de personagens em quase qualquer
tipo de jogo.

b) Amostragem: Podem existir muitos tipos diferentes de oponentes que
podem ser gerados. Gerando as características variáveis aleatoriaente
estamos de fato fazendo uma amostragem de um tipo de personagem
específico dentre o conjunto de todos os possíveis. Por exemplo, em um
jogo de Pokémon, determinada espécie de criatura pode possuir muitas
variações diferentes de atributos possíveis a depender de sua genética
e os tipos de ataques que ela tem também podem mudar. Gerando uma
criatura aleatória, obtemos uma amostragem de um indivíduo dentre
vários outros possíveis. Há o caso de simulações de mundos inteiros
gerados de maneira procedural.

c) Programação: Durante o desenvolvimento do jogo, o programador
poderia usar o gerador para simular escolhas aleatórias que um jogador
poderia fazer para ver se alguma das combinações causa problemas ou
falhas. Ou gerar fases aleatórias somente para checar se algum bug
envolvendo o motor do jogo pode ser encontrado.

d) Estética: Adicionar aleatoriamente pequenas aleatoriedades em
padrões de imagens que de outra forma seriam regulares pode aumentar o
apelo estético de uma imagem. A aleatoriedade pode tornar também
efeitos de transição de telas mais interessantes.

e) Entretenimento: A diversão de um jogo pode estar inteiramente na
aleatoriedade. Este é o caso de jogos envolvendo dados, roletas,
embaralhamento de cartas.

Com tantos casos de uso diferentes para geradores de números
aleatórios em jogos eletrônicos, o gerador ideal deve ser um projetado
para propósito geral. Nunca um que apresenta um bom desempenho em só
alguns dos casos de uso acima.

Geralmente há uma tolerância maior para quando um gerador de números
aleatórios gera resultados previsíveis em um jogo.  Muitos jogos
antigos usavam geradores que consistiam apenas em uma sequência de
números fixos, e mesmo assim o viés do gerador não era aparente. Mas
existem casos em que as consequências de um gerador ruim são bastante
severas. Por exemplo, um simulador que se propõe a ser realista pode
falhar em treinar alguém para uma atividade real. Um gerador ruim pode
ser explorado em jogos envolvendo apostas ou competições sérias.

Nosso objetivo aqui será definir uma API para um gerador de números
aleatórios e programar diferentes opções de algoritmos para
instanciá-la. Um usuário poderá escolher qual algoritmo será usado,
mas se nenhuma escolha será feita, a biblioteca que será definida fará
sua própria escolha sobre o algoritmo.

\subsecao{1.2. Programação Literária e Notação Usada no Artigo}

Este artigo utiliza a técnica de ``Programação Literária'' para
desenvolver a API de gerador de números aleatórios. Esta técnica foi
apresentada em [Knuth, 1984] e tem por objetivo desenvolver
\italico{softwares} de tal forma que um programa de computador a ser compilado
é exatamente igual a um documento escrito para pessoas detalhando e
explicando o código. O presente documento não é algo independente do
código, mas sim consiste no próprio código-fonte do projeto.
Ferramentas automáticas são utilizadas para extrair o código deste
documento, colocá-lo na ordem correta e produzir o código que é
passado para o compilador.

Por exemplo, neste artigo serão definidos dois arquivos
diferentes: \monoespaco{random.c} e \monoespaco{random.h}, os quais
podem ser inseridos estaticamente em qualquer projeto, ou compilados
como uma biblioteca compartilhada. O conteúdo de \monoespaco{random.h}
é:

\iniciocodigo
@(src/random.h@>=
#ifndef WEAVER_RANDOM
#define WEAVER_RANDOM
#ifdef __cplusplus
extern "C" {
#endif
#include <stdint.h>
#include <stdbool.h>
#if defined(__unix__) || defined(__APPLE__)
#include <pthread.h>
#endif
@<Escolhe Algoritmo Padrão do RNG@>
@<Estrutura RNG@>
@<Declarações de Gerador Aleatório@>
#ifdef __cplusplus
}
#endif
#endif
@
\fimcodigo

As duas primeiras linhas assim como a última são macros que impedem
que garantem que as funções e variáveis declaradas ali serão inseridas
no máximo uma só vez em cada unidade de compilação. Também colocamos
macros para checar se estamos compilando o código como C ou C++. Se
estivermos em C++, avisamos o compilador que estamos definindo tudo
como código C e garantimos que não vamos modificar nada usando
sobrecarga de operadores. O código poderá ser armazenado de maneira
mais compacta.

A parte vermelha no código acima mostra que código será inserido ali
no futuro. Em ``Declarações de Gerador Aleatório'', por exemplo, nós
colocaremos declarações das funções.

Cada trecho de código tem um título, que no caso acima
é \monoespaco{random.h}. O título indica onde o código será
inserido. No caso acima, o código irá para um arquivo. Em trechos de
código futuros, haverá um com o título ``Declarações de Gerador
Aleatório'' com o código que irá nesta parte vermelha indicada.

\subsecao{1.3. Funções de API a serem Definidas}

Nosso gerador de números aleatórios deverá fornecer um total de 3
novas funções.

A primeira função gera e inicializa um gerador retornando um ponteiro
para ele. Para inicializá-lo, precisamos de uma função de alocação de
memória passada como primiero argumento (pode ser o \monoespaco{malloc} da
biblioteca padrão ou qualquer outro alocador personalizado). Em
seguida nossa semente será um valor com um tamanho de um vetor e o
.petor propriamente dito de números de 64 bits que assumimos terem sido
obtidos de maneira aleatória e uniforme.

Dessa forma, nossa função de inicialização é suficientemente genérica
para que o usuário possa fornecer um número personalizado de entropia
ao gerador. Caso ele seja usado em um jogo que simula o embaralhamento
de 54 cartas, por exemplo, o ideal é que ele passe um número de bits
aleatório $n$ tal que $2^n$ não seja muito menor que $54!$. Do
contrário, não seremos capazes de simular a maioria dos
embaralhamentos possíveis. Por outro lado, em algumas máquinas ou
ambientes pode ser difícil conseguir uma quantidade muito grande de
bits aleatórios para inicialização. Então, a API permite que um número
menor seja passado.

\iniciocodigo
@<Declarações de Gerador Aleatório@>=
struct _Wrng *_Wcreate_rng(void *(*alloc)(size_t), size_t size, uint64_t *seed);
@
\fimcodigo

Alguns algoritmos podem, contudo, ignorar qualquer valor maior que o
recomendado ou podem não dar qualquer garantia de qualidade se
receberem uma quantidade menor de bits aleatórios em sua
inicialização. Justamente por isso, vamos fazer com que a nossa API
avise a quantidade recomendada para o tamanho do vetor da semente,
deixando a região abaixo a ser definida em breve:

\iniciocodigo
@<Declarações de Gerador Aleatório@>=
@<Avisa Tamanho Ideal da Semente@>
@
\fimcodigo


A segunda função é a que efetivamente é usada para nos dar o próximo
número aleatório da sequência, que será um elemento de 64 bits:

\iniciocodigo
@<Declarações de Gerador Aleatório@>+=
uint64_t _Wrand(struct _Wrng *);
@
\fimcodigo

A última função será apenas para finalizar o uso de um gerador de
números aleatórios. Ela recebe também um ponteiro para função que
desalocará o gerador (ou NULL). Se receber NULL, o gerador não será
desalocado (alguns alocadores podem ter seu próprio coletor de lixo
sem fornecerem função de desalocação). De qualquer forma, ele não
poderá mais ser usado depois de finalizado:

\iniciocodigo
@<Declarações de Gerador Aleatório@>+=
bool _Wdestroy_rng(void (*free)(void *), struct _Wrng *);
@
\fimcodigo

\subsecao{1.4. Suporte à Threads}

O código adicionado aqui servirá para garantir que mais de uma thread
possa usar o gerador de números randômicos sem problemas após a sua
inicialização. Para isso precisamos de um cabeçalho adequado para
declarar os tipos que iremos usar nos ambientes em que for suportado:

\iniciocodigo
@<Incluir Cabeçalhos Necessários@>+=
#if defined(__unix__) || defined(__APPLE__)
#include <pthread.h>
#endif
#if defined(_WIN32)
#include <windows.h>
#endif
@
\fimcodigo

Tudo o que será preciso fazer é, para cada gerador, definir um mutex a
ser colocado dentro do gerador (dentro do \monoespaco{struct \_Wrng}):

\iniciocodigo
@<Declaração de Mutex@>=
#if defined(__unix__) || defined(__APPLE__)
pthread_mutex_t mutex;
#endif
#if defined(_WIN32)
CRITICAL_SECTION mutex;
#endif
@
\fimcodigo

Se temos um ponteiro para \monoespaco{struct \_Wrng} chamado
de \monoespaco{rng}, podemos inicializar seu mutex com:

\iniciocodigo
@<Inicialização de Mutex@>=
#if defined(__unix__) || defined(__APPLE__)
pthread_mutex_init(&(rng -> mutex), NULL);
#endif
#if defined(_WIN32)
InitializeCriticalSection(&(rng -> mutex));
#endif
@
\fimcodigo

Para pedirmos o uso de um Mutex, fazemos:

\iniciocodigo
@<Mutex:WAIT@>=
#if defined(__unix__) || defined(__APPLE__)
pthread_mutex_lock(&(rng -> mutex));
#endif
#if defined(_WIN32)
EnterCriticalSection(&(rng -> mutex));
#endif
@
\fimcodigo

E o código para liberarmos o uso do Mutex:

\iniciocodigo
@<Mutex:SIGNAL@>=
#if defined(__unix__) || defined(__APPLE__)
pthread_mutex_unlock(&(rng -> mutex));
#endif
#if defined(_WIN32)
LeaveCriticalSection(&(rng -> mutex));
#endif
@
\fimcodigo

O principal motivo de precisarmos de uma função para finalizar nosso
gerador, a função \monoespaco{\_Wdestroy\_rng}, é que precisaremos
finalizar o uso do mutex, além de desalocar o gerador se
necessário. Como é só isso, podemos já declarar esta função nesta
parte, pois já temos o necessário para defini-la:

\iniciocodigo
@<Definição de \_Wdestroy\_rng@>=
bool _Wdestroy_rng(void (*free)(void *), struct _Wrng *rng){
  bool ret;
#if defined(__unix__) || defined(__APPLE__)
  ret = pthread_mutex_destroy(&(rng -> mutex));
#elif defined(_WIN32)
  DeleteCriticalSection(&(rng -> mutex));
  ret = true; // Sempre é bem-sucedida segundo a documentação
#endif
  if(free != NULL)
    free(rng);
  return ret;
}
@
\fimcodigo

\secao{2. Algoritmos Geradores de Números Pseudo-Randômicos}

\subsecao{2.1. Gerador Linear Congruente (LGC)}

Um Gerador Linear Congruente (também chamado pela sigla inglesa
de \italico{Linear Congruent Generator}) é simplesmente um gerador de
sequências de valores aleatórias $(x_0, x_1, \ldots)$ definido com
ajuda de constantes inteiras $a$, $c$ e $m$ tal que:

$$
x_{i+1}=ax_i+c\; (mod\; m)
$$

A qualidade dos geradores varia enormemente de acordo com os
parâmetros escolhidos. Como queremos gerar sempre valores de 64 bits
com nossa API, vamos escolher $m=2^{64}$, o que irá facilitar bastante
a operação. Não será necessário usar o operador de módulo
explicitamente.

Com relação ao multiplicador $a$, nós usaremos o valor
0xfa346cbfd5890825 devido a este valor ser um dos listados entre
multiplicadores com uma qualidade boa para ser usado com este módulo
$m$ no artigo [Steele, 2021].

Com relação ao valor de $c$, um requisito para sua qualidade é que ele
seja ímpar (primo em relação à $m$), mas fora isso não há impacto na
qualidade se $c$ for algum valor ímpar. J'o valor inicial, também deve
ser sempre um número ímpar, mas sem outros requisitos.

Sendo assim, podemos usar os primeiros 64 bits da semente para
inicializar $x_0$, apenas forçando o bit menos significativo a ser
1. Se existir mais um valor de 64 bits na semente, usamos ele como
nossa constante $c$, apenas ajustando o bit menos significativo para
um. Do contrário apenas usamos $c=1$.

Isso nos dá os seguintes tamanhso recomendados para nossa semente:

\iniciocodigo
@<Avisa Tamanho Ideal da Semente@>=
#ifdef W_RNG_LCG
#define _W_RNG_MINIMUM_RECOMMENDED_SEED_SIZE  1
#define _W_RNG_MAXIMUM_RECOMMENDED_SEED_SIZE  2
#endif
@
\fimcodigo

A estrutura do nosso gerador é bastante simples, só precisa armazenar
o último valor gerado e o valor da constante $c$:

\iniciocodigo
@<Estrutura RNG@>=
#ifdef W_RNG_LCG
struct _Wrng{
  uint64_t last_value, c;
  @<Declaração de Mutex@>
};
#endif
@
\fimcodigo

E ela é inicializada com o código:

\iniciocodigo
@<Definição de \_Wcreate\_rng@>=
#ifdef W_RNG_LCG
struct _Wrng *_Wcreate_rng(void *(*allocator)(size_t), size_t size,
                           uint64_t *seed){
  struct _Wrng *rng = (struct _Wrng *) allocator(sizeof(struct _Wrng));
  if(rng != NULL){
    // Se não temos semente nenhuma, usamos uma constante aleatória como semente:
    rng -> last_value = ((size > 0)?(seed[0]):(0x1c3b9d10b1d41adc));
    rng -> c = ((size > 1)?(seed[1]):(1));
    rng -> last_value |= (1u);
    rng -> c |= (1u);
    @<Inicialização de Mutex@>
  }
  return rng;
}
#endif
@
\fimcodigo

E finalmente, podemos agora definir a função que gera os próximos
valores pseudo-randômicos:

\iniciocodigo
@<Definição de \_Wrand@>=
#ifdef W_RNG_LCG
uint64_t _Wrand(struct _Wrng *rng){
  uint64_t ret;
  @<Mutex:WAIT@>
  rng -> last_value = 0xfa346cbfd5890825 * rng -> last_value + rng -> c;
  ret = rng -> last_value;
  @<Mutex:SIGNAL@>
  return ret;
}
#endif
@
\fimcodigo

O período esperado para esse gerador é $2^{64}$, depois dessa
quantidade os valores começam a se repetir.


\subsecao{2.2. SFMT}

O SFMT é o ``SIMD-Oriented Fast Mersenne Twister'', proposto pela
primeira vez em [Saito, 2006]. Ele tem esse nome porque o período
deste gerador é projetado para ter valores bastante grandes, que são
primos de Mersenne ($2^n-1$ para algum $n$ inteiro positivo). As
implementações mais usadas deste algoritmo suportam períodos de
$2^{19937}-1$ sem que haja repetição. O que geralmente é mais que
suficiente (ou mesmo um exagero) para qualquer caso de uso. Já o
``SIMD'' é um conjunto de instruções para valores vetorizados de 128
bits que CPUs mais recentes suportam. Tais instruções, se suportadas,
são usadas para tornar o desempenho mais rápido quando o compilador é
esperto o bastante para usá-las ou se a usamos explicitamente com
assembly.

O SFMT funciona por meio da seguinte recursão onde cada valor $X_i$
tem 128 bits:

$$
X_{i+n}=g(X_i, \ldots, X_{i+n-1})
$$

No caso, o gerador usa como $n$ o número 156. O que significa que nós
sempre precisamos memorizar os últimos 156 valores da sequência (cada
um com 128 bits). Isso explica porque este gerador precisa de uma
quantidade tão maior de memória quando comparado a outros.

A função $g$ funciona concatenando toda a sua entrada na forma de um
vetor binário com $156 \times 128$ elementos. Então, ela multiplica
uma matriz binária por este vetor binário. A matriz binária tem
$156 \times 128$ colunas e 128 linhas. Exceto que ao invés de
trabalharmos usando o corpo convencional de números racionais ou
reais, nós trabalhamos no corpo finito GF(2) composto somente pelos
números 0 e 1 (o vetor e a matriz são binários). Isso significa que o
que entendemos por ``multiplicação'' aqui é um AND lógico e a adição é
um XOR lógico. E também escolhemos uma matriz conveniente para que não
seja necessário representar ela explicitamente. Poderemos representar
a multiplicação da matriz por um vetor por meio de um número compacto
de operações.

Na estrutura de nosso gerador, precisamos armazenar todos os valores
anteriores necessários e também um \italico{offset} que armazena a
posição do próximo valor de 64 bits a ser retornado. Se este for um
valor par, assumimos que temos que gerar um novo valor de 128 bits e
retornar seus primeiros 64 bits. Se for um valor ímpar, não
precisaremos gerar um novo valor, apenas retornaremos os 64 bits
restantes do último valor.

\iniciocodigo
@<Estrutura RNG@>=
#ifdef W_RNG_MERSENNE_TWISTER
struct _Wrng{
  char w[128 * 156 / 8]; // Todos N valores gerados, cada um com _W bits
  int offset;          // Índice para o próximo valor retornado
  @<Declaração de Mutex@>
};
#endif
@
\fimcodigo

O padrão ISO da linguagem C não suporta variáveis que garantidamente
tenham 128 bits. Contudo, versões recentes do compilador GCC e Clang
as suportam. Usando elas, podemos computar a função $g$ que computa o
próximo elemento da sequência (ou seja, a multiplicação de nosso vetor
por uma matriz em $F_2$) por meio da seguintes operações:

\iniciocodigo
@<SFMT: Computa próximo elemento@>=
#ifdef __SIZEOF_INT128__
unsigned __int128 result, tmp;
uint32_t aux[4];
int i, index = rng -> offset / 2;
result = ((unsigned __int128 *) (rng -> w))[index];
result = result << 8;
result = result ^ ((unsigned __int128 *) (rng -> w))[index];
i = (index + 122) % 156;
aux[0] = *(((uint32_t *) &(((unsigned __int128 *) (rng -> w))[i])) + 0);
aux[1] = *(((uint32_t *) &(((unsigned __int128 *) (rng -> w))[i])) + 1);
aux[2] = *(((uint32_t *) &(((unsigned __int128 *) (rng -> w))[i])) + 2);
aux[3] = *(((uint32_t *) &(((unsigned __int128 *) (rng -> w))[i])) + 3);
aux[0] = (aux[0] >> 11) & 0xDFFFFFEF; // 0xBFFFFFF6
aux[1] = (aux[1] >> 11) & 0xDDFECB7F; // 0xBFFAFFFF
aux[2] = (aux[2] >> 11) & 0xBFFAFFFF; // 0xDDFECB7F
aux[3] = (aux[3] >> 11) & 0xBFFFFFF6; // 0xDFFFFFEF
memcpy(&tmp, aux, 16);
result = result ^ tmp;
i = (index + 156 - 2) % 156;
result = result ^ (((unsigned __int128 *) (rng -> w))[i] >> 8);
i = (index + 156 - 1) % 156;
aux[0] = *(((uint32_t *) &(((unsigned __int128 *) (rng -> w))[i])) + 0);
aux[1] = *(((uint32_t *) &(((unsigned __int128 *) (rng -> w))[i])) + 1);
aux[2] = *(((uint32_t *) &(((unsigned __int128 *) (rng -> w))[i])) + 2);
aux[3] = *(((uint32_t *) &(((unsigned __int128 *) (rng -> w))[i])) + 3);
aux[0] = (aux[0] << 18);
aux[1] = (aux[1] << 18);
aux[2] = (aux[2] << 18);
aux[3] = (aux[3] << 18);
memcpy(&tmp, aux, 16);
result = result ^ tmp;
((unsigned __int128 *) (rng -> w))[index] = result;
#else
#error "Mersenne Twister unsupported without 128 bit integer support." 
#endif
@
\fimcodigo

Note que se estivermos em um compilador que não suporta variáveis de
tamanho fixo com 128 bits, nós simplesmente retornamos um erro e não
compilaremos.

O código acima representa a multiplicação de uma matriz por um
vetor. Note que a matriz usada é bastante esparsa. Se nós queremos
computar o valor $x_i$, nós o obtemos consultando apenas os valores
anteriores $x_{i-1}$, $x_{i-2}$, $x_{i-156}$ e $x_{i-34}$.

Se sabemos como computar o próxim elemento $x_i$, o código da função
que nos dá o próximo valor de 64 bits será:

\iniciocodigo
@<Definição de \_Wrand@>=
#ifdef W_RNG_MERSENNE_TWISTER
uint64_t _Wrand(struct _Wrng *rng){
  uint64_t ret;
  @<Mutex:WAIT@>
  if(rng -> offset % 2 == 0){
    @<SFMT: Computa próximo elemento@>
  }
  ret = ((uint64_t *) (rng -> w))[rng -> offset];
  rng -> offset = (rng -> offset + 1) % (128 * 156 / 64);
  @<Mutex:SIGNAL@>
  return ret;
}
#endif
@
\fimcodigo

Para inicializar o nosso gerador, precisamos preencher nosso histórico
de v156 valores anteriores com ajuda de nossa semente. Como a semente
provavelmente será menor do que isso, aplicamos a ela um mini-gerador
de números pseudo-randômicos só para esticar a aleatoriedade inicial e
preencher os valores. Os primeiros dois componentes de nosso
mini-gerador são:

\iniciocodigo
@<Definição de \_Wcreate\_rng@>=
#ifdef W_RNG_MERSENNE_TWISTER
static uint32_t f1(uint32_t x){
  return (x ^ (x >> 27)) * (uint32_t) 1664525UL;
}
static uint32_t f2(uint32_t x){
  return (x ^ (x >> 27)) * (uint32_t) 1566083941UL;
}
#endif
@
\fimcodigo

Como inicializaremos nossa estrutura é definido pela implementação de
referência do algoritmo. Assim, tal como na referência, usamos um
mini-gerador baseado em números de 32 bits para preencher o estado
inicial:

\iniciocodigo
@<Definição de \_Wcreate\_rng@>=
#ifdef W_RNG_MERSENNE_TWISTER
struct _Wrng *_Wcreate_rng(void *(*allocator)(size_t), size_t size,
                           uint64_t *seed){
  struct _Wrng *rng = (struct _Wrng *) allocator(sizeof(struct _Wrng));
  if(rng != NULL){
    uint32_t *dst = (uint32_t *) (rng -> w), *origin = (uint32_t *) seed;
    size_t size_dst = 128 * 156 / 32, size_origin = size * 2;
    int count, r, i, j, mid = 306, lag = 11;
    // Preenchemos inicialmente tudo com 0x8b:
    memset(rng -> w, 0x8b, 128 * 156 / 8);
    count = ((size_origin + 1 >= size_dst)?(size_origin + 1):(size_dst));
    r = f1(dst[0] ^ dst[mid] ^ dst[size_dst - 1]);
    dst[mid] += r;
    r += size * 2;
    dst[mid + 11] += r;
    dst[0] = r;
    count --;
    for(i = 1, j = 0; j < count && j  < size_origin; j ++){
      r = f1(dst[i] ^  dst[(i + mid) % size_dst] ^
             dst[(size_dst - 1 + i) % size_dst]);
      dst[(i + mid) % size_dst] += r;
      r += origin[j] + i;
      dst[(i + mid + 11) % size_dst] += r;
      dst[i] = r;
      i = (i + 1) % size_dst;
    }
    for(; j < count; j++){
      r = f1(dst[i] ^ dst[(i + mid) % size_dst] ^
             dst[(i + size_dst - 1) % size_dst]);
      dst[(i + mid) % size_dst] += r;
      r += i;
      dst[(i + mid + 11) % size_dst] += r;
      dst[i] = r;
      i = (i + 1) % size_dst;
    }
    for (j = 0; j < size_dst; j++) {
      r = f2(dst[i] + dst[(i + mid) % size_dst] +
             dst[(i + size_dst - 1) % size_dst]);
      dst[(i + mid) % size_dst] ^= r;
      r -= i;
      dst[(i + mid + lag) % size_dst] ^= r;
      dst[i] = r;
      i = (i + 1) % size_dst;
    }
    rng -> offset = 0;
    @<SFMT: Garante Período@>
    @<Inicialização de Mutex@>
  }
  return rng;
}
#endif
@
\fimcodigo

Entretanto, antes de retornar o código acima é preciso checar se o
estado que geramos realmente tem o período desejado de
$2^{19937}-1$. O modo de fazer isso é por meio da checagem de paridade
para alguns bits específicos de nossa semente inicial. Se a paridade
nao estiver correta, apenas ajustamos ela sem precisar mudar os demais
valores gerados:

\iniciocodigo
@<SFMT: Garante Período@>=
{
  // Máscara de bits a serem checados:
  uint32_t parity = (dst[0] & 0x00000001U) ^ (dst[3] & 0xc98e126aU);
  // Checagem de paridade:
  for (i = 16; i > 0; i >>= 1)
    parity ^= parity >> i;
  parity = parity & 1;
  if(parity != 1)
    dst[0] = dst[0] ^ 1;
}
@
\fimcodigo

Por fim, temos que documentar o tamanho recomendado para a semente do
algoritmo. Podemos usar sem problemas uma semente com um único valor
de 64 bits, já que a implementação de referência chega a suportar
sementes com apenas 32 bits. Se passarmos sementes maiores, elas serão
usadas para ajudar a preencher o estado inicial. Contudo, como o
estado inicial é um vetor de 19968 bits, fornecer mais de 312 valores
com 64 bits seria redundante:

\iniciocodigo
@<Avisa Tamanho Ideal da Semente@>=
#ifdef W_RNG_MERSENNE_TWISTER
#define _W_RNG_MINIMUM_RECOMMENDED_SEED_SIZE  1
#define _W_RNG_MAXIMUM_RECOMMENDED_SEED_SIZE  312
#endif
@
\fimcodigo

\subsecao{2.3. SplitMix64}

O SplitMix é um gerador de números aleatórios projetado para ser
possível de ser dividido. Isso significa que existe um algoritmo tal
que dada a estrutura do gerador, retorna uma nova estrutura tal que
agora ambas podem gerar números pseudo-randômicos sem correlação entre
si e sem a necessidade de fornecer novas sementes.

Mas o verdadeiro motivo de estarmos fornecendo o SplitMIx aqui não é
esta propriedade. Não há planos de fornecer o suporte para geradores
divisíveis aqui nesta API. Nosso verdadeiro interesse nete gerador é o
fato de que este gerador tem uma natureza diferente dos demais
definidos aqui e pelo fato dele funcionar precisando apenas de uma
semente de 64 bits.

Lembre-se que no final da seção do Mersenne Twister foi necessário
iicializar um estado de 312 bits mesmo que a nossa semente inicial
tenha menos de 312 bits aleatórios. No caso do Mersenne Twister,
usamos exatamente o método sugerido em sua implementação de
referência, a qual consistia em usar outro RNG para preencher o estado
inicial dada a semente que temos. Mas nes todos os geradores possuem
um método padrão recomendado para preencher o estado inicial. Para os
casos em que não há um método padrão, precisamos justamente de um
gerador que use sementes pequenas, e publicações como [Matsumoto,
2007] mostram a importância de usarmos para este fim geradores
radicalmente diferentes para evitar correlação entre sequências
geradas com sementes semelhantes.

Ignorando os mecanismos de dividir a estrutura do gerador, o SplitMix
é bastante simples. Ele foi criado à partir da ideia de ter um estado
de 64 bits, e cada vez que um novo valor precisa ser gerado, o estado
é atualizado para um novo valor e retornamos o resultado de uma função
hash aplicada sobre o estado.

Exceto que por razões de performance, na prática ao invés de aplicar
uma função hash inteira, usamos um misturador de bits: um componente
que é parte de algumas funções hash e é responsável por evitar
correlações entre valores de entrada semelhantes. O misturador de bits
usado no SplaitMix64 é um qu foi definido originalmente para uma
função hash chamada de MurmurHash3:

\iniciocodigo
@<SplitMix: Misturador de Bits@>=
{
  uint64_t tmp = *state;
  tmp = (tmp ^ (tmp >> 33)) * 0xff51afd7ed558ccdl;
  tmp = (tmp ^ (tmp >> 33)) * 0xc4ceb9fe1a85ec53l;
  ret = tmp ^ (tmp >> 33);
} 
@
\fimcodigo

E antes de computar o misturador de bits acima, nós também temos que
atualizar o estado do RNG com a seguinte linha de código,
onde \monoespaco{gamma} é uma constante ímpar:

\iniciocodigo
@<SplitMix: Atualiza Estado@>=
{
  *state +=  gamma;
} 
@
\fimcodigo

A construção final da função que retorna um novo número
pseudo-randômico é:

\iniciocodigo
@<Funções Auxiliares@>=
#if defined(W_RNG_SPLITMIX) || defined(W_RNG_XOSHIRO) || defined(W_RNG_PCG) || \
  defined(W_RNG_CHACHA20)
static inline uint64_t splitmix_next(uint64_t *state, uint64_t gamma){
  uint64_t ret;
  @<SplitMix: Misturador de Bits@>
  @<SplitMix: Atualiza Estado@>
  return ret;
}
#endif
@
\fimcodigo

Se estivermos usando o SplitMix não para inicializar outros geradores
de números pseudo-randômicos, para para usar ele mesmo como nosso
gerador escolhido, iremos definir a estrutura de nosso gerador com o
estado e o valor gamma dele:

\iniciocodigo
@<Estrutura RNG@>=
#ifdef W_RNG_SPLITMIX
struct _Wrng{
  uint64_t state, gamma;
  @<Declaração de Mutex@>
};
#endif
@
\fimcodigo

E a função da API para gerar um novo número pseudo-randômico é:

\iniciocodigo
@<Definição de \_Wrand@>=
#ifdef W_RNG_SPLITMIX
uint64_t _Wrand(struct _Wrng *rng){
  uint64_t ret;
  @<Mutex:WAIT@>
  ret = splitmix_next(&(rng -> state), rng -> gamma);
  @<Mutex:SIGNAL@>
  return ret;
}
#endif
@
\fimcodigo

Podemos inicializar a estrutura do gerador diretamente usando a
semente para preencher o estado inicial. Se tiver mais bits aleatórios
na semente, podemos também usá-los para escolher o valor de gamma,
apenas cuidando para manter ele um valor ímpar e assim poder garantir
um período de $2^{64}$:

\iniciocodigo
@<Definição de \_Wcreate\_rng@>=
#ifdef W_RNG_SPLITMIX
struct _Wrng *_Wcreate_rng(void *(*allocator)(size_t), size_t size,
                           uint64_t *seed){
  struct _Wrng *rng = (struct _Wrng *) allocator(sizeof(struct _Wrng));
  if(rng != NULL){
    if(size < 1)
      rng -> state = 0x32147198b5436569;
    else
      rng -> state = seed[0];
    if(size < 2)
      rng -> gamma = 0x9e3779b97f4a7c15;
    else
      rng -> gamma = (seed[1] | 1);
    @<Inicialização de Mutex@>
  }
  return rng;
}
#endif
@
\fimcodigo

Isso significa que se usarmos o SplitMix como nosso gerador,
precisamos de uma semente com 1 ou 2 números de 64 bits:

\iniciocodigo
@<Avisa Tamanho Ideal da Semente@>=
#ifdef W_RNG_SPLITMIX
#define _W_RNG_MINIMUM_RECOMMENDED_SEED_SIZE  1
#define _W_RNG_MAXIMUM_RECOMMENDED_SEED_SIZE  2
#endif
@
\fimcodigo


\subsecao{2.4. Xoshiro256**}

Xoshiro e Mersenne Twister são geradores com algumas
similaridades. Ambos usam valores gerados do passado para produzir
novos valores envolvendo uma multiplicação de matriz e vetor
binários. Mas ao contrario do Mersenne Twister, cada novo valor gerado
pelo Xoshiro tem 256 bits ao invés de 128. E ao invés de gerar o
próximo valor levando em conta os 156 valores anteriores gerados,
Xoshiro gera o próximo valor apenas à partir do valor imediatamente
anterior gerado, precisando assim de menos memória.

Na estrutura do gerador nós armazenamos o número de 256 bits como uma
sequência de quatro valores de 64 bits:

\iniciocodigo
@<Estrutura RNG@>=
#ifdef W_RNG_XOSHIRO
struct _Wrng{
  uint64_t w[4];   // Valores de estado
  @<Declaração de Mutex@>
};
#endif
@
\fimcodigo

Assim como o Mersenne Twister, para produzir o próximo valor, nós
multiplicamos o valor anterior por uma matrix com operações no corpo
GF(2): usamos o operador AND nos bits ao invés da multiplicação e
usamos o operador XOR ao invés da soma. E também usamos uma matriz que
ao mesmo tempo produz bons resultados e é conveniente o bastante para
não precisar ser armazenada. Ao invés disso, a multiplicação por
matriz pode ser escrita usando apenas poucas operações rápidas
compostas por AND, XOR e SHIFT.

De fato, a multiplicação por matriz que atualiza nosso estado é
representada apenas por esse código de 7 linhas:

\iniciocodigo
@<Xoshiro: Multiplicação por Matriz@>=
{
  uint64_t t = rng -> w[1] << 17;
  rng -> w[2] ^= rng -> w[0];
  rng -> w[3] ^= rng -> w[1];
  rng -> w[1] ^= rng -> w[2];
  rng -> w[0] ^= rng -> w[3];
  rng -> w[2] ^= t;
  rng -> w[3] = ((rng -> w[3] << 45) | (rng -> w[3] >> 19));
}
@
\fimcodigo

Mas como estamos gerando números pseudo-randômicos usando uma simples
transfrmação linear e o nosso estado é muito pequeno (ao contrário do
estado do Mersenne Twister), precisamos usar um truque adicional para
evitar falhar em testes estatísticos. Ao invés de retornar o número de
256 bits que geramos, nós iremos sempe produzir a saída à partir de um
misturador que escolhe parte do valor total e aplica sobre ele
operações que mascaram a relação entre o valor atual e os anteriores:

\iniciocodigo
@<Xoshiro: Misturador@>=
{
  uint64_t tmp = rng -> w[1] * 5;
  ret = ((tmp << 7) | (tmp >> 57)) * 9;
} 
@
\fimcodigo

E finalmente, definimos a função que retorna o próximo valor
pseudo-randômico como abaixo, primeiro gerando o valor a ser
retornado \monoespaco{ret} à partir do misturador, e então obtendo o
próximo estado com a multiplicação por matriz:

\iniciocodigo
@<Definição de \_Wrand@>=
#ifdef W_RNG_XOSHIRO
uint64_t _Wrand(struct _Wrng *rng){
  uint64_t ret;
  @<Mutex:WAIT@>
  @<Xoshiro: Misturador@>
  @<Xoshiro: Multiplicação por Matriz@>
  @<Mutex:SIGNAL@>
  return ret;
}
#endif
@
\fimcodigo


Por fim, a inicialização do gerador Xoshiro** consistirá em copiar
para seu estado inicial a semente, caso esta tenha 256 bits ou
mais. Caso contrário, copiaremos o que temos e usaremos os últimos 64
bits da semente para inicializar o estado do algoritmo SplitMix64
usando um valor gama padrão e usamos os gerador do SplitMix64 para
produzir os bits restantes:

\iniciocodigo
@<Definição de \_Wcreate\_rng@>=
#ifdef W_RNG_XOSHIRO
struct _Wrng *_Wcreate_rng(void *(*allocator)(size_t), size_t size,
                           uint64_t *seed){
  int i;
  struct _Wrng *rng = (struct _Wrng *) allocator(sizeof(struct _Wrng));
  if(rng != NULL){
    if(size >= 4){
      for(i = 0; i < 4; i ++)
        rng -> w[i] = seed[i];
    }
    else{
      uint64_t state = 0x32147198b5436569, gamma = 0x9e3779b97f4a7c15;
      for(i = 0; i < size - 1; i ++)
        rng -> w[i] = seed[i];
      if(size > 1)
        state = seed[i];
      for(; i < 4; i ++)
        rng -> w[i] = splitmix_next(&state, gamma);
    }
    @<Inicialização de Mutex@>
  }
  return rng;
}
#endif
@
\fimcodigo

Devido a isso, vamos indicar como um valor entre 1 e 4 o tamanho
recomendado para a semente deste gerador:

\iniciocodigo
@<Avisa Tamanho Ideal da Semente@>=
#ifdef W_RNG_XOSHIRO
#define _W_RNG_MINIMUM_RECOMMENDED_SEED_SIZE  1
#define _W_RNG_MAXIMUM_RECOMMENDED_SEED_SIZE  4
#endif
@
\fimcodigo


\subsecao{2.5. PCG (\italico{Permuted Congruential Generator})}

O gerador de números aleatórios PCG é similar ao SplitMix em sua
filosofia de funcionamento. El usa uma operação simples para atualizar
o estado interno. Mas ao invés de tratar cada novo estado interno como
o novo número a ser retornado, ele o passa para uma função mais
complicada para assim produzir cada número aleatório de saída.

A atualização simples de estado, que no SplitMix consistia em somar
uma constante ao estado interno, aqui significa atualizar o estado por
meio de um gerador linear congruente (LCG) como o apresentado na seção
2.1.

Contudo, o gerador linear congruente que definimos funcionava usando
64 bits. Mas o estado interno do PCG tem 128 bits e por isso
precisamos de um gerador LCG com 128 bits para atualizar o estado.

A estrutura de nosso gerador é definida como:

\iniciocodigo
@<Estrutura RNG@>=
#ifdef W_RNG_PCG
#ifdef __SIZEOF_INT128__
struct _Wrng{
  unsigned __int128 state;
  unsigned __int128 increment; // Sempre deve ser ímpar
  @<Declaração de Mutex@>
};
#else
#error "PCG unsupported without 128 bit integer support."
#endif
#endif
@
\fimcodigo

Nós atualizamos o estado interno realizando uma operação LCG de
multiplicar por um multipli9cador constante escolhido cuidadosamente a
somar um incremente ímpar ao resultado:

\iniciocodigo
@<PCG: Atualizar Estado@>=
{
  unsigned __int128 multiplier;
  multiplier = 2549297995355413924ULL;
  multiplier = multiplier << 64;
  multiplier += 4865540595714422341ULL;
  rng -> state = rng -> state * multiplier + rng -> increment;
}
@
\fimcodigo

Mas tal como no SplitMix, ao invés de retornar este estado como o
próximo número da sequência, nós o passamos para algum tipo de função
hash, embaralhador ou misturador. No nosso caso, nós combinamos os 128
bits em 64 por meio de uma operação de XOR e em seguida aplicamos uma
função de permutação ao resultado:

\iniciocodigo
@<PCG: Permutação@>=
{
  uint64_t xorshifted, rot;
  xorshifted = (((uint64_t)(rng -> state >> 64u)) ^ ((uint64_t) rng -> state));
  rot = rng -> state >> 122u;
  ret = (xorshifted >> rot) | (xorshifted << ((-rot) & 63));
}
@
\fimcodigo

E finalmente, a função completa que gera o próximo número aleatório
combina as operações acima da seguinte forma:

\iniciocodigo
@<Definição de \_Wrand@>=
#ifdef W_RNG_PCG
uint64_t _Wrand(struct _Wrng *rng){
  uint64_t ret;
  @<Mutex:WAIT@>
  @<PCG: Atualizar Estado@>
  @<PCG: Permutação@>
  @<Mutex:SIGNAL@>
  return ret;
}
#endif
@
\fimcodigo

O que resta ser feito é inicializar o estado inicial. Se
inicializarmos com 256 bits ou mais, nós poderíamos usar a própria
semente para inicializar o estado inicial. Ao invés disso, para
agirmos assim como a implementação de referência, nós fazemos uma
mistura mais complicada com os bits de entrada tentando lidar com
casos em que a inicialização não é muito aleatória.


Se passarmos uma semente com somente 128 bits, podemos usá-la
diretamente para inicializar o estado inicial e podemos ajustar o
incremento do LCG como sendo 1. Se tivermos 192 bits aleatórios,
podemos usar os bits restantes para escolher um valor diferente para o
incremento. Já se tivermos menos de 128 bits, usaremos então o
SplitMix para esticar a nossa aleatoriedade inicial até 128. A função
de inicialização é definida conforme se vê abaixo:

\iniciocodigo
@<Definição de \_Wcreate\_rng@>=
#ifdef W_RNG_PCG
struct _Wrng *_Wcreate_rng(void *(*allocator)(size_t), size_t size,
                           uint64_t *seed){
  struct _Wrng *rng = (struct _Wrng *) allocator(sizeof(struct _Wrng));
  if(rng != NULL){
    if(size >= 4){
      unsigned __int128 multiplier;
      multiplier = 2549297995355413924ULL;
      multiplier = multiplier << 64;
      multiplier += 4865540595714422341ULL;
      unsigned __int128 initstate = seed[0], initseq;
      initstate = initstate << 64;
      initstate += seed[1];
      initseq = seed[2];
      initseq = initseq << 64;
      initseq += seed[3];
      rng->state = 0U;
      rng -> increment = (initseq << 1) | 1;
      rng->state = rng->state * multiplier + rng -> increment;
      rng -> state += initstate;
      rng->state = rng->state * multiplier + rng -> increment;
    }
    else if(size >= 2){
      rng -> state = seed[0];
      rng -> state = (rng -> state << 64);
      rng -> state += seed[1];
      if(size == 3){
        rng -> increment = seed[2];
        rng -> increment = (rng -> increment) | 1;
      }
      else
        rng -> increment = 1;
    }
    else{
      uint64_t state, gamma = 0x9e3779b97f4a7c15;
      if(size > 0)
        state = seed[0];
      else
        state = 0x32147198b5436569;
      rng -> state = splitmix_next(&state, gamma);
      rng -> state = (rng -> state << 64);
      rng -> state += splitmix_next(&state, gamma);
      rng -> increment = 1;
    }
    @<Inicialização de Mutex@>
  }
  return rng;
}
#endif
@
\fimcodigo

Isso significa que o tamanho recomendado para a semente é entre dois e
quatro números de 64 bits aleatórios para assim não precisarmos
esticar a aleatoriedae com o SplitMix:

\iniciocodigo
@<Avisa Tamanho Ideal da Semente@>=
#ifdef W_RNG_PCG
#define _W_RNG_MINIMUM_RECOMMENDED_SEED_SIZE  2
#define _W_RNG_MAXIMUM_RECOMMENDED_SEED_SIZE  4
#endif
@
\fimcodigo

\subsecao{2.6. ChaCha20}

O ChaCha20 é um gerador projetado para ser criptograficamente seguro,
mas essa afirmação só é verdadeira se ele for alimentado com ua
semente com um tamanho adequado, e se a semente realmente for obtida
de maneira aleatória e uniforme e sem que ela seja reaproveitada
novamente. Nossa API não garante tais coisas, aqui não estamos
interessados em geradores criptograficamente seguros. Para nós o
CHaCha20 será apenas um gerador como todos os outros anteriores, com o
diferencial de que espera-se que ele tenha uma qualidade maior com um
maior custo em desempenho para compensar.

O algoritmo tem como primeiro componente de seu funcionamento uma
função de preenchimento, que receberá como entrada 384 bits (a serem
interpretados como separados em componentes de 64 bits) e gera como
saída 512 bits. A ideia é que a saída seja interpretada como uma
matriz $4\times 4$ de elementos de 32 bits, mesmo que não a
armazenemos neste formato.

A função de preenchimento é:

@<ChaCha20: Preenchimento@>=
#ifdef W_RNG_CHACHA20
static void chacha_padding(uint64_t input[6], uint32_t output[16]){
  int i, j;
  output[0] = ('e' << 24) + ('x' << 16) + ('p' << 8) + 'a';
  output[1] = ('n' << 24) + ('d' << 16) + (' ' << 8) + '3';
  output[2] = ('2' << 24) + ('-' << 16) + ('b' << 8) + 'y';
  output[3] = ('t' << 24) + ('e' << 16) + (' ' << 8) + 'k';
  for(j=4, i = 0; i < 6; i ++, j += 2){
    output[j] = (input[i] / 4294967296llu);
    output[j+1] = input[i] % 4294967296llu;
  }
}
#endif
@

Na entrada dessa função de preenchimento, os primeiros 4 valores
sempre são retirados da semente inicial. O próximo é sempre um
contador que no primeiro bloco gerado é 0, e a cada próximo incrementa
em 1. E o último é um número que deve ser sempre usado uma só vez.

O próximo componente é a função abaixo que recebe 4 valores de 32bits
e os modifica realizando uma série de operações:

@<ChaCha20: QuarterRound@>=
#ifdef W_RNG_CHACHA20
void quarter_round(uint32_t *a, uint32_t *b, uint32_t *c, uint32_t *d){
  *a = *a + *b;
  *d = *d ^ *a;
  *d = (*d << 16) | (*d >> 16);
  *c = *c + *d;
  *b = *b ^ *c;
  *b = (*b << 12) | (*b >> 20);
  *a = *a + *b;
  *d = *d ^ *a;
  *d = (*d << 8) | (*d >> 24);
  *c = *c + *d;
  *b = *b ^ *c;
  *b = (*b << 7) | (*b >> 25);
}
#endif
@

A operação anterior é utilizada na seguinte função de permutação que
recebe como entrada 16 elementos de 32 bits e os transforma:

@<ChaCha20: Permutação@>=
#ifdef W_RNG_CHACHA20
void chacha_permutation(uint32_t elements[16]){
  int i;
  for(i = 0; i < 10; i ++){
    quarter_round(&elements[0], &elements[4], &elements[8], &elements[12]);
    quarter_round(&elements[1], &elements[5], &elements[9], &elements[13]);
    quarter_round(&elements[2], &elements[6], &elements[10], &elements[14]);
    quarter_round(&elements[3], &elements[7], &elements[11], &elements[15]);
    quarter_round(&elements[0], &elements[5], &elements[10], &elements[15]);
    quarter_round(&elements[1], &elements[6], &elements[11], &elements[12]);
    quarter_round(&elements[2], &elements[7], &elements[8], &elements[13]);
    quarter_round(&elements[3], &elements[4], &elements[9], &elements[14]);
  }
}
#endif
@

Com esses pré-requisitos podemos começar a definir a estrutura e
funções de nossa API. A estrutura do gerador deverá armazenar um vetor
com 6 elementos de 64 bits (que depois serão passados para a função
que gera novos números após antes passar pelo preenchimento), um
espaço de 512 bits com os últimos valores gerados (se existirem) e um
índice para saber qual o próximo valor que deve ser retornado. A
estrutura será então:

\iniciocodigo
@<Estrutura RNG@>=
#ifdef W_RNG_CHACHA20
struct _Wrng{
  uint64_t array[6];
  uint32_t generated_values[16];
  int index;
  @<Declaração de Mutex@>
};
#endif
@
\fimcodigo

Quando a estrutura é alocada, os primeiros 4 valores do nosso vetor de
6 elementos são copiados da semente, os próximo valor é preenchido
como 0 e o último também é removido da semente, mas se não existir
pode ser tratado como 0 (ele seria um nounce). Isso significa que o
ideal é que nossa semente tenha entre 4 e 5 valores:

\iniciocodigo
@<Avisa Tamanho Ideal da Semente@>=
#ifdef W_RNG_CHACHA20
#define _W_RNG_MINIMUM_RECOMMENDED_SEED_SIZE  4
#define _W_RNG_MAXIMUM_RECOMMENDED_SEED_SIZE  5
#endif
@
\fimcodigo

A questão é o que fazr se nossa semente for menor que isso. Faremos a
mesma coisa que fizemos com o Xoshiro: usamos o algoritmo splitmix64
para preencher os próximos à partir do valor que temos (e usamos uma
constante se não existir nenhum). Mas não se deve ter nenhuma
expenctativa de que a qualidade é mantida com uma semente reduzida
(não é, e é catastrófico se for usado para fins criptográficos):

\iniciocodigo
@<Definição de \_Wcreate\_rng@>=
#ifdef W_RNG_CHACHA20
struct _Wrng *_Wcreate_rng(void *(*allocator)(size_t), size_t size,
                           uint64_t *seed){
  int i, j;
  struct _Wrng *rng = (struct _Wrng *) allocator(sizeof(struct _Wrng));
  if(rng != NULL){
    for(i = 0; i < 4; i ++){
      if(i < size - 1 || size >= 4)
        rng -> array[i] = seed[i];
      else{
        // Using SplitMix to fill the remaining values
        uint64_t state, gamma = 0x9e3779b97f4a7c15;
        if(size > 0)
          state = seed[i];
        else
          state = 0x32147198b5436569;
        for(j = i; j < 4; j ++)
          rng -> array[j] = splitmix_next(&state, gamma);
        break;        
      }
    }
    rng -> array[4] = 0;
    if(size > 4)
      rng -> array[5] = seed[4];
    else
      rng -> array[5] = 0;
    rng -> index = 0;
    @<Inicialização de Mutex@>
  }
  return rng;
}
#endif
@
\fimcodigo

Quando tivermos que gerar um novo valor, geralmente temos valores já
gerados no vetor de 16 números de 32 bits. Com eles dá para gerarmos 8
saídas de 64 bits. Procedemos da seguinte forma quando não precisamos
gerar novos valores:

\iniciocodigo
@<Definição de \_Wrand@>=
#ifdef W_RNG_CHACHA20
@<ChaCha20: Preenchimento@>
@<ChaCha20: QuarterRound@>
@<ChaCha20: Permutação@>
uint64_t _Wrand(struct _Wrng *rng){
  uint64_t ret;
  @<Mutex:WAIT@>
  if(rng -> index % 16 == 0){
    rng -> index = 0;
    @<ChaCha20: Gera Novos Valores@>
    rng -> array[4] ++;
  }
  ret = rng -> generated_values[rng -> index];
  ret = ret << 32;
  ret += rng -> generated_values[rng -> index + 1];
  rng -> index += 2; 
  @<Mutex:SIGNAL@>
  return ret;
}
#endif
@
\fimcodigo

E finalmente a especificação fica completa mostrando como usando o seu
estado interno o gerador obtém novos valores pseudo-aleatórios:

@<ChaCha20: Gera Novos Valores@>=
{
  int i;
  uint32_t padded_array[16], copied_array[16];
  chacha_padding(rng -> array, padded_array);
  for(i = 0; i < 16; i ++)
    copied_array[i] = padded_array[i];
  chacha_permutation(padded_array);
  for(i = 0; i < 16; i ++)
    rng -> generated_values[i] = padded_array[i] + copied_array[i];
}
@

\subsecao{2.6. Estrutura Final do Arquivo}

O arquivo com o código-fonte de nossas funções terá a seguinte forma:


\iniciocodigo
@(src/random.c@>=
@<Incluir Cabeçalhos Necessários@>
#include "random.h"
#include <string.h> // memcpy
@<Funções Auxiliares@>
@<Definição de \_Wrand@>
@<Definição de \_Wcreate\_rng@>
@<Definição de \_Wdestroy\_rng@>
@
\fimcodigo

\secao{3. Testes de Qualidade}

Como forma de testar a qualidade de todos os algoritmos fornecidos e
fazer comparações, fazemos alguns testes estatísticos empíricos
sugeridos em [Knuth, 1998]. Eles são um conjunto de testes iniciais,
mas mais testes podem ser adicionados à esta seção futuramente.

Uma ressalva é que muitos dos testes sugeridos lá presume o desejo de
gerar números aleatórios entre 0 e um valor $n$, ou então valores
reais entre 0 e 1. Contudo, se assumirmos isso acabaremos dando um
peso maior para os bits mais significativos gerados. Mas queremos
garantir a qualidade mínima dele para todos os bits. Desta forma,
buscamos em diferentes testes tratar os valores gerados como uma
sequência de bits (não de números de 64 bits), e escolhemos de maneira
arbitrária, mas diferente em cada um dos testes, escolher um tamanho
em bits diferente para cada valor gerado.

Em todos os testes, aplicamos um teste como o Chi-Quadrado como
sugerido na referência [Knuth, 1998]. Repetimos as medidas 3 vezes, se
uma delas um valor aleatório tiver uma chance inferior a 1\% de ser
tão distante ou de ser tão próximo do esperado, interrompemos os
testes e consideramos uma falha. Se por duas das três vezes testadas o
valor obtido for algo com apenas 5\% de chance de ser tão distante ou
tão próximo do esperado, também consideramos uma falha. Repetimos os
três testes mil vezes e geramos assim uma porcentagem de sucesso. Como
referência, a menos que seja dito o contrário, o valor esperado de
sucesso é de 92,34\%.

Todos os testes são feitos também para a mesma semente, escolhida de
maneira elatória e uniforme. No caso, ela é a sequência de 5 valores:
\monoespaco{0x32147198b5436569}, \monoespaco{0x260287febfeb34e9},
\monoespaco{0x0b6cc94a91a265e4},
\monoespaco{0xc6a109c50dd52f1b} e \monoespaco{0x8298497f3992d73a}.

\subsecao{3.1. Teste de Equidistribuição}

Basicamente neste teste geramos uma sequência de dez mil bits
aleatórios e avaliamos o quão bem distribuído o valor está entre zeros
e uns. A taxa de sucesso para as implementações apresentadas foi:

\vbox{%A forma mais gseral de remover espaçamento entre linhas:
\baselineskip-1000pt\lineskip0pt\lineskiplimit16383.99999pt\tabskip0pt
\def\linha{\noalign{\hrule}}
\def\hidewidth{\hskip-1000pt plus 1fill}
\def\col{\hbox{\vrule height12pt depth3.5pt width0pt}}
\halign to15cm{\col#& \vrule#\tabskip=1em plus2em&
\hfil#& \vrule#& \hfil#\hfil& \vrule#&
\hfil#& \vrule#&\hfil#& \vrule#\tabskip=0pt\cr\linha
&&\omit\hidewidth Gerador\hidewidth&&\omit\hidewidth
Sucesso\hidewidth&&
\omit\hidewidth Gerador\hidewidth&&Sucesso&\cr\linha
&&SFMT&&92\%&&Xoshiro**&&91\%&\cr\linha
&&PCG&&93\%&&\monoespaco{LCG}&&93\%&\cr\linha
&&ChaCha20&&91\%&&SplitMix64&&92\%&\cr\linha}}

\subsecao{3.2. Teste Serial}

O teste serial consiste em gerar sequências de $d$ bits e verificar se
cada combinação possível ocorre com uma proporção semelhante a
$1/2^d$. No nosso caso escolhemos $d=15$. O resultado obtido foi:

\vbox{%A forma mais gseral de remover espaçamento entre linhas:
\baselineskip-1000pt\lineskip0pt\lineskiplimit16383.99999pt\tabskip0pt
\def\linha{\noalign{\hrule}}
\def\hidewidth{\hskip-1000pt plus 1fill}
\def\col{\hbox{\vrule height12pt depth3.5pt width0pt}}
\halign to15cm{\col#& \vrule#\tabskip=1em plus2em&
\hfil#& \vrule#& \hfil#\hfil& \vrule#&
\hfil#& \vrule#&\hfil#& \vrule#\tabskip=0pt\cr\linha
&&\omit\hidewidth Gerador\hidewidth&&\omit\hidewidth
Sucesso\hidewidth&&
\omit\hidewidth Gerador\hidewidth&&Sucesso&\cr\linha
&&SFMT&&92\%&&Xoshiro**&&92\%&\cr\linha
&&PCG&&93\%&&\monoespaco{LCG}&&91\%&\cr\linha
&&ChaCha20&&93\%&&SplitMix64&&93\%&\cr\linha}}

\subsecao{3.3. Teste de lacuna}

Este teste também foi sugerido em [Knuth, 1998]. Mas ao invés de fazer
a medida de lacuna interpretando os números obtidos como entre 0 e 1
como sugerido (o que talvez desse um peso alto para os bits mais
significativos), nós medimos a lacuna de bits: interpretamos os
valores retornados pelo gerados como uma sequência contínua de bits. E
contamos o tamanho das lacunas encontradas (sequências de bits 0
terminadas em bits 1).

Por exemplo, na sequência ``100110010100001'' obtemos as lacunas ``1''
(tamanho 0), ``001'' (tamanho 2), ``1'' (tamanho 0), ``001'' (tamanho
2), ``01'' (tamanho 1) e ``00001'' (tamanho 4). Em cada gerador
obtemos uma amostragem de $5\cdot2^{20}$ lacunas e avaliamos a
quantidade de cada tamanho de lacunas com um teste de
chi-quadrado. Aplicamos então a mesma avaliação dos outros testes e
medimos a taxa de sucesso. O resultado obtido foi:

\vbox{%A forma mais gseral de remover espaçamento entre linhas:
\baselineskip-1000pt\lineskip0pt\lineskiplimit16383.99999pt\tabskip0pt
\def\linha{\noalign{\hrule}}
\def\hidewidth{\hskip-1000pt plus 1fill}
\def\col{\hbox{\vrule height12pt depth3.5pt width0pt}}
\halign to15cm{\col#& \vrule#\tabskip=1em plus2em&
\hfil#& \vrule#& \hfil#\hfil& \vrule#&
\hfil#& \vrule#&\hfil#& \vrule#\tabskip=0pt\cr\linha
&&\omit\hidewidth Gerador\hidewidth&&\omit\hidewidth
Sucesso\hidewidth&&
\omit\hidewidth Gerador\hidewidth&&Sucesso&\cr\linha
&&SFMT&&92\%&&Xoshiro**&&92\%&\cr\linha
&&PCG&&91\%&&\monoespaco{LCG}&&86\%&\cr\linha
&&ChaCha20&&91\%&&SplitMix64&&90\%&\cr\linha}}

\subsecao{3.4. Teste de Pôquer}

Neste teste interpretamos o retorno de nossos geradores como
sequências de números de 0 a 15. Geramos então sucessivas tuplas com 5
elementos. E fazemos a contagem de fenômenos como tuplas em que todos
os valores são diferentes, todos são iguais, há quatro valores iguais,
há uma tripla e um par, há uma tripla, há dois pares ou há só um
par. Medimos o quanto os valores obtidos divergem do que seria
esperado com valores completamente aleatórios.

O resultado obtido foi:

\vbox{%A forma mais gseral de remover espaçamento entre linhas:
\baselineskip-1000pt\lineskip0pt\lineskiplimit16383.99999pt\tabskip0pt
\def\linha{\noalign{\hrule}}
\def\hidewidth{\hskip-1000pt plus 1fill}
\def\col{\hbox{\vrule height12pt depth3.5pt width0pt}}
\halign to15cm{\col#& \vrule#\tabskip=1em plus2em&
\hfil#& \vrule#& \hfil#\hfil& \vrule#&
\hfil#& \vrule#&\hfil#& \vrule#\tabskip=0pt\cr\linha
&&\omit\hidewidth Gerador\hidewidth&&\omit\hidewidth
Sucesso\hidewidth&&
\omit\hidewidth Gerador\hidewidth&&Sucesso&\cr\linha
&&SFMT&&92\%&&Xoshiro**&&93\%&\cr\linha
&&PCG&&91\%&&\monoespaco{LCG}&&92\%&\cr\linha
&&ChaCha20&&92\%&&SplitMix64&&91\%&\cr\linha}}

\subsecao{3.5. Teste do Colecionador}

Este teste é um complemento do anterior. Enquanto o anterior mostrava
a equidistributividade de uma mão de pôquer, este tenta medir o quão
aleatória é a ordem na qual obtemos cada valor de 4 bits. Geramos
valores novos até que tenhamos gerado os 16 valores diferentes, quando
encerramos. Fazemos isso um total de 4408394 vezes. E notamos em
quantas dessas tentativas paramos de gerar novos valores por já termos
obtido todos após 16, 17, $\ldots$, 115 ou 116+ tentativas. O teste
compara então a quantidade que cada caso ocorreu comparando com o
esperado.

O resultado do teste foi:

\vbox{%A forma mais gseral de remover espaçamento entre linhas:
\baselineskip-1000pt\lineskip0pt\lineskiplimit16383.99999pt\tabskip0pt
\def\linha{\noalign{\hrule}}
\def\hidewidth{\hskip-1000pt plus 1fill}
\def\col{\hbox{\vrule height12pt depth3.5pt width0pt}}
\halign to15cm{\col#& \vrule#\tabskip=1em plus2em&
\hfil#& \vrule#& \hfil#\hfil& \vrule#&
\hfil#& \vrule#&\hfil#& \vrule#\tabskip=0pt\cr\linha
&&\omit\hidewidth Gerador\hidewidth&&\omit\hidewidth
Sucesso\hidewidth&&
\omit\hidewidth Gerador\hidewidth&&Sucesso&\cr\linha
&&SFMT&&91\%&&Xoshiro**&&92\%&\cr\linha
&&PCG&&92\%&&\monoespaco{LCG}&&00\%&\cr\linha
&&ChaCha20&&92\%&&SplitMix64&&93\%&\cr\linha}}

Este é o primeiro teste no qual um de nossos algoritmos falha. O LCG
produz resultdos que são idealizados demais para serem considerados
aleatórios.

\subsecao{3.6. Teste da Permutação}

Este teste consiste em gerar números entre 0 e 7, ignorando valores
repetidos, até que os oito valores diferentes apareçam. Em seguida,
repetimos o teste e vamos contando todas as diferentes permutações
destes oito valores assim obtida, verificando se a distribuição das
$8!$ diferentes permutações está de acordo com o esperado.

O nosso resultado foi:

\vbox{%A forma mais gseral de remover espaçamento entre linhas:
\baselineskip-1000pt\lineskip0pt\lineskiplimit16383.99999pt\tabskip0pt
\def\linha{\noalign{\hrule}}
\def\hidewidth{\hskip-1000pt plus 1fill}
\def\col{\hbox{\vrule height12pt depth3.5pt width0pt}}
\halign to15cm{\col#& \vrule#\tabskip=1em plus2em&
\hfil#& \vrule#& \hfil#\hfil& \vrule#&
\hfil#& \vrule#&\hfil#& \vrule#\tabskip=0pt\cr\linha
&&\omit\hidewidth Gerador\hidewidth&&\omit\hidewidth
Sucesso\hidewidth&&
\omit\hidewidth Gerador\hidewidth&&Sucesso&\cr\linha
&&SFMT&&93\%&&Xoshiro**&&92\%&\cr\linha
&&PCG&&93\%&&\monoespaco{LCG}&&02\%&\cr\linha
&&ChaCha20&&92\%&&SplitMix64&&93\%&\cr\linha}}

\subsecao{3.7. Teste de Tamanho de Sequêncas Ascendentes}

Este teste, como o anterior, consiste em gerar permutações de
elementos. Mas ao invés de fazer uma contagem de todas as permutações,
ele avalia elas por meio da quantidade de elementos ascendentes, onde
o elemento gerado é maior que o anterior.

Para este teste escolhemos gerar permutações de números entre 0 e
8192. E fizemos uma análise estatística da quantidade de sequências
ascendentes de 1, 2, 3, 4, 5 ou maiores ou iguais a 6 elementos.

O resultado obtido foi:

\vbox{%A forma mais gseral de remover espaçamento entre linhas:
\baselineskip-1000pt\lineskip0pt\lineskiplimit16383.99999pt\tabskip0pt
\def\linha{\noalign{\hrule}}
\def\hidewidth{\hskip-1000pt plus 1fill}
\def\col{\hbox{\vrule height12pt depth3.5pt width0pt}}
\halign to15cm{\col#& \vrule#\tabskip=1em plus2em&
\hfil#& \vrule#& \hfil#\hfil& \vrule#&
\hfil#& \vrule#&\hfil#& \vrule#\tabskip=0pt\cr\linha
&&\omit\hidewidth Gerador\hidewidth&&\omit\hidewidth
Sucesso\hidewidth&&
\omit\hidewidth Gerador\hidewidth&&Sucesso&\cr\linha
&&SFMT&&92\%&&Xoshiro**&&91\%&\cr\linha
&&PCG&&90\%&&\monoespaco{LCG}&&90\%&\cr\linha
&&ChaCha20&&90\%&&SplitMix64&&90\%&\cr\linha}}

\subsecao{3.8. Teste do Maior de $t$}

Este teste consiste em gerar $t$ sequências de números entre 0 e algum
valor (no nosso caso foi $2^6-1$) e armazenar o maior dos valores. No
nosso caso foi escolhido $t=3$. Em seguida, compara-se a distribuição
dos maiores valores com o que é estatisticamente esperado.

O resultado foi:

\vbox{%A forma mais gseral de remover espaçamento entre linhas:
\baselineskip-1000pt\lineskip0pt\lineskiplimit16383.99999pt\tabskip0pt
\def\linha{\noalign{\hrule}}
\def\hidewidth{\hskip-1000pt plus 1fill}
\def\col{\hbox{\vrule height12pt depth3.5pt width0pt}}
\halign to15cm{\col#& \vrule#\tabskip=1em plus2em&
\hfil#& \vrule#& \hfil#\hfil& \vrule#&
\hfil#& \vrule#&\hfil#& \vrule#\tabskip=0pt\cr\linha
&&\omit\hidewidth Gerador\hidewidth&&\omit\hidewidth
Sucesso\hidewidth&&
\omit\hidewidth Gerador\hidewidth&&Sucesso&\cr\linha
&&SFMT&&92\%&&Xoshiro**&&92\%&\cr\linha
&&PCG&&91\%&&\monoespaco{LCG}&&86\%&\cr\linha
&&ChaCha20&&92\%&&SplitMix64&&92\%&\cr\linha}}

\subsecao{3.9. Teste de Colisões}

Este teste, ao contrário dos demais, não usa o teste do
Chi-Quadrado. Isso porque ele tenta medir de maneira aproximada a
ocorrência de eventos que tem uma probabilidade muito baixa para o
Chi-Quadrado. Se a saída do nosso gerador de números aleatórios fôsse
a saída de uma função hash, com que probabilidade ocorreria uma
colisão?

Este teste assume uma saída de 20 bits para o nosso gerador e gera um
total de $2^{14}$ saídas diferentes. Em seguida, o número de colisões
que ocorreram é testada para ver se ficou muito acima ou muito abaixo
do esperado. Tal teste é repetido um total de cerca de 3000 vezes,
onde o número de colisões esperada é ajustada para que a saída deste
teste tenha uma porcentagem de sucesso esperada semelhante aos testes
anteriores com o Chi-Quadrado (que é próximo de 92\%).

O resultado do teste foi:

\vbox{%A forma mais gseral de remover espaçamento entre linhas:
\baselineskip-1000pt\lineskip0pt\lineskiplimit16383.99999pt\tabskip0pt
\def\linha{\noalign{\hrule}}
\def\hidewidth{\hskip-1000pt plus 1fill}
\def\col{\hbox{\vrule height12pt depth3.5pt width0pt}}
\halign to15cm{\col#& \vrule#\tabskip=1em plus2em&
\hfil#& \vrule#& \hfil#\hfil& \vrule#&
\hfil#& \vrule#&\hfil#& \vrule#\tabskip=0pt\cr\linha
&&\omit\hidewidth Gerador\hidewidth&&\omit\hidewidth
Sucesso\hidewidth&&
\omit\hidewidth Gerador\hidewidth&&Sucesso&\cr\linha
&&SFMT&&94\%&&Xoshiro**&&93\%&\cr\linha
&&PCG&&94\%&&\monoespaco{LCG}&&92\%&\cr\linha
&&ChaCha20&&93\%&&SplitMix64&&92\%&\cr\linha}}

\subsecao{3.10. Teste de Espaçamento de Aniversário}

Neste teste nós geramos um número $n=512$ de valores de 25 bits. Em
seguida, os ordenamos e medimos a diferença entre valores
sucessivos. E contamos o número de distâncias iguais que
encontramos. O resultado é comparado com o esperado em sequências
aleatórias.

Este teste foi sugerido por George Marsaglia e quando proposto
encontrou falhas em geradores que passavam nos testes estatísticos
existentes na época.

Nosso resultado foi:

\vbox{%A forma mais gseral de remover espaçamento entre linhas:
\baselineskip-1000pt\lineskip0pt\lineskiplimit16383.99999pt\tabskip0pt
\def\linha{\noalign{\hrule}}
\def\hidewidth{\hskip-1000pt plus 1fill}
\def\col{\hbox{\vrule height12pt depth3.5pt width0pt}}
\halign to15cm{\col#& \vrule#\tabskip=1em plus2em&
\hfil#& \vrule#& \hfil#\hfil& \vrule#&
\hfil#& \vrule#&\hfil#& \vrule#\tabskip=0pt\cr\linha
&&\omit\hidewidth Gerador\hidewidth&&\omit\hidewidth
Sucesso\hidewidth&&
\omit\hidewidth Gerador\hidewidth&&Sucesso&\cr\linha
&&SFMT&&92\%&&Xoshiro**&&92\%&\cr\linha
&&PCG&&92\%&&\monoespaco{LCG}&92\%&\cr\linha
&&ChaCha20&&91\%&&SplitMix64&&91\%&\cr\linha}}

\subsecao{3.11. Teste da Correlação Serial}

Este teste gera uma sequência de $n$ elementos (onde $n=1000$) de 64
bits, que aqui vamos interpretar como elementos entre 0 e 1. Ou seja,
dividimos o resultado por $2^{61}-1$ para obter um número em ponto
flutuante. Calculamos então o coeficiente de correlação serial entre
cada um dos números gerados quando comparado com cada número anterior
e medimos se ele está dentro de um intervalo considerado
aceitável.

Em seguida, repetimos esta mesma comparação não apenas entre cada
elemento e seu sucessor, mas também entre cada elemento e o elemento
duas posições à frente. Depois três posições. Até chegar nas 500
posições de deslocamento, tentando encontrar qual nos dá o pior índice
de correlação serial. A medida do pior índice é usada e comparada com
valores esperados de forma que o esperado seja passar neste teste
derca de 93\% das vezes. E repetimos isso mil vezes para ver se a
porcentagem de testes qu passou é próxima desta.

Assim como o anterior, este teste não é uma medida de Chi-Quadrado.

O resultado obtido foi:

\vbox{%A forma mais gseral de remover espaçamento entre linhas:
\baselineskip-1000pt\lineskip0pt\lineskiplimit16383.99999pt\tabskip0pt
\def\linha{\noalign{\hrule}}
\def\hidewidth{\hskip-1000pt plus 1fill}
\def\col{\hbox{\vrule height12pt depth3.5pt width0pt}}
\halign to15cm{\col#& \vrule#\tabskip=1em plus2em&
\hfil#& \vrule#& \hfil#\hfil& \vrule#&
\hfil#& \vrule#&\hfil#& \vrule#\tabskip=0pt\cr\linha
&&\omit\hidewidth Gerador\hidewidth&&\omit\hidewidth
Sucesso\hidewidth&&
\omit\hidewidth Gerador\hidewidth&&Sucesso&\cr\linha
&&SFMT&&93\%&&Xoshiro**&&93\%&\cr\linha
&&PCG&&93\%&&\monoespaco{LCG}&93\%&\cr\linha
&&ChaCha20&&93\%&&SplitMix64&&93\%&\cr\linha}}

\subsubsecao{3.12. Generating all 32-bit values}

Este teste difere dos anteriores porque ao contrário de obter uma
porcentagem de sucesso, nós estamos interessados se é possível gerar
todos os números de 32 bits diferentes com nossos geradores se eles
executarem por tempo suficiente. Se for possível, estamos interessados
na quantidade de valores que precisaram ser gerados até termos todos
os possíveis. Consideramos que um gerador falha se ele não for capaz
de gerar algum valor.

Nosso resultado é:

\vbox{%A forma mais gseral de remover espaçamento entre linhas:
\baselineskip-1000pt\lineskip0pt\lineskiplimit16383.99999pt\tabskip0pt
\def\linha{\noalign{\hrule}}
\def\hidewidth{\hskip-1000pt plus 1fill}
\def\col{\hbox{\vrule height12pt depth3.5pt width0pt}}
\halign to15cm{\col#& \vrule#\tabskip=1em plus2em&
\hfil#& \vrule#& \hfil#\hfil& \vrule#&
\hfil#& \vrule#&\hfil#& \vrule#\tabskip=0pt\cr\linha
&&\omit\hidewidth Gerador\hidewidth&&\omit\hidewidth
Sucesso\hidewidth&&
\omit\hidewidth Gerador\hidewidth&&Sucesso&\cr\linha
&&SFMT&&98852849277&&Xoshiro**&&92842427748&\cr\linha
&&PCG&&100979563727&&\monoespaco{LCG}&8589934581&\cr\linha
&&ChaCha20&&99686167785&&&&&\cr\linha}}

Todos os algoritmos conseguiram gerar todos os valores de 32 bits
diferentes e o número de tentativas foi similar, exceto pelo LCG. O
número de valores gerados pelo LCG foi baixo demais, praticamente
metade dos valores retornados eram valores novos, não repetidos. Assim
como o teste do colecionador, isso mostra um viés conhecido deste
gerador de evitar produzir valores repetidos.


\subsecao{3.13. Conclusão dos Testes}

Mós produzimos testes que conseguiram encontrar falhas no algorithmo
LCG. Os resultados indicam um viés neste algoritmo contra a geração de
valores repetidos.

\secao{4. Medida de Desempenho}

Para medir o desempenho de cada gerador, pedimos para cada um deles
gerar sequencialmente um total de 100 milhões de números aleatórios de
64 bits. Também somamos os valores gerados apenas para evitar
otimizações do compilador que poderiam ignorar a computação. Testamos
o tempo que levou em 4 ambientes diferentes divididos em dois
computadores.

O teste abaixo foi feito em um OpenBSD 6.7 usando o Computador A
(Intel Pentium B980 dual core, 2,40 GHz, 4 GB RAM) e o compilador
Clang 8.0.1.

O resultado em segundos foi:

\vbox{%A forma mais gseral de remover espaçamento entre linhas:
\baselineskip-1000pt\lineskip0pt\lineskiplimit16383.99999pt\tabskip0pt
\def\linha{\noalign{\hrule}}
\def\hidewidth{\hskip-1000pt plus 1fill}
\def\col{\hbox{\vrule height12pt depth3.5pt width0pt}}
\halign to15cm{\col#& \vrule#\tabskip=1em plus2em&
\hfil#& \vrule#& \hfil#\hfil& \vrule#&
\hfil#& \vrule#&\hfil#& \vrule#\tabskip=0pt\cr\linha
&&\omit\hidewidth Gerador\hidewidth&&\omit\hidewidth
Tempo (s)\hidewidth&&
\omit\hidewidth Gerador\hidewidth&&Tempo (s)&\cr\linha
&&SFMT&&9.571183&&Xoshiro**&&7.811493&\cr\linha
&&PCG&&8.133946&&\monoespaco{LCG}&7.411282&\cr\linha
&&ChaCha20&&10.498551&&SplitMix64&7.529885&&\cr\linha}}

Fazendo os testes no mesmo computador, mas usando o Windows 10 e
compilando com o compilador padrão do Visual Studio (que não suporta
variáveis de 128 bits e por isso não pode executar nosso código do PCG
e Mersenne Twister), obtivemos como resultado:

\vbox{%A forma mais gseral de remover espaçamento entre linhas:
\baselineskip-1000pt\lineskip0pt\lineskiplimit16383.99999pt\tabskip0pt
\def\linha{\noalign{\hrule}}
\def\hidewidth{\hskip-1000pt plus 1fill}
\def\col{\hbox{\vrule height12pt depth3.5pt width0pt}}
\halign to15cm{\col#& \vrule#\tabskip=1em plus2em&
\hfil#& \vrule#& \hfil#\hfil& \vrule#&
\hfil#& \vrule#&\hfil#& \vrule#\tabskip=0pt\cr\linha
&&\omit\hidewidth Gerador\hidewidth&&\omit\hidewidth
Tempo (s)\hidewidth&&
\omit\hidewidth Gerador\hidewidth&&Tempo (s)&\cr\linha
&&SFMT&&-&&Xoshiro**&&4.570623&\cr\linha
&&PCG&&-&&\monoespaco{LCG}&3.615481&\cr\linha
&&ChaCha20&&8.019865&&SplitMix64&4.340963&&\cr\linha}}


Repetindo os testes em um outro computador mais novo, um Intel
i5-3210M quad core com 2,50GHz e também 4 GB de memória RAM, e desta
vez utilizando um Ubuntu 20.04.2 e um compilador GCC 9.3.0, o
resultado foi:

\vbox{%A forma mais gseral de remover espaçamento entre linhas:
\baselineskip-1000pt\lineskip0pt\lineskiplimit16383.99999pt\tabskip0pt
\def\linha{\noalign{\hrule}}
\def\hidewidth{\hskip-1000pt plus 1fill}
\def\col{\hbox{\vrule height12pt depth3.5pt width0pt}}
\halign to15cm{\col#& \vrule#\tabskip=1em plus2em&
\hfil#& \vrule#& \hfil#\hfil& \vrule#&
\hfil#& \vrule#&\hfil#& \vrule#\tabskip=0pt\cr\linha
&&\omit\hidewidth Gerador\hidewidth&&\omit\hidewidth
Tempo (s)\hidewidth&&
\omit\hidewidth Gerador\hidewidth&&Tempo (s)&\cr\linha
&&SFMT&&2.848333&&Xoshiro**&&2.565788&\cr\linha
&&PCG&&2.292101&&\monoespaco{LCG}&2.378609&\cr\linha
&&ChaCha20&&7.085438&&SplitMix64&2.437079&&\cr\linha}}

O mais surpreendente é o PCG estar rodando mais rápido que o LCG. É
inesperado tal resultado pelo fato do PCG usar internamente uma versão
de gerador linear congruente para functionar. Este também foi o único
ambiente no qual o PCG foi mais rápido que outros algoritmos como o
Xoshiro** e SplitMix64.

Usando o mesmo computador, mas compilando usando o Emscripten 2.0.14
para produzir Web Assembly e executar em uma página de Internet no
navegador  Firefox 86.0, o resultado foi:

\vbox{%A forma mais gseral de remover espaçamento entre linhas:
\baselineskip-1000pt\lineskip0pt\lineskiplimit16383.99999pt\tabskip0pt
\def\linha{\noalign{\hrule}}
\def\hidewidth{\hskip-1000pt plus 1fill}
\def\col{\hbox{\vrule height12pt depth3.5pt width0pt}}
\halign to15cm{\col#& \vrule#\tabskip=1em plus2em&
\hfil#& \vrule#& \hfil#\hfil& \vrule#&
\hfil#& \vrule#&\hfil#& \vrule#\tabskip=0pt\cr\linha
&&\omit\hidewidth Gerador\hidewidth&&\omit\hidewidth
Tempo (s)\hidewidth&&
\omit\hidewidth Gerador\hidewidth&&Tempo (s)&\cr\linha
&&SFMT&&1.318&&Xoshiro**&&0.460&\cr\linha
&&PCG&&1.324&&\monoespaco{LCG}&0.349&\cr\linha
&&ChaCha20&&3.687&&SplitMix64&0.393&&\cr\linha}}

O resultado surpreendente é o quão mais rápido este teste rodou quando
comparado com código compilado executando nativamente. Além disso, o
resultado sugere que implementações que são baseadas em variáveis de
128 bits tem um desempenho pior neste ambiente.

\secao{5. Conclusão: Escolhendo o Algoritmo Padrão}

Para escolhermos qual será o algoritmo usado por padrão em nossa
biblioteca, caso o usuário não defina macro que especifique o
algoritmo, iremos primeiro levar em conta se o algoritmo pode ser
executado na máquina ou arquitetura atual. Por exemplo, a
implementação do PCG e de Mersenne Twister que fizemos não é muito
portável, exigindo suporte do compilador para inteiros de 128
bits. Compiladores como o Clang e GCC atualmente cumprem o
requisito. Mas o Visual Studio da Microsoft não.

Também iremos levar em conta se o algoritmo passou em todos os testes
que fizemos. No momento, somente o LCG reprovou em um dos testes e por
isso apesar de tender a ser tão rápido não deverá ser a prioridade.

Por fim, levaremos em conta o desempenho deles e, se necessário, como
critério de desempate, a quantidade de memória que ocupam (o que
penaliza neste caso o Mersenne Twister que precisa de um vetor de
elementos enorme como estado interno).

Levando em conta tais critérios, se estivermos gerando código Web
Assembly, iremos usar o Xoshiro por ser o algoritmo que é mais rápido
dentre os que passaram em todos os testes neste ambiente.

Nos demais ambientes, o PCG é ligeiramente mais rápido, mas nem sempre
ele pode ser usado por ter um código não tão portável. Em tais casos,
se for possível usaremos ele, e se não for, usaremos o Xoshiro que
tende a ter um desempenho próximo do PCG mesmo:

\iniciocodigo
@<Escolhe Algoritmo Padrão do RNG@>=
#if !defined(W_RNG_MERSENNE_TWISTER) && !defined(W_RNG_XOSHIRO) && \
!defined(W_RNG_PCG) && !defined(W_RNG_LCG) && !defined(W_RNG_CHACHA20) && \
 !defined(W_RNG_SPLITMIX)
#if !defined(__SIZEOF_INT128__) || defined(__EMSCRIPTEN__)
#define W_RNG_XOSHIRO
#else
#define W_RNG_PCG
#endif
#endif
@
\fimcodigo

\secao{Referências}

\referencia{Knuth, D. E. (1984) ``Literate Programming'', The Computer Journal,
volume 27, edição 2, p. 97--111}

\referencia{Knuth, D. (1998) ``The Art of Computer Programming, v. 2:
Seminumerical Algorithms'', Addison-Wesley Professional, terceira
edição.}

\referencia{Saito, M.; Matsumoto M. (2006) ``SIMD-oriented fast Mersenne Twister: a 128-bit pseudorandom number generator'', Monte Carlo and Quasi-Monte Carlo Methods, Springer, p. 607--622}.

\referencia{Steele, G.; Vigna S. (2021) ``Computationally Easy, Spectrally Good Multipliers for Congruential Pseudorandom Number Generators'', arXiv preprint arXiv:2001.05304}.

\referencia{Matsumoto, M; Wada I.; Kuramoto A.; Ashihara, H. (2007) ``Common Defects in the Initialization of Pseudrandom Number Generators'', ACM Trans. Model. Comput. Simul, 17, 4.}

\fim
