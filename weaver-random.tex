\font\sixteen=cmbx16
\font\twelve=cmr12
\font\fonteautor=cmbx12
\font\fonteemail=cmtt10
\font\twelvenegit=cmbxti12
\font\twelvebold=cmbx12
\font\trezebold=cmbx13
\font\twelveit=cmsl12
\font\monodoze=cmtt12
\font\it=cmti12
\voffset=0,959994cm % 3,5cm de margem superior e 2,5cm inferior
\parskip=6pt

\def\titulo#1{{\noindent\sixteen\hbox to\hsize{\hfill#1\hfill}}}
\def\autor#1{{\noindent\fonteautor\hbox to\hsize{\hfill#1\hfill}}}
\def\email#1{{\noindent\fonteemail\hbox to\hsize{\hfill#1\hfill}}}
\def\negrito#1{{\twelvebold#1}}
\def\italico#1{{\twelveit#1}}
\def\monoespaco#1{{\monodoze#1}}
\def\iniciocodigo{\lineskip=0pt\parskip=0pt}
\def\fimcodigo{\twelve\parskip=0pt plus 1pt\lineskip=1pt}

\long\def\abstract#1{\parshape 10 0.8cm 13.4cm 0.8cm 13.4cm
0.8cm 13.4cm 0.8cm 13.4cm 0.8cm 13.4cm 0.8cm 13.4cm 0.8cm 13.4cm
0.8cm 13.4cm 0.8cm 13.4cm 0.8cm 13.4cm
\noindent{{\twelvenegit Abstract: }\twelveit #1}}

\def\resumo#1{\parshape  10 0.8cm 13.4cm 0.8cm 13.4cm
0.8cm 13.4cm 0.8cm 13.4cm 0.8cm 13.4cm 0.8cm 13.4cm 0.8cm 13.4cm
0.8cm 13.4cm 0.8cm 13.4cm 0.8cm 13.4cm
\noindent{{\twelvenegit Resumo: }\twelveit #1}}

\def\secao#1{\vskip12pt\noindent{\trezebold#1}\parshape 1 0cm 15cm}
\def\subsecao#1{\vskip12pt\noindent{\twelvebold#1}}
\def\referencia#1{\vskip6pt\parshape 5 0cm 15cm 0.5cm 14.5cm 0.5cm 14.5cm
0.5cm 14.5cm 0.5cm 14.5cm {\twelve\noindent#1}}

%@* .

\twelve
\vskip12pt
\titulo{Gerador de Números Aleatórios Weaver}
\vskip12pt
\autor{Thiago Leucz Astrizi}
\vskip6pt
\email{thiago@@bitbitbit.com.br}
\vskip6pt

\abstract{This article is an informal survey and analysis of random
number generators in games. It also contains a random number generator
described using literary programming.}

\vskip 0.5cm plus 3pt minus 3pt

\resumo{Este artigo é uma pesquisa informal sobre geradores de números
aleatórios em jogos. Ela também contém um gerador de números aleatórios
decrito usando programação literária.}

\secao{1. Introdução}

\subsecao{1.1. Geradores de Números Aleatórios em Jogos Eletrônicos}

Um gerador de números aleatórios é útil para os seguintes cenários em
um jogo eletrônico:

a) Simular Fenômenos: Isso se aplica tanto a fenômenos naturais que
podem ser aleatórios como a fenômonos sociais, por exemplo. Isso é
usado exaustivamente em jogos que são simuladores, como simuladores de
cidades como SimCity. Mas pode ser usado para simular variações no
clima ou direção do vento em quase qualquer tipo de jogo.

b) Amostragem: Podem existir muitos tipos diferentes de oponentes que
podem ser gerados. Gerando as características variáveis aleatoriaente
estamos simulando a amostragem de um deles dentre os vários outros
possíveis. Por exemplo, em um jogo de Pokémon, determinada espécie de
criatura pode possuir muitas variações diferentes como atributos que
dependem de sua genética ou em que tipos de golpes ela pode
realizar. Ao gerar uma criatura aleatória, obtemos uma amostragem de
uma dentre várias outras. Há o caso de simulações de mundos inteiros
gerados de maneira procedural.

c) Programação: Durante o desenvolvimento de certos tipos de jogos,
pode-se qurer testar seu funcionamento fazendo-o rodar substituindo as
opções do jogador por alguma IA que realiza suas escolhas de maneira
não-probabilística. Ou então, para se gerar níveis aleatórios de um
jogo para fins de experimentos.

d) Tomada de decisão: Em um RPG, por exemplo, um personagem poderia
conversar com algum tipo de personagem conselheiro, que diante de uma
questão, poderia responder aleatoriamente para o jogador qual ele acha
que é a melhor ação a ser tomada.

e) Estética: Adicionar aleatoriamente pequenas aleatoriedades em
padrões de imagens que de outra forma seriam regulares pode aumentar o
apelo estético de uma imagem. A aleatoriedade pode tornar também
efeitos de transição de telas mais interessantes.

f) Recreação: Um jogo eletrônico em princípio é inteiramente sobre
recreação. Mas é possível simular dentro deles jogos recreativos mais
clássicos que dependem da sorte como dados, roletas, baralhos.

Esta lista foi adaptada da apresentada em [Knuth 1998]. Todos os
exemplos apresentados nela foram adaptados acima para o cenário de
jogos eletrônicos, exceto um (análise numérica).

Como há tantos casos de uso diferentes para geradores de números
aleatórios em jogos eletrônicos, o gerador ideal usado neles deve ser
um projetado para casos bastante gerais. Um motor para criação de
jogos eletrônicos não deve fornecer um gerador otimizado para uma das
tarefas, mas que seja ruim nas demais.

Apesar disso, geralmente há uma tolerância maior para quando um
gerador de números aleatórios gera resultados não tão aleatórios em um
jogo eletrônico. As consequências são muito menos desagradáveis do que
quando isso ocorre em aplicações criptográficas, por exemplo. Muitos
jogos antigos usavam geradores que consistiam apenas em uma sequência
de números fixos. E o viés de seus geradores não era necessariamente
aparente. Por outro lado, há ocasiões nas quais um gerador ruim pode
ter consequências desagradáveis. Em simuladores que almejam realismo,
em jogos envolvendo apostas ou competições sérias, ou naqueles em que
certas ocorrências aleatórias devam ser raras e o desenvolvedor não
deseja que o gerador seja manipulado para aumentar as chances de um
jogador. Levando isso em conta, o padrão de um gerador de números
aleatórios de um jogo deve ser alto e deve-se esperar que ele passe em
testes estatísticos.

\subsecao{1.2. Programação Literária e Notação Usada no Artigo}

Este artigo utiliza a técnica de ``Programação Literária'' para
desenvolver o seu gerador de números aleatórios. Esta técnica foi apresentada
em [Knuth, 1984] e consiste e uma filosofia de desenvolvimento de
\italico{software} na qual um programador desenvolve um programa escrevendo
e explicando didaticamente o código necessário, se preocupando em
deixar o seu funcionamento claro para as pessoas que lerem a
explicação. Ferramentas automáticas são então utilizadas para extrair
o código existente na expicação, mudar a ordem do código conforme for
mais adequado para o compilador e produzir à partir do código extraído
um programa executável.

Por exemplo, neste artigo serão definidos dois arquivos
diferentes: \monoespaco{random.c} e \monoespaco{random.h}, os quais
podem ser inseridos estaticamente em qualquer projeto, ou compilados
como uma biblioteca compartilhada. O conteúdo de \monoespaco{random.h}
é:

\iniciocodigo
@(src/random.h@>=
#ifndef WEAVER_RANDOM
#define WEAVER_RANDOM
#ifdef __cplusplus
extern "C" {
#endif
#include <stdint.h>
#include <stdbool.h>
#if !defined(W_RNG_MERSENNE_TWISTER) && !defined(W_RNG_XORSHIRO)
#define W_RNG_MERSENNE_TWISTER
#endif
@<Constantes RNG@>
@<Estrutura RNG@>
@<Declarações de Gerador Aleatório@>
#ifdef __cplusplus
}
#endif
#endif
@
\fimcodigo

As duas primeiras linhas assim como a última são macros de segurança
que impedem que as funções e variáveis declaradas ali sejam declaradas
redundantemente se alguém incluir mais de uma vez o arquivo em um
código-fonte. As demais linhas contém macros que checam se estamos
compilando usando C++ ao invés de C. Se for o caso, nós declaramos
todas as funções que existem neste arquivo como funções do tipo C,
para que o compilador C++ saiba que elas não serão modificadas por
meio de sobrecarga de operadores e que por isso não é necessário
armazenar informações adicionais além do nome da função para
reconhecê-la.

No meio do código acima, deixamos indicado em letras vermelhas que
iremos adicionais mais tarde ali um novo trecho de código chamado
``Declarações de Gerador Aleatório'', com a declaração das funções que
iremos usar. Folheando o artigo, você encontrará nas páginas seguintes
um outro trecho de código cujo título não será \monoespaco{random.h},
mas sim ``Declarações de Gerador Aleatório''. Será ali que o código que
vai nesta parte do arquivo será encontrado. Pdem existir mais de um
trecho de código com este título. Isso significa que para produzir o
código funcional utilizado pelo compilador, devemos concatenar todos
estes trechos com o mesmo título e colocar na parte indicada deste
arquivo. Isso nos permite colocar a declaração de funções à medida que
formos explicando elas no código, sem precisar declarar todas de uma
vez só porque elas pertencem a um mesmo trecho de código.

\subsecao{1.4. Funções a serem Definidas}

Nosso gerador de números aleatórios deverá fornecer um total de 3
novas funções. A primeira delas irá criar a estrutura que contém o
gerador após receber a semente inicial (de 32 bits) e deve receber a
função que alocará o alocador (ou \monoespaco{NULL} para usar
o \monoespaco{malloc}):

\iniciocodigo
@<Declarações de Gerador Aleatório@>=
struct _Wrng *_Wcreate_rng(void *(*alloc)(size_t), uint32_t seed);
@
\fimcodigo

A segunda é a que efetivamente é usada para nos dar o próximo número
aleatório da sequência:

\iniciocodigo
@<Declarações de Gerador Aleatório@>+=
uint32_t _Wrand(struct _Wrng *);
@
\fimcodigo

A última função será apenas para finalizar o uso de um gerador de
números aleatórios. Ele não poderá mais ser usado depois de
finalizado:

\iniciocodigo
@<Declarações de Gerador Aleatório@>+=
bool _Wdestroy_rng(struct _Wrng *);
@
\fimcodigo

\subsecao{1.5. Suporte à Threads}

O código adicionado aqui servirá para garantir que mais de uma thread
possa usar o gerador de números randômicos sem problemas após a sua
inicialização. Para isso precisamos de um cabeçalho adequado para
declarar os tipos que iremos usar:

\iniciocodigo
@<Incluir Cabeçalhos Necessários@>+=
#if defined(__unix__) || defined(__APPLE__)
#include <pthread.h>
#endif
#if defined(_WIN32)
#include <windows.h>
#endif
@
\fimcodigo

Tudo o que serápreciso fazer é, para cada gerador, definir um mutex a
ser colocado dentro do gerador (dentro do \monoespaco{struct \_Wrng}):

\iniciocodigo
@<Declaração de Mutex@>=
#if defined(__unix__) || defined(__APPLE__)
pthread_mutex_t mutex;
#endif
#if defined(_WIN32)
CRITICAL_SECTION mutex;
#endif
@
\fimcodigo

Se temos um ponteiro para \monoespaco{struct \_Wrng} chamado
de \monoespaco{rng}, podemos inicializar seu mutex com:

\iniciocodigo
@<Inicialização de Mutex@>=
#if defined(__unix__) || defined(__APPLE__)
pthread_mutex_init(rng -> mutex, NULL);
#endif
#if defined(_WIN32)
InitializeCriticalSection(rng -> mutex);
#endif
@
\fimcodigo

Para pedirmos o uso de um Mutex, fazemos:

\iniciocodigo
@<Mutex:WAIT@>=
#if defined(__unix__) || defined(__APPLE__)
pthread_mutex_lock(rng -> mutex);
#endif
#if defined(_WIN32)
EnterCriticalSection(rng -> mutex);
#endif
@
\fimcodigo

E o código para liberarmos o uso do Mutex:

\iniciocodigo
@<Mutex:SIGNAL@>=
#if defined(__unix__) || defined(__APPLE__)
pthread_mutex_unlock(rng -> mutex);
#endif
#if defined(_WIN32)
LeaveCriticalSection(rng -> mutex);
#endif
@
\fimcodigo

Como desalocar o gerador de números aleatórios é deixado a cargo do
usuário, a única coisa que a função de finalizar o uso do gerador de
números aleatórios precisa fazer é finalizar seu Mutex. Podemos
definir o código dela inteiro aqui:

\iniciocodigo
@<Definição de \_Wdestroy\_rng@>=
bool _Wdestroy_rng(struct _Wrng *rng){
#if defined(__unix__) || defined(__APPLE__)
  return pthread_mutex_destroy(&(rng -> mutex));
#endif
#if defined(_WIN32)
  DeleteCriticalSection(rng -> mutex);
  return true; // Sempre é bem-sucedida segundo a documentação
#endif
}
@
\fimcodigo

\secao{2. Algoritmos Geradores de Números Pseudo-Randômicos}

\subsecao{2.1. SFMT}

O SFMT é o ``SIMD-Oriented Fast Mersenne Twister'', proposto pela
primeira vez em [Saito, 2006]. Ele tem esse nome porque o período
deste gerador é projetado para ter valores bastante grandes, que são
primos de Mersenne ($2^n-1$ para algum $n$ inteiro positivo). As
implementações mais usadas deste algoritmo suportam períodos de
$2^{19937}-1$ sem que haja repetição. O que é um exagero para qualquer
caso de uso, inclusive o uso em jogos eletrônicos. Já o ``SIMD'' é um
conjunto de instruções para valores de 128 bits que CPUs mais recentes
suportam. Tais instruções, se suportadas, são usadas para tornar o
desempenho mais rápido.

Como geradores de números randômicos são na prática tão rápidos, algo
que mpacta muito na velocidade de sua geração é a necessidade de
invocar funções. Há duas formas de mitigar isso: gerar uma grande
quantidade de números em um buffer, que deve ser preenchido novamente
quando eles acabarem, ou pedir para o compilador tornar a
função \italico{inline\/}. Geralmente a primeira opção é
preferida. Contudo, o momento de preencher o buffer de números será
quando a função acabará tendo uma queda de desempenho inesperada. Em
programas interativos, geralmente é melhor que as funções tenham um
comportamento mais fixo. É melhor garantir uma taxa de atualização de
tela constante do que garantir uma taxa de atualização mais rápida, só
que com quedas de desempenho ocasionais. Sendo assim, iremos preferir
usar a função do modo \italico{inline\/}. Isso não é um grande
problema devido aos algoritmos de geração de números pseudo-randômico
na prática serem bastante simples.

O SFMT funciona por meio da seguinte recursão onde cada valor $X_i$ tem
$w$ bits:

$$
X_{i+n}=g(X_i, \ldots, X_{i+n-1})
$$

A função $g$ funciona concatenando toda a sua entrada na forma de um
vetor binário em $\mathbb{F}_2$ com $wn$ elementos e multiplicando por
uma matriz $A\in\mathbb{F}_2^{wn\times w}$. Na prática essa matriz é
escolhida de modo a ser simples e assim não precisar ser representada
ou armazenada. Ao invés disso ela simplesmente existirá na forma de
rotações de bits, operações AND e operações de XOR (os quais
correspondem à multiplicação e adição no corpo $\mathbb{F}$_2).

Essa descrição genérica descreve toda a família de geradores Mersenne
Twister. Para o caso específico do SFMT, os valores que escolheremos
para nossas constantes será:

\iniciocodigo
@<Constantes RNG@>=
#ifdef W_RNG_MERSENNE_TWISTER
#define _W 128 // Tamanho de cada X_i gerado
#define _N 156 // Tamanho da recursão
#endif
@
\fimcodigo

Cada vez que geramos um novo número, temos que fazer a atualização
$w_0 \leftarrow w_1$, $w_1 \leftarrow w_2$ e assim por diante. Mas
podemos evitar ter que fazer tantas atribuições ou cópia mantendo
apenas uma variável com o índice de qual deverá ser o próximo elemento
a ser retornado pelo nosso gerador. por meio dele saberemos qual a
posição do próximo elemento de $(w_0, \ldots, w_{156})$ a ser
esquecido e substituído por um novo. Sendo assim, o formato de nossa
estrutura de RNG será:

\iniciocodigo
@<Estrutura RNG@>=
#ifdef W_RNG_MERSENNE_TWISTER
struct _Wrng{
  char w[_W * _N / 8]; // Todos N valores gerados, cada um com _W bits
  int offset;         // Index for the next 32-bit value to be returned
  @<Declaração de Mutex@>
};
#endif
@
\fimcodigo

O padrão ISO da linguagem C não suporta variáveis que garantidamente
tenham 128 bits. Contudo, versões recentes do compilador GCC e Clang
as suportam. Usando elas, podemos computar a função $g$ que computa o
próximo elemento da sequência (ou seja, a multiplicação de nosso vetor
por uma matriz em $\mathbb{F}_2$) por meio da seguintes operações:

\iniciocodigo
@<SFMT: Computa próximo elemento@>=
#ifdef __SIZEOF_INT128__
unsigned __int128 result, tmp;
uint32_t aux[4];
int i, index = rng -> offset / 4;
result = ((unsigned __int128 *) (rng -> w))[index];
result = result << 8;
result = result ^ ((unsigned __int128 *) (rng -> w))[index];
i = (index + 122) % _N;
aux[0] = *(((uint32_t *) &(((unsigned __int128 *) (rng -> w))[i])) + 0);
aux[1] = *(((uint32_t *) &(((unsigned __int128 *) (rng -> w))[i])) + 1);
aux[2] = *(((uint32_t *) &(((unsigned __int128 *) (rng -> w))[i])) + 2);
aux[3] = *(((uint32_t *) &(((unsigned __int128 *) (rng -> w))[i])) + 3);
aux[0] = (aux[0] >> 11) & 0xDFFFFFEF; // 0xBFFFFFF6
aux[1] = (aux[1] >> 11) & 0xDDFECB7F; // 0xBFFAFFFF
aux[2] = (aux[2] >> 11) & 0xBFFAFFFF; // 0xDDFECB7F
aux[3] = (aux[3] >> 11) & 0xBFFFFFF6; // 0xDFFFFFEF
memcpy(&tmp, aux, 16);
result = result ^ tmp;
i = (index + _N - 2) % _N;
result = result ^ (((unsigned __int128 *) (rng -> w))[i] >> 8);
i = (index + _N - 1) % _N;
aux[0] = *(((uint32_t *) &(((unsigned __int128 *) (rng -> w))[i])) + 0);
aux[1] = *(((uint32_t *) &(((unsigned __int128 *) (rng -> w))[i])) + 1);
aux[2] = *(((uint32_t *) &(((unsigned __int128 *) (rng -> w))[i])) + 2);
aux[3] = *(((uint32_t *) &(((unsigned __int128 *) (rng -> w))[i])) + 3);
aux[0] = (aux[0] << 18);
aux[1] = (aux[1] << 18);
aux[2] = (aux[2] << 18);
aux[3] = (aux[3] << 18);
memcpy(&tmp, aux, 16);
result = result ^ tmp;
((unsigned __int128 *) (rng -> w))[index] = result;
#else
#error "Mersenne Twister unsupported without 128 bit integer support." 
#endif
@
\fimcodigo

Essa geração de um próximo valor pseudo-randômico $w_i$ à partir de
$w_{i-1}$, $w_{i-2}$, $w_{i-N}$ e $w_{i-34}$ precisa ser então
integrada à função que efetivamente retorna o próximo número:

\iniciocodigo
@<Definição de \_Wrand@>=
#ifdef W_RNG_MERSENNE_TWISTER
uint32_t _Wrand(struct _Wrng *rng){
  uint32_t ret;
  if(rng -> offset % 4 == 0){
    @<SFMT: Computa próximo elemento@>
  }
  ret = ((uint32_t *) (rng -> w))[rng -> offset];
  rng -> offset = (rng -> offset + 1) % (_W * _N / 32);
  return ret;
}
#endif
@
\fimcodigo

Inicializar o nosso gerador é então uma questão de alocar espaço e
usar o seguinte método obtido da implementação de referência sobre
como inicializar o estado do gerador à partir da semente de 32 bits:

\iniciocodigo
@<Definição de \_Wcreate\_rng@>=
#ifdef W_RNG_MERSENNE_TWISTER
struct _Wrng *_Wcreate_rng(void *(*allocator)(size_t), uint32_t seed){
  int i;
  struct _Wrng *new = (struct _Wrng *) allocator(sizeof(struct _Wrng));
  if(new != NULL){
    uint32_t *p = (uint32_t *) (new -> w);
    p[0] = seed;
    for(i = 1; i < _N * 4; i ++){
      p[i] = 1812433253UL * (p[i - 1] ^ (p[i - 1] >> 30)) + i;
    }
    new -> offset = 0;
    @<SFMT: Garante Período@>
  }
  return new;
}
#endif
@
\fimcodigo

Entretanto, antes de retornar o código acima é preciso checar se o
estado que geramos realmente tem o período desejado de
$2^{19937}-1$. O modo de fazer isso é por meio da checagem de paridade
para alguns bits específicos de nossa semente inicial. Se a paridade
nao estiver correta, apenas ajustamos ela sem precisar mudar os demais
valores gerados:

\iniciocodigo
@<SFMT: Garante Período@>=
{
  // Máscara de bits a serem checados:
  uint32_t parity = (p[0] & 0x00000001U) ^ (p[3] & 0xc98e126aU);
  // Checagem de paridade:
  for (i = 16; i > 0; i >>= 1)
    parity ^= parity >> i;
  parity = parity & 1;
  if(parity != 1)
    p[0] = p[0] ^ 1;
}
@
\fimcodigo

\subsecao{2.2. Xorshiro256**}

Este é um gerador mais recente que o SFMT e tem como objetivo manter a
mesma qualidade, fazê-lo passar em testes estatísticos adicionais, ser
rápido e não precisar de um estado tão grande assim. O Xorshiro é um
gerador linear, mas que combina sua operação linear com operações
não-lineares adicionais para que sua linearidade não o penalize em
testes estatísticos.

Basicamente ele precisa armazenar apenas os últimos 4 valores gerados
e realizar sobre eles a seguinte operação para gerar o próximo valor:

\iniciocodigo
@<Xorshiro: Computa próximo elemento@>=
{
  uint64_t tmp = rng -> w[0] + rng -> w[3];
  rng -> result = ((tmp << 23) | (tmp >> 41)) + rng -> w[0];
} 
@
\fimcodigo

O estado formado por apenas 4 números é atualizado após a geração de
cada elemento. Isso é feito como na definição de função abaixo:

\iniciocodigo
@<Definição de \_Wrand@>=
#ifdef W_RNG_XORSHIRO
uint32_t _Wrand(struct _Wrng *rng){
uint32_t ret;
  if(rng -> offset % 2 == 0){
    uint64_t t = rng -> w[1] << 17;
    @<Xorshiro: Computa próximo elemento@>
    // Atualiza estado:
    rng -> w[2] ^= rng -> w[0];
    rng -> w[3] ^= rng -> w[1];
    rng -> w[1] ^= rng -> w[2];
    rng -> w[0] ^= rng -> w[3];
    rng -> w[2] ^= t;
    rng -> w[3] = ((rng -> w[3] << 45) | (rng -> w[3] >> 19));
  }
  ret = (uint32_t) ((rng -> result) >> (32 * rng -> offset));
  rng -> offset = (rng -> offset + 1) % 2;
  return ret;
}
#endif
@
\fimcodigo

A estrutura do gerador de números aleatórios Xorshiro então possui
apenas os últimos 4 valores de estado e o último valor de 64 bits a
ser retornado. Como nosso gerador só retorna números de 32 bits,
podemos aproveitar um mesmo valor gerado para retornar duas vezes:

\iniciocodigo
@<Estrutura RNG@>=
#ifdef W_RNG_XORSHIRO
struct _Wrng{
  uint64_t w[4];        // Valores de estado
  uint32_t result; // Last returned value
  int offset;      // If we have pending bits in result 
  @<Declaração de Mutex@>
};
#endif
@
\fimcodigo

Por fim, a inicialização do gerador Xorshiro** será:

\iniciocodigo
@<Definição de \_Wcreate\_rng@>=
#ifdef W_RNG_XORSHIRO
struct _Wrng *_Wcreate_rng(void *(*allocator)(size_t), uint32_t seed){
  int i;
  struct _Wrng *new = (struct _Wrng *) allocator(sizeof(struct _Wrng));
  if(new != NULL){
    uint32_t *p = (uint32_t *) (new -> w);
    p[0] = seed;
    for(i = 1; i < 8; i ++)
      p[i] = 1812433253UL * (p[i - 1] ^ (p[i - 1] >> 30)) + i;
    new -> offset = 0;
    new -> result = 0;
  }
  return new;
}
#endif
@
\fimcodigo


\subsecao{2.X. Estrutura Final do Arquivo}

O arquivo com o código-fonte de nossas funções terá a seguinte forma:


\iniciocodigo
@(src/random.c@>=
@<Incluir Cabeçalhos Necessários@>
#include "random.h"
#include <string.h> // memcpy
@<Definição de \_Wrand@>
@<Definição de \_Wcreate\_rng@>
@<Definição de \_Wdestroy\_rng@>
@
\fimcodigo

\secao{3. Testes de Qualidade}

\subsecao{3.1. Teste de Chi-Quadrado}

Este teste é mencionado em [Knuth, 1998]. Ele consiste em simular a
rolagem de dois dados de seis faces e obter a soma dos resultados em
ambos. Após repetir isso por um número suficiente de vezes, o número
de vezes que o número 2 e 12 é obtido não deve ser muito diferente de
1/36, o número 3 e 11 não deve ser muito mais raro que 1/18 das vezes
e assim por diante. Mas da mesma forma que não devemos obter uma
quantidade muito longe do esperado neste teste, também não devemos
receber uma quantidade precisa demais para atestar a qualidade do
gerador.

Simular rolagens de dados é uma forma importante de testar um gerador
a ser usado em jogos. Primeiro porque uma aplicação possível para o
nosso gerador pode ser justamente a simulação de dados. Uma segunda
aplicação poderia ser a implementação fiel de um sistema de RPG de
mesa que utiliza dados. Um exemplo é o sistema de ``Dungeons and
Dragons'' onde existe uma magia ``Meteor Swarm'' onde o dano é
calculado obtendo a soma de 20 lançamentos de dados de 6 faces.

Contudo, fazer a simulação de tantos lançamentos de dados para testar
a qualidade do gerador não é viável. Primeiro porque o número de
simulações de lançamentos feitos para obter uma medida confiável
aumenta exponencialmente com o número de dados usados, sendo
recomendado lançar um número de vezes tal que o resultado menos
provável possível tenha uma esperança de receber 5 resultados (segundo
Knuth). E segundo porque quanto mais dados e medidas fazendo, maiores
serão as chances de que um gerador falhe no teste, já que ele não é
exatamente a mesma coisa que um dado. Qualquer gerador
pseudo-randômico como os que usamos acaba uma hora falhando, já que os
seus números gerados não são realmente independentes. Um gerador bom
apenas consegue simular mais lançamentos antes de finalmente falhar.

Dito isso, o teste que realizamos foi simular lançamento de dados de 6
faces. Para isso lemos continuamente os bits fornecidos pelo gerador
de três em três. Em seguida, repetimos o teste do Chi-Quadrado três
vezes. Se em uma delas obtivermos $p > 99\%$ ou $p < 1\%$, o teste
falha. Se duas vezes tivermos $p> 95\%$ ou $p < 5\%$, o teste também
falha. Fizemos isso para a soma de 1, 2 e 3 dados. Os resultados
foram:

Dados    1   2    3    
SFMT:  092 085  017

\secao{Referências}

\referencia{Knuth, D. E. (1984) ``Literate Programming'', The Computer Journal,
volume 27, edição 2, p. 97--111}

\referencia{Knuth, D. (1998) ``The Art of Computer Programming, v. 2:
Seminumerical Algorithms'', Addison-Wesley Professional, terceira
edição.}

\referencia{Saito, M.; Matsumoto M. (2006) ``SIMD-oriented fast Mersenne Twister: a 128-bit pseudorandom number generator'', Monte Carlo and Quasi-Monte Carlo Methods, Springer, p. 607--622}.

\fim
