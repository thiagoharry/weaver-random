\font\sixteen=cmbx16
\font\twelve=cmr12
\font\fonteautor=cmbx12
\font\fonteemail=cmtt10
\font\twelvenegit=cmbxti12
\font\twelvebold=cmbx12
\font\trezebold=cmbx13
\font\twelveit=cmsl12
\font\monodoze=cmtt12
\font\it=cmti12
\voffset=0,959994cm % 3,5cm de margem superior e 2,5cm inferior
\parskip=6pt

\def\titulo#1{{\noindent\sixteen\hbox to\hsize{\hfill#1\hfill}}}
\def\autor#1{{\noindent\fonteautor\hbox to\hsize{\hfill#1\hfill}}}
\def\email#1{{\noindent\fonteemail\hbox to\hsize{\hfill#1\hfill}}}
\def\negrito#1{{\twelvebold#1}}
\def\italico#1{{\twelveit#1}}
\def\monoespaco#1{{\monodoze#1}}
\def\iniciocodigo{\lineskip=0pt\parskip=0pt}
\def\fimcodigo{\twelve\parskip=0pt plus 1pt\lineskip=1pt}

\long\def\abstract#1{\parshape 10 0.8cm 13.4cm 0.8cm 13.4cm
0.8cm 13.4cm 0.8cm 13.4cm 0.8cm 13.4cm 0.8cm 13.4cm 0.8cm 13.4cm
0.8cm 13.4cm 0.8cm 13.4cm 0.8cm 13.4cm
\noindent{{\twelvenegit Abstract: }\twelveit #1}}

\def\resumo#1{\parshape  10 0.8cm 13.4cm 0.8cm 13.4cm
0.8cm 13.4cm 0.8cm 13.4cm 0.8cm 13.4cm 0.8cm 13.4cm 0.8cm 13.4cm
0.8cm 13.4cm 0.8cm 13.4cm 0.8cm 13.4cm
\noindent{{\twelvenegit Resumo: }\twelveit #1}}

\def\secao#1{\vskip12pt\noindent{\trezebold#1}\parshape 1 0cm 15cm}
\def\subsecao#1{\vskip12pt\noindent{\twelvebold#1}}
\def\referencia#1{\vskip6pt\parshape 5 0cm 15cm 0.5cm 14.5cm 0.5cm 14.5cm
0.5cm 14.5cm 0.5cm 14.5cm {\twelve\noindent#1}}

%@* .

\twelve
\vskip12pt
\titulo{Gerador de Números Aleatórios Weaver}
\vskip12pt
\autor{Thiago Leucz Astrizi}
\vskip6pt
\email{thiago@@bitbitbit.com.br}
\vskip6pt

\abstract{This article is an informal analysis of some random
number generators. It contains implementation of several random number
generator algorithms and is intended to be used with Weaver game
engine, but can also be used standalone in other projects. All
different algorithms are encapsulated in the same API, therefore using
the API you can change easily the algorithm using macro definitions.}

\vskip 0.5cm plus 3pt minus 3pt

\resumo{Este artigo é uma análise informal sobre geradores de números
aleatórios. Ele contém a implementação de vários algoritmos diferentes
de geração de números pseudo-randômicos e é feito para ser usado com o
motor de jogos Weaver, mas também pode ser usado sozinho junto com
outros projetos. Todos os diferentes algoritmos são encapsulados na
mesma API, então usando esta API, é possível mudar facilmente o
algoritmo usando definição de macros.}

\secao{1. Introdução}

\subsecao{1.1. Geradores de Números Aleatórios em Jogos Eletrônicos}

Um gerador de números aleatórios é útil para os seguintes cenários em
um jogo eletrônico:

a) Simular Fenômenos: Isso se aplica tanto a fenômenos naturais que
podem ser aleatórios como a fenômonos sociais, por exemplo. Isso é
usado exaustivamente em jogos que são simuladores, como simuladores de
cidades como SimCity. Mas pode ser usado para simular variações no
clima ou direção do vento em quase qualquer tipo de jogo.

b) Amostragem: Podem existir muitos tipos diferentes de oponentes que
podem ser gerados. Gerando as características variáveis aleatoriaente
estamos simulando a amostragem de um deles dentre os vários outros
possíveis. Por exemplo, em um jogo de Pokémon, determinada espécie de
criatura pode possuir muitas variações diferentes como atributos que
dependem de sua genética ou em que tipos de golpes ela pode
realizar. Ao gerar uma criatura aleatória, obtemos uma amostragem de
uma dentre várias outras. Há o caso de simulações de mundos inteiros
gerados de maneira procedural.

c) Programação: Durante o desenvolvimento de certos tipos de jogos,
pode-se qurer testar seu funcionamento fazendo-o rodar substituindo as
opções do jogador por alguma IA que realiza suas escolhas de maneira
não-probabilística. Ou então, para se gerar níveis aleatórios de um
jogo para fins de experimentos.

d) Tomada de decisão: Em um RPG, por exemplo, um personagem poderia
conversar com algum tipo de personagem conselheiro, que diante de uma
questão, poderia responder aleatoriamente para o jogador qual ele acha
que é a melhor ação a ser tomada.

e) Estética: Adicionar aleatoriamente pequenas aleatoriedades em
padrões de imagens que de outra forma seriam regulares pode aumentar o
apelo estético de uma imagem. A aleatoriedade pode tornar também
efeitos de transição de telas mais interessantes.

f) Recreação: Um jogo eletrônico em princípio é inteiramente sobre
recreação. Mas é possível simular dentro deles jogos recreativos mais
clássicos que dependem da sorte como dados, roletas, baralhos.

Esta lista foi adaptada da apresentada em [Knuth 1998]. Todos os
exemplos apresentados nela foram adaptados acima para o cenário de
jogos eletrônicos, exceto um (análise numérica).

Como há tantos casos de uso diferentes para geradores de números
aleatórios em jogos eletrônicos, o gerador ideal usado neles deve ser
um projetado para casos bastante gerais. Um motor para criação de
jogos eletrônicos não deve fornecer um gerador otimizado para uma das
tarefas, mas que seja ruim nas demais.

Apesar disso, geralmente há uma tolerância maior para quando um
gerador de números aleatórios gera resultados não tão aleatórios em um
jogo eletrônico. As consequências são muito menos desagradáveis do que
quando isso ocorre em aplicações criptográficas, por exemplo. Muitos
jogos antigos usavam geradores que consistiam apenas em uma sequência
de números fixos. E o viés de seus geradores não era necessariamente
aparente. Por outro lado, há ocasiões nas quais um gerador ruim pode
ter consequências desagradáveis. Por exemplo, em simuladores que
almejam realismo, em jogos envolvendo apostas ou competições sérias,
ou naqueles em que certas ocorrências aleatórias devam ser raras e o
desenvolvedor não deseja que o gerador seja manipulado para aumentar
as chances de um jogador. Levando isso em conta, o padrão de um
gerador de números aleatórios de um jogo deve ser alto e deve-se
esperar que ele passe em testes estatísticos.

\subsecao{1.2. Programação Literária e Notação Usada no Artigo}

Este artigo utiliza a técnica de ``Programação Literária'' para
desenvolver o seu gerador de números aleatórios. Esta técnica foi apresentada
em [Knuth, 1984] e consiste e uma filosofia de desenvolvimento de
\italico{software} na qual um programador desenvolve um programa escrevendo
e explicando didaticamente o código necessário, se preocupando em
deixar o seu funcionamento claro para as pessoas que lerem a
explicação. Ferramentas automáticas são então utilizadas para extrair
o código existente na expicação, mudar a ordem do código conforme for
mais adequado para o compilador e produzir à partir do código extraído
um programa executável.

Por exemplo, neste artigo serão definidos dois arquivos
diferentes: \monoespaco{random.c} e \monoespaco{random.h}, os quais
podem ser inseridos estaticamente em qualquer projeto, ou compilados
como uma biblioteca compartilhada. O conteúdo de \monoespaco{random.h}
é:

\iniciocodigo
@(src/random.h@>=
#ifndef WEAVER_RANDOM
#define WEAVER_RANDOM
#ifdef __cplusplus
extern "C" {
#endif
#include <stdint.h>
#include <stdbool.h>
#if defined(__unix__) || defined(__APPLE__)
#include <pthread.h>
#endif
@<Escolhe Algoritmo Padrão do RNG@>
@<Constantes RNG@>
@<Estrutura RNG@>
@<Declarações de Gerador Aleatório@>
#ifdef __cplusplus
}
#endif
#endif
@
\fimcodigo

As duas primeiras linhas assim como a última são macros de segurança
que impedem que as funções e variáveis declaradas ali sejam declaradas
redundantemente se alguém incluir mais de uma vez o arquivo em um
código-fonte. As demais linhas contém macros que checam se estamos
compilando usando C++ ao invés de C. Se for o caso, nós declaramos
todas as funções que existem neste arquivo como funções do tipo C,
para que o compilador C++ saiba que elas não serão modificadas por
meio de sobrecarga de operadores e que por isso não é necessário
armazenar informações adicionais além do nome da função para
reconhecê-la.

No meio do código acima, deixamos indicado em letras vermelhas que
iremos adicionais mais tarde ali um novo trecho de código chamado
``Declarações de Gerador Aleatório'', com a declaração das funções que
iremos usar. Folheando o artigo, você encontrará nas páginas seguintes
um outro trecho de código cujo título não será \monoespaco{random.h},
mas sim ``Declarações de Gerador Aleatório''. Será ali que o código que
vai nesta parte do arquivo será encontrado. Pdem existir mais de um
trecho de código com este título. Isso significa que para produzir o
código funcional utilizado pelo compilador, devemos concatenar todos
estes trechos com o mesmo título e colocar na parte indicada deste
arquivo. Isso nos permite colocar a declaração de funções à medida que
formos explicando elas no código, sem precisar declarar todas de uma
vez só porque elas pertencem a um mesmo trecho de código.

\subsecao{1.4. Funções de API a serem Definidas}

Nosso gerador de números aleatórios deverá fornecer um total de 3
novas funções.

A primeira função gera e inicializa um gerador retornando um ponteiro
para ele. Para inicializá-lo, precisamos de uma função de alocação de
memória passada como primiero argumento (pode ser o \texttt{malloc} da
biblioteca padrão ou qualquer outro alocador personalizado). Em
seguida nossa semente será um valor com um tamanho de um vetor e o
petor propriamente dito de números de 64 bits que assumimos terem sido
obtidos de maneira aleatória e uniforme.

Dessa forma, nossa função de inicialização é suficientemente genérica
para que o usuário possa fornecer um número personalizado de entropia
ao gerador. Caso ele seja usado em um jogo que simula o embaralhamento
de 54 cartas, por exemplo, o ideal é que ele passe um número de bits
aleatório $n$ tal que $2^n$ não seja muito menor que $54!$, por
exemplo. Por outro lado, em algumas máquinas ou ambientes pode ser
difícil conseguir uma quantidade muito grande de bits aleatórios para
inicialização. Então, a API permite que um número menor seja passado.

\iniciocodigo
@<Declarações de Gerador Aleatório@>=
struct _Wrng *_Wcreate_rng(void *(*alloc)(size_t), size_t size, uint64_t *seed);
@
\fimcodigo

Alguns algoritmos podem, contudo, ignorar qualquer valor maior que o
recomendado ou podem não dar qualquer garantia de qualidade se
receberem uma quantidade menor de bits aleatórios em sua
inicialização. Justamente por isso, vamos fazer com que a nossa API
avise a quantidade recomendada para o tamanho do vetor da semente,
deixando a região abaixo a ser definida em breve:

\iniciocodigo
@<Declarações de Gerador Aleatório@>=
@<Avisa Tamanho Ideal da Semente@>
@
\fimcodigo


A segunda função é a que efetivamente é usada para nos dar o próximo
número aleatório da sequência, que será um elemento de 64 bits:

\iniciocodigo
@<Declarações de Gerador Aleatório@>+=
uint64_t _Wrand(struct _Wrng *);
@
\fimcodigo

A última função será apenas para finalizar o uso de um gerador de
números aleatórios. Ela recebe também um ponteiro para função que
desalocará o gerador (ou NULL). Se receber NULL, o gerador não será
desalocado (alguns alocadores podem ter seu próprio coletor de lixo
sem fornecerem função de desalocação). De qualquer forma, ele não
poderá mais ser usado depois de finalizado:

\iniciocodigo
@<Declarações de Gerador Aleatório@>+=
bool _Wdestroy_rng(void (*free)(void *), struct _Wrng *);
@
\fimcodigo

\subsecao{1.5. Suporte à Threads}

O código adicionado aqui servirá para garantir que mais de uma thread
possa usar o gerador de números randômicos sem problemas após a sua
inicialização. Para isso precisamos de um cabeçalho adequado para
declarar os tipos que iremos usar:

\iniciocodigo
@<Incluir Cabeçalhos Necessários@>+=
#if defined(__unix__) || defined(__APPLE__)
#include <pthread.h>
#endif
#if defined(_WIN32)
#include <windows.h>
#endif
@
\fimcodigo

Tudo o que serápreciso fazer é, para cada gerador, definir um mutex a
ser colocado dentro do gerador (dentro do \monoespaco{struct \_Wrng}):

\iniciocodigo
@<Declaração de Mutex@>=
#if defined(__unix__) || defined(__APPLE__)
pthread_mutex_t mutex;
#endif
#if defined(_WIN32)
CRITICAL_SECTION mutex;
#endif
@
\fimcodigo

Se temos um ponteiro para \monoespaco{struct \_Wrng} chamado
de \monoespaco{rng}, podemos inicializar seu mutex com:

\iniciocodigo
@<Inicialização de Mutex@>=
#if defined(__unix__) || defined(__APPLE__)
pthread_mutex_init(&(rng -> mutex), NULL);
#endif
#if defined(_WIN32)
InitializeCriticalSection(&(rng -> mutex));
#endif
@
\fimcodigo

Para pedirmos o uso de um Mutex, fazemos:

\iniciocodigo
@<Mutex:WAIT@>=
#if defined(__unix__) || defined(__APPLE__)
pthread_mutex_lock(&(rng -> mutex));
#endif
#if defined(_WIN32)
EnterCriticalSection(&(rng -> mutex));
#endif
@
\fimcodigo

E o código para liberarmos o uso do Mutex:

\iniciocodigo
@<Mutex:SIGNAL@>=
#if defined(__unix__) || defined(__APPLE__)
pthread_mutex_unlock(&(rng -> mutex));
#endif
#if defined(_WIN32)
LeaveCriticalSection(&(rng -> mutex));
#endif
@
\fimcodigo

Só com isso podemos definir a função que desaloca nossos geradores,
que será a mesma independente do algoritmo utilizado:

\iniciocodigo
@<Definição de \_Wdestroy\_rng@>=
bool _Wdestroy_rng(void (*free)(void *), struct _Wrng *rng){
  bool ret;
#if defined(__unix__) || defined(__APPLE__)
  ret = pthread_mutex_destroy(&(rng -> mutex));
#elif defined(_WIN32)
  DeleteCriticalSection(rng -> mutex);
  ret = true; // Sempre é bem-sucedida segundo a documentação
#endif
  if(free != NULL)
    free(rng);
  return ret;
}
@
\fimcodigo

\secao{2. Algoritmos Geradores de Números Pseudo-Randômicos}

\subsecao{2.1. Gerador Linear Congruente (LGC)}

Um Gerador Linear Congruente (também chamado pela sigla inglesa
de \italico{Livear Congruent Generator}) é simplesmente um gerador de
sequências de valores aleatórias $(x_0, x_1, \ldots)$ tal que para
determinados valores de $a, c, m$ temos que:

$$
x_{i+1}=ax_i+c \mod m
$$

A qualidade dos geradores varia enormemente de acordo com os
parâmetros escolhidos. Como queremos gerar sempre valores de 64 bits
com nossa API, vamos escolher $m=2^{64}$, o que irá facilitar bastante
a operação. Não será necessário usar o operador de módulo
explicitamente.

Com relação ao multiplicador $a$, nós usaremos o valor
0xfa346cbfd5890825 devido a este valor ser um dos listados entre
multiplicadores com uma qualidade boa para ser usado com este módulo
$m$ no artigo [Steele, 2021].

Com relação ao valor de $c$, um requisito para sua qualidade é que ele
seja ímpar (primo em relação à $m$), mas fora isso não há impacto na
qualidade se $c$ for algum valor ímpar. J'o valor inicial, também deve
ser sempre um número ímpar, mas sem outros requisitos.

Sendo assim, podemos usar os primeiros 64 bits da semente para
inicizlizar $x_0$, apenas forçando o bit menos significativo a ser
1. Se existir mais um valor de 64 bits na semente, usamos ele como
nossa constante $c$, apenas ajustando o bit menos significativo para
um. Do contrário apenas usamos $c=1$.

Isso nos dá os seguintes tamanhso recomendados para nossa semente:

\iniciocodigo
@<Avisa Tamanho Ideal da Semente@>=
#ifdef W_RNG_LCG
#define _W_RNG_MINIMUM_RECOMMENDED_SEED_SIZE  1
#define _W_RNG_MAXIMUM_RECOMMENDED_SEED_SIZE  2
#endif
@
\fimcodigo

A estrutura do nosso gerador é bastante simples, só precisa armazenar
o último valor gerado e o valor da constante $c$:

\iniciocodigo
@<Estrutura RNG@>=
#ifdef W_RNG_LCG
struct _Wrng{
  uint64_t last_value, c;
  @<Declaração de Mutex@>
};
#endif
@
\fimcodigo

E ela é inicializada com o código:

\iniciocodigo
@<Definição de \_Wcreate\_rng@>=
#ifdef W_RNG_LCG
struct _Wrng *_Wcreate_rng(void *(*allocator)(size_t), size_t size,
                           uint64_t *seed){
  struct _Wrng *rng = (struct _Wrng *) allocator(sizeof(struct _Wrng));
  if(rng != NULL){
    // If have a seed, use it, otherwise pick this constant chosen at random:
    rng -> last_value = ((size > 0)?(seed[0]):(0x1c3b9d10b1d41adc));
    rng -> c = ((size > 1)?(seed[1]):(1));
    rng -> last_value |= (1u);
    rng -> c |= (1u);
    @<Inicialização de Mutex@>
  }
  return rng;
}
#endif
@
\fimcodigo

E finalmente, podemos agora definir a função que gera os próximos
valores pseudo-randômicos:

\iniciocodigo
@<Definição de \_Wrand@>=
#ifdef W_RNG_LCG
uint64_t _Wrand(struct _Wrng *rng){
  uint64_t ret;
  @<Mutex:WAIT@>
  rng -> last_value = 0xfa346cbfd5890825 * rng -> last_value + rng -> c;
  ret = rng -> last_value;
  @<Mutex:SIGNAL@>
  return ret;
}
#endif
@
\fimcodigo

O período esperado para esse gerador é $2^{64}$, depois dessa
quantidade os valores começam a se repetir.


\subsecao{2.2. SFMT}

O SFMT é o ``SIMD-Oriented Fast Mersenne Twister'', proposto pela
primeira vez em [Saito, 2006]. Ele tem esse nome porque o período
deste gerador é projetado para ter valores bastante grandes, que são
primos de Mersenne ($2^n-1$ para algum $n$ inteiro positivo). As
implementações mais usadas deste algoritmo suportam períodos de
$2^{19937}-1$ sem que haja repetição. O que geralmente é mais que
suficiente (ou mesmo um exagero) para qualquer caso de uso. Já o
``SIMD'' é um conjunto de instruções para valores de 128 bits que CPUs
mais recentes suportam. Tais instruções, se suportadas, são usadas
para tornar o desempenho mais rápido.

O SFMT funciona por meio da seguinte recursão onde cada valor $X_i$
tem $w$ bits (no nosso caso $w=128$):

$$
X_{i+n}=g(X_i, \ldots, X_{i+n-1})
$$

A função $g$ funciona concatenando toda a sua entrada na forma de um
vetor binário em $F_2$ com $wn$ elementos e multiplicando por
uma matriz $A\in F_2^{wn\times w}$. Na prática essa matriz é
escolhida de modo a ser simples e assim não precisar ser representada
ou armazenada. Ao invés disso ela simplesmente existirá na forma de
rotações de bits, operações AND e operações de XOR (os quais
correspondem à multiplicação e adição no corpo $F_2$).

Essa descrição genérica descreve toda a família de geradores Mersenne
Twister. Para o caso específico do SFMT, os valores que escolheremos
para nossas constantes será:

\iniciocodigo
@<Constantes RNG@>=
#ifdef W_RNG_MERSENNE_TWISTER
#define _W 128 // Tamanho de cada X_i gerado
#define _N 156 // Tamanho da recursão
#endif
@
\fimcodigo

Cada vez que geramos um novo número, temos que fazer a atualização
$w_0 \leftarrow w_1$, $w_1 \leftarrow w_2$ e assim por diante (já que
o gerador usa uma recursão sobre os $n=156$ valores anteriores). Mas
podemos evitar ter que fazer tantas atribuições ou cópias mantendo
apenas uma variável (\texttt{offset}) com o índice de qual deverá ser
o próximo elemento a ser retornado pelo nosso gerador. por meio dele
saberemos qual a posição do próximo elemento de $(w_0, \ldots,
w_{156})$ a ser esquecido e substituído por um novo. Sendo assim, o
formato de nossa estrutura de RNG será:

\iniciocodigo
@<Estrutura RNG@>=
#ifdef W_RNG_MERSENNE_TWISTER
struct _Wrng{
  char w[_W * _N / 8]; // Todos N valores gerados, cada um com _W bits
  int offset;          // Index for the next 64-bit value to be returned
                       // If even, we need to generate next 128 bits
  @<Declaração de Mutex@>
};
#endif
@
\fimcodigo

O padrão ISO da linguagem C não suporta variáveis que garantidamente
tenham 128 bits. Contudo, versões recentes do compilador GCC e Clang
as suportam. Usando elas, podemos computar a função $g$ que computa o
próximo elemento da sequência (ou seja, a multiplicação de nosso vetor
por uma matriz em $F_2$) por meio da seguintes operações:

\iniciocodigo
@<SFMT: Computa próximo elemento@>=
#ifdef __SIZEOF_INT128__
unsigned __int128 result, tmp;
uint32_t aux[4];
int i, index = rng -> offset / 2;
result = ((unsigned __int128 *) (rng -> w))[index];
result = result << 8;
result = result ^ ((unsigned __int128 *) (rng -> w))[index];
i = (index + 122) % _N;
aux[0] = *(((uint32_t *) &(((unsigned __int128 *) (rng -> w))[i])) + 0);
aux[1] = *(((uint32_t *) &(((unsigned __int128 *) (rng -> w))[i])) + 1);
aux[2] = *(((uint32_t *) &(((unsigned __int128 *) (rng -> w))[i])) + 2);
aux[3] = *(((uint32_t *) &(((unsigned __int128 *) (rng -> w))[i])) + 3);
aux[0] = (aux[0] >> 11) & 0xDFFFFFEF; // 0xBFFFFFF6
aux[1] = (aux[1] >> 11) & 0xDDFECB7F; // 0xBFFAFFFF
aux[2] = (aux[2] >> 11) & 0xBFFAFFFF; // 0xDDFECB7F
aux[3] = (aux[3] >> 11) & 0xBFFFFFF6; // 0xDFFFFFEF
memcpy(&tmp, aux, 16);
result = result ^ tmp;
i = (index + _N - 2) % _N;
result = result ^ (((unsigned __int128 *) (rng -> w))[i] >> 8);
i = (index + _N - 1) % _N;
aux[0] = *(((uint32_t *) &(((unsigned __int128 *) (rng -> w))[i])) + 0);
aux[1] = *(((uint32_t *) &(((unsigned __int128 *) (rng -> w))[i])) + 1);
aux[2] = *(((uint32_t *) &(((unsigned __int128 *) (rng -> w))[i])) + 2);
aux[3] = *(((uint32_t *) &(((unsigned __int128 *) (rng -> w))[i])) + 3);
aux[0] = (aux[0] << 18);
aux[1] = (aux[1] << 18);
aux[2] = (aux[2] << 18);
aux[3] = (aux[3] << 18);
memcpy(&tmp, aux, 16);
result = result ^ tmp;
((unsigned __int128 *) (rng -> w))[index] = result;
#else
#error "Mersenne Twister unsupported without 128 bit integer support." 
#endif
@
\fimcodigo

O código acima é o que representa a multiplicação de matriz por vetor
que gera o próximo elemento. Essa geração do próximo valor
pseudo-randômico $w_i$ à partir de $w_{i-1}$, $w_{i-2}$, $w_{i-N}$ e
$w_{i-34}$ precisa ser então integrada à função que efetivamente
retorna o próximo número. Como nós só retornamos valores de 64 bits,
cada novo valor de 128 bits rende dois próximos valores a serem
retornados:

\iniciocodigo
@<Definição de \_Wrand@>=
#ifdef W_RNG_MERSENNE_TWISTER
uint64_t _Wrand(struct _Wrng *rng){
  uint64_t ret;
  @<Mutex:WAIT@>
  if(rng -> offset % 2 == 0){
    @<SFMT: Computa próximo elemento@>
  }
  ret = ((uint64_t *) (rng -> w))[rng -> offset];
  rng -> offset = (rng -> offset + 1) % (_W * _N / 64);
  @<Mutex:SIGNAL@>
  return ret;
}
#endif
@
\fimcodigo

Inicializar o nosso gerador é então ler o que existir na nossa semente
e preencher valores iniciais do vetor de estado aplicando um
mini-gerador pseudo-randômico sobre a semente. O método que usaremos é
o mesmo utilizado na implementação de referência do SFMT. Os primeiro
componente deste mini-gerador são as seguintes funções:

\iniciocodigo
@<Definição de \_Wcreate\_rng@>=
#ifdef W_RNG_MERSENNE_TWISTER
static uint32_t f1(uint32_t x){
  return (x ^ (x >> 27)) * (uint32_t) 1664525UL;
}
static uint32_t f2(uint32_t x){
  return (x ^ (x >> 27)) * (uint32_t) 1566083941UL;
}

#endif
@
\fimcodigo

A função é aplicada então na nossa inicialização abaixo. Usamos os
mesmos algoritmos utilizados para inicializar o algoritmo em sua
implementação de referência quando a semente é um vetor. Como a
especificação lida com vetores de 32 bits e aplica os algoritmos os
tratando assim, dentro da função de inicialização tratamos tais
elementos como tendo 32 bits, traduzindo assim o tamanho dos vetores
conforme apropriado:

\iniciocodigo
@<Definição de \_Wcreate\_rng@>=
#ifdef W_RNG_MERSENNE_TWISTER
struct _Wrng *_Wcreate_rng(void *(*allocator)(size_t), size_t size,
                           uint64_t *seed){
  struct _Wrng *rng = (struct _Wrng *) allocator(sizeof(struct _Wrng));
  if(rng != NULL){
    uint32_t *dst = (uint32_t *) (rng -> w), *origin = (uint32_t *) seed;
    size_t size_dst = _W * _N / 32, size_origin = size * 2;
    int count, r, i, j, mid = 306, lag = 11;
    // Preenchemos inicialmente tudo com 0x8b:
    memset(rng -> w, 0x8b, _W * _N / 8);
    count = ((size_origin + 1 >= size_dst)?(size_origin + 1):(size_dst));
    r = f1(dst[0] ^ dst[mid] ^ dst[size_dst - 1]);
    dst[mid] += r;
    r += size * 2;
    dst[mid + 11] += r;
    dst[0] = r;
    count --;
    for(i = 1, j = 0; j < count && j  < size_origin; j ++){
      r = f1(dst[i] ^  dst[(i + mid) % size_dst] ^
             dst[(size_dst - 1 + i) % size_dst]);
      dst[(i + mid) % size_dst] += r;
      r += origin[j] + i;
      dst[(i + mid + 11) % size_dst] += r;
      dst[i] = r;
      i = (i + 1) % size_dst;
    }
    for(; j < count; j++){
      r = f1(dst[i] ^ dst[(i + mid) % size_dst] ^
             dst[(i + size_dst - 1) % size_dst]);
      dst[(i + mid) % size_dst] += r;
      r += i;
      dst[(i + mid + 11) % size_dst] += r;
      dst[i] = r;
      i = (i + 1) % size_dst;
    }
    for (j = 0; j < size_dst; j++) {
      r = f2(dst[i] + dst[(i + mid) % size_dst] +
             dst[(i + size_dst - 1) % size_dst]);
      dst[(i + mid) % size_dst] ^= r;
      r -= i;
      dst[(i + mid + lag) % size_dst] ^= r;
      dst[i] = r;
      i = (i + 1) % size_dst;
    }
    rng -> offset = 0;
    @<SFMT: Garante Período@>
    @<Inicialização de Mutex@>
  }
  return rng;
}
#endif
@
\fimcodigo

Entretanto, antes de retornar o código acima é preciso checar se o
estado que geramos realmente tem o período desejado de
$2^{19937}-1$. O modo de fazer isso é por meio da checagem de paridade
para alguns bits específicos de nossa semente inicial. Se a paridade
nao estiver correta, apenas ajustamos ela sem precisar mudar os demais
valores gerados:

\iniciocodigo
@<SFMT: Garante Período@>=
{
  // Máscara de bits a serem checados:
  uint32_t parity = (dst[0] & 0x00000001U) ^ (dst[3] & 0xc98e126aU);
  // Checagem de paridade:
  for (i = 16; i > 0; i >>= 1)
    parity ^= parity >> i;
  parity = parity & 1;
  if(parity != 1)
    dst[0] = dst[0] ^ 1;
}
@
\fimcodigo

Por fi, temos que indicar qual o tamanho recomendado para a semente do
SFMT. O algoritmo não espcifica um limite muito rigoroso, mas sabemos
qu informar um número maior de bits de semente que o vetor de estado é
inútil. Desta forma, o valor mínimo pedido será 1 elemento e o máximo
será o número de elementos que preenche o vetor. Para os parâmetros
que usamos, o vetor tem 19968 bits, o que daria 312 elementos de 64
bits para preenchê-lo:

\iniciocodigo
@<Avisa Tamanho Ideal da Semente@>=
#ifdef W_RNG_MERSENNE_TWISTER
#define _W_RNG_MINIMUM_RECOMMENDED_SEED_SIZE  1
#define _W_RNG_MAXIMUM_RECOMMENDED_SEED_SIZE  312
#endif
@
\fimcodigo

\subsecao{2.3. Xorshiro256**}

Este é um gerador mais recente que o SFMT e tem como objetivo manter a
mesma qualidade, fazê-lo passar em testes estatísticos adicionais, ser
rápido e não precisar de um estado tão grande assim. O Xorshiro é um
gerador linear, mas que combina sua operação linear com operações
não-lineares adicionais para que sua linearidade não o penalize em
testes estatísticos.

Basicamente ele precisa armazenar apenas os últimos 4 valores gerados
e realizar sobre eles a seguinte operação para gerar o próximo valor:

\iniciocodigo
@<Xorshiro: Computa próximo elemento@>=
{
  uint64_t tmp = rng -> w[1] * 5;
  ret = ((tmp << 7) | (tmp >> 57)) * 9;
} 
@
\fimcodigo

O estado formado por apenas 4 números é atualizado após a geração de
cada elemento. Isso é feito como na definição de função abaixo:

\iniciocodigo
@<Definição de \_Wrand@>=
#ifdef W_RNG_XORSHIRO
uint64_t _Wrand(struct _Wrng *rng){
  uint64_t ret;
  @<Mutex:WAIT@>
  uint64_t t = rng -> w[1] << 17;
  @<Xorshiro: Computa próximo elemento@>
  // Atualiza estado:
  rng -> w[2] ^= rng -> w[0];
  rng -> w[3] ^= rng -> w[1];
  rng -> w[1] ^= rng -> w[2];
  rng -> w[0] ^= rng -> w[3];
  rng -> w[2] ^= t;
  rng -> w[3] = ((rng -> w[3] << 45) | (rng -> w[3] >> 19));
  @<Mutex:SIGNAL@>
  return ret;
}
#endif
@
\fimcodigo

A estrutura do gerador de números aleatórios Xorshiro então possui
apenas os últimos 4 valores de estado e o último valor de 64 bits a
ser retornado. Como nosso gerador só retorna números de 32 bits,
podemos aproveitar um mesmo valor gerado para retornar duas vezes:

\iniciocodigo
@<Estrutura RNG@>=
#ifdef W_RNG_XORSHIRO
struct _Wrng{
  uint64_t w[4];   // Valores de estado
  @<Declaração de Mutex@>
};
#endif
@
\fimcodigo

Por fim, a inicialização do gerador Xorshiro** consistirá em copiar
para seu estado inicial a semente e caso ela não tenha um tamanho
grande o bastante, nós a aumentamos usando o gerador ``splitmix64''
como sugerido pela implementação de referência:

\iniciocodigo
@<Definição de \_Wcreate\_rng@>=
#ifdef W_RNG_XORSHIRO
struct _Wrng *_Wcreate_rng(void *(*allocator)(size_t), size_t size,
                           uint64_t *seed){
  int i;
  struct _Wrng *rng = (struct _Wrng *) allocator(sizeof(struct _Wrng));
  if(rng != NULL){
    for(i = 0; i < 4; i ++){
      if(i < size)
        rng -> w[i] = seed[i];
      else{
        if(i == 0)
          rng -> w[i] = 0;
        else{
          uint64_t z;
          rng -> w[i - 1] += 0x9e3779b97f4a7c15; 
          z = rng -> w[i - 1];
          z = (z ^ (z >> 30)) * 0xbf58476d1ce4e5b9;
          z = (z ^ (z >> 27)) * 0x94d049bb133111eb;
          rng -> w[i] = z ^ (z >> 31);
        }
      }
    }
    @<Inicialização de Mutex@>
  }
  return rng;
}
#endif
@
\fimcodigo

Devido a isso, vamos indicar como um valor entre 1 e 4 o tamanho
recomendado para a semente deste gerador:

\iniciocodigo
@<Avisa Tamanho Ideal da Semente@>=
#ifdef W_RNG_XORSHIRO
#define _W_RNG_MINIMUM_RECOMMENDED_SEED_SIZE  1
#define _W_RNG_MAXIMUM_RECOMMENDED_SEED_SIZE  4
#endif
@
\fimcodigo


\subsecao{2.4. PCG (\italico{Permuted Congruential Generator})}

O terceiro gerador que será fornecido como opção é o PCG. Este gerador
usa ainda menos bits de estado que o Xorshiro, mas assim como o
Mersenne Twister usa variáveis de 128 bits:

\iniciocodigo
@<Estrutura RNG@>=
#ifdef W_RNG_PCG
#ifdef __SIZEOF_INT128__
struct _Wrng{
  unsigned __int128 state;
  unsigned __int128 increment; // Sempre deve ser ímpar
  @<Declaração de Mutex@>
};
#else
#error "PCG unsupported without 128 bit integer support."
#endif
#endif
@
\fimcodigo

O ideal é que ele seja inicializado com uma semente entre 128 e 256
bits, o que equivale para nós a um vetor com entre 2 e 4 números de 64
bits. Os primeiros 128 bits são a semente em si. O segundo valor serve
como uma variação, um deslocamento sobre a sequência gerada pela
semente.

Se um número maior de bits for passado como semente, os bits
adicionais serão ignorados. Se um número menor for passado, usamos o
que temos sequencialmente. Para preencher os valores faltantes: os
primeiros 64 bits são preenchidos por uma constante, usamos a versão
de 32 bits do PCG com a semente inicial para preencher os próximos 64
bits (tendo que gerar dois valores de 32 bits) e os bits restantes
também serão preenchidos por uma constante.

Sendo assim, vamos primeiro definir como eum estado (de 64 bits) é
inicializado à partir de uma semente de 64 bits:

\iniciocodigo
@<Definição de \_Wcreate\_rng@>=
#ifdef W_RNG_PCG
// Initializes 'state' with 'seed' using PCG 32 (that requires 64-bit seeds):
// The constants are derived from 0xd6b9e1860000218e chosen as a stream value
static void pcg_init32(uint64_t seed, uint64_t *state){
  const uint64_t initseq = 0xd6b9e1860000218e;
  uint64_t inc = (initseq << 1) | 1;
  *state = 0;
  *state = *state * 6364136223846793005ull + inc;
  *state += seed;
  *state = *state * 6364136223846793005ull + inc;
}
#endif
@
\fimcodigo

Depois de inicializar o gerador PCG de 32 bits, pode-se gerar novos
valores com ele usando a função abaixo:

\iniciocodigo
@<Definição de \_Wrand@>=
#ifdef W_RNG_PCG
static uint64_t pcg_rand32(uint64_t *state){
  const uint64_t initseq = 0xd6b9e1860000218e;
  uint64_t inc = (initseq << 1) | 1;
  uint64_t oldstate = *state, value, rot;
  *state = *state * 6364136223846793005ull + inc;
  value = ((oldstate >> 18u) ^ oldstate) >> 27u;
  rot = oldstate >> 59u;
  return (value >> rot) | (value << ((- rot) & 31));
}
#endif
@
\fimcodigo

Tudo isso é para que possamos inicializar de alguma forma nossa
semente se recebermos uma semente com menos de 128 bits no verdadeiro
inicializador do PCG de 64 bits:

\iniciocodigo
@<Definição de \_Wcreate\_rng@>=
#ifdef W_RNG_PCG
struct _Wrng *_Wcreate_rng(void *(*allocator)(size_t), size_t size,
                           uint64_t *seed){
  struct _Wrng *rng = (struct _Wrng *) allocator(sizeof(struct _Wrng));
  unsigned __int128 initstate, initseq, multiplier;
  if(rng != NULL){
    if(size > 0)
      initstate = seed[0];
    else // If not specified, use this value chosen randomly
      initstate = 0xf1168ab461218bb5;
    initstate = initstate << 64;
    if(size > 1)
      initstate += seed[1];
    else{ // If not specified, fill with values from PCG of 32 bits
      uint64_t state, value;
      pcg_init32(seed[0], &state);
      value = pcg_rand32(&state);
      value = value << 32;
      value += pcg_rand32(&state);
      initstate += value;;
    }
    if(size > 2)
      initseq = seed[2];
    else // If not specified, use this constant chosen randomly:
      initseq = 0x4ecbb136926970a4;
    initseq = initseq << 64;
    if(size > 3)
      initseq += seed[3];
    else // If not specified, use this constant chosen randomly:
      initseq += 0x25b22530f844f87b;
    // Inicializando o estado inicial:
    multiplier = 2549297995355413924ULL;
    multiplier = multiplier << 64;
    multiplier += 4865540595714422341ULL;
    rng -> state = 0;
    rng -> increment = (initseq << 1) | 1;
    rng->state = rng->state * multiplier + rng->increment;
    rng -> state += initstate;
    rng->state = rng->state * multiplier + rng->increment;
    @<Inicialização de Mutex@>
  }
  return rng;
}
#endif
@
\fimcodigo

Para evitar inicializar o gerador de maniera não-recomendada,
precisamos de pelo menos 128 bits. E no máximo 256, já que qualquer
valor restante é ignorado:

\iniciocodigo
@<Avisa Tamanho Ideal da Semente@>=
#ifdef W_RNG_PCG
#define _W_RNG_MINIMUM_RECOMMENDED_SEED_SIZE  2
#define _W_RNG_MAXIMUM_RECOMMENDED_SEED_SIZE  4
#endif
@
\fimcodigo


E por fim, geramos um novos números aleatórios com o PCG de 64 bits
por meio do seguinte código:

\iniciocodigo
@<Definição de \_Wrand@>=
#ifdef W_RNG_PCG
uint64_t _Wrand(struct _Wrng *rng){
  unsigned __int128 multiplier;
  uint64_t ret, xorshifted, rot;
  multiplier = 2549297995355413924ULL;
  multiplier = multiplier << 64;
  multiplier += 4865540595714422341ULL;
  @<Mutex:WAIT@>
  // Advance internal state
  rng -> state = rng -> state * multiplier + rng -> increment;
  xorshifted = (((uint64_t)(rng -> state >> 64u)) ^ ((uint64_t) rng -> state));
  rot = rng -> state >> 122u;
  ret = (xorshifted >> rot) | (xorshifted << ((-rot) & 63));
  @<Mutex:SIGNAL@>
  return ret;
}
#endif
@
\fimcodigo

\subsecao{2.5. ChaCha20}

O ChaCha20 é um gerador projetado para ser criptograficamente
seguro. Mas isso só é verdade se ele for alimentado com ua semente com
um tamanho adequado, e se a semente realmente for obtida de maneira
aleatória e uniforme, sem ser reaproveitada novamente. A implementação
aqui feita, contudo, não deve ser usada para fins criptográficos
sérios. Ela está sendo projetada para ser usada para fins de testes e
também em projetos onde a segurança do gerador não é crítica. O motivo
é que há implementações muito mais testadas e examidadas que
esta. Aqui não estamos nos preocupando com detalhes como ataques de
canais laterais ou coisas que são importantes para usos sérios da
criptografia, então pode ser que a implementação aqui feita tenha
falhas de segurança. A ideia de implementar esta função é apenas
garantir uma alternativa de gerador para aqueles que querem uma
qualidade de geração muito maior que o normal, mesmo que tenham que
pagar caro em performance para isso.

O ChaCha20 tem como primeiro componente de seu funcionamento uma
função de preenchimento, que receberá como entrada 384 bits (a serem
interpretados como separados em componentes de 64 bits) e gera como
saída 512 bits. A ideia é que a saída seja interpretada como uma
matriz $4\times 4$ de elementos de 32 bits, mesmo que não a
armazenemos neste formato:

@<ChaCha20: Preenchimento@>=
#ifdef W_RNG_CHACHA20
static void chacha_padding(uint64_t input[6], uint32_t output[16]){
  int i, j;
  output[0] = ('e' << 24) + ('x' << 16) + ('p' << 8) + 'a';
  output[1] = ('n' << 24) + ('d' << 16) + (' ' << 8) + '3';
  output[2] = ('2' << 24) + ('-' << 16) + ('b' << 8) + 'y';
  output[3] = ('t' << 24) + ('e' << 16) + (' ' << 8) + 'k';
  for(j=4, i = 0; i < 6; i ++, j += 2){
    output[j] = (input[i] / 4294967296llu);
    output[j+1] = input[i] % 4294967296llu;
  }
}
#endif
@

Na entrada dessa função de preenchimento, os primeiros 4 valores
sempre são retirados da semente inicial. O próximo é sempre um
contador que no primeiro bloco gerado é 0, e a cada próximo incrementa
em 1. E o último é um número que deve ser sempre usado uma só vez.

O próximo componente é a função abaixo que recebe 4 valores de 32bits
e os modifica realizando uma série de operações:

@<ChaCha20: QuarterRound@>=
#ifdef W_RNG_CHACHA20
void quarter_round(uint32_t *a, uint32_t *b, uint32_t *c, uint32_t *d){
  *a = *a + *b;
  *d = *d ^ *a;
  *d = (*d << 16) | (*d >> 16);
  *c = *c + *d;
  *b = *b ^ *c;
  *b = (*b << 12) | (*b >> 20);
  *a = *a + *b;
  *d = *d ^ *a;
  *d = (*d << 8) | (*d >> 24);
  *c = *c + *d;
  *b = *b ^ *c;
  *b = (*b << 7) | (*b >> 25);
}
#endif
@

A operação anterior é utilizada na seguinte função de permutação que
recebe como entrada 16 elementos de 32 bits e os transforma:

@<ChaCha20: Permutação@>=
#ifdef W_RNG_CHACHA20
void chacha_permutation(uint32_t elements[16]){
  int i;
  for(i = 0; i < 10; i ++){
    quarter_round(&elements[0], &elements[4], &elements[8], &elements[12]);
    quarter_round(&elements[1], &elements[5], &elements[9], &elements[13]);
    quarter_round(&elements[2], &elements[6], &elements[10], &elements[14]);
    quarter_round(&elements[3], &elements[7], &elements[11], &elements[15]);
    quarter_round(&elements[0], &elements[5], &elements[10], &elements[15]);
    quarter_round(&elements[1], &elements[6], &elements[11], &elements[12]);
    quarter_round(&elements[2], &elements[7], &elements[8], &elements[13]);
    quarter_round(&elements[3], &elements[4], &elements[9], &elements[14]);
  }
}
#endif
@

Com esses pré-requisitos podemos começar a definir a estrutura e
funções de nossa API. A estrutura do gerador deverá armazenar um vetor
com 6 elementos de 64 bits (que depois serão passados para a função
que gera novos números após antes passar pelo preenchimento), um
espaço de 512 bits com os últimos valores gerados (se existirem) e um
índice para saber qual o próximo valor que deve ser retornado. A
estrutura será então:

\iniciocodigo
@<Estrutura RNG@>=
#ifdef W_RNG_CHACHA20
struct _Wrng{
  uint64_t array[6];
  uint32_t generated_values[16];
  int index;
  @<Declaração de Mutex@>
};
#endif
@
\fimcodigo

Quando a estrutura é alocada, os primeiros 4 valores do nosso vetor de
6 elementos são copiados da semente, os próximo valor é preenchido
como 0 e o último também é removido da semente, mas se não existir
pode ser tratado como 0 (ele seria um nounce). Isso significa que o
ideal é que nossa semente tenha entre 4 e 5 valores:

\iniciocodigo
@<Avisa Tamanho Ideal da Semente@>=
#ifdef W_RNG_CHACHA20
#define _W_RNG_MINIMUM_RECOMMENDED_SEED_SIZE  4
#define _W_RNG_MAXIMUM_RECOMMENDED_SEED_SIZE  5
#endif
@
\fimcodigo

A questão é o que fazr se nossa semente for menor que isso. Faremos a
mesma coisa que fizemos com o Xorshiro: usamos o algoritmo splitmix64
para preencher os próximos à partir do valor que temos (e usamos uma
constante se não existir nenhum). Mas não se deve ter nenhuma
expenctativa de que a qualidade é mantida com uma semente reduzida
(não é, e é catastrófico se for usado para fins criptográficos):

\iniciocodigo
@<Definição de \_Wcreate\_rng@>=
#ifdef W_RNG_CHACHA20
struct _Wrng *_Wcreate_rng(void *(*allocator)(size_t), size_t size,
                           uint64_t *seed){
  int i;
  struct _Wrng *rng = (struct _Wrng *) allocator(sizeof(struct _Wrng));
  if(rng != NULL){
    for(i = 0; i < 4; i ++){
      if(i < size)
        rng -> array[i] = seed[i];
      else{
        if(i == 0)
          rng -> array[i] = 0;
        else{
          uint64_t z;
          rng -> array[i - 1] += 0x9e3779b97f4a7c15; 
          z = rng -> array[i - 1];
          z = (z ^ (z >> 30)) * 0xbf58476d1ce4e5b9;
          z = (z ^ (z >> 27)) * 0x94d049bb133111eb;
          rng -> array[i] = z ^ (z >> 31);
        }
      }
    }
    rng -> array[4] = 0;
    if(size > 4)
      rng -> array[5] = seed[4];
    else
      rng -> array[5] = 0;
    rng -> index = 0;
    @<Inicialização de Mutex@>
  }
  return rng;
}
#endif
@
\fimcodigo

Quando tivermos que gerar um novo valor, geralmente temos valores já
gerados no vetor de 16 números de 32 bits. Com eles dá para gerarmos 8
saídas de 64 bits. Procedemos da seguinte forma quando não precisamos
gerar novos valores:

\iniciocodigo
@<Definição de \_Wrand@>=
#ifdef W_RNG_CHACHA20
@<ChaCha20: Preenchimento@>
@<ChaCha20: QuarterRound@>
@<ChaCha20: Permutação@>
uint64_t _Wrand(struct _Wrng *rng){
  uint64_t ret;
  @<Mutex:WAIT@>
  if(rng -> index % 16 == 0){
    rng -> index = 0;
    @<ChaCha20: Gera Novos Valores@>
    rng -> array[4] ++;
  }
  ret = rng -> generated_values[rng -> index];
  ret = ret << 32;
  ret += rng -> generated_values[rng -> index + 1];
  rng -> index += 2; 
  @<Mutex:SIGNAL@>
  return ret;
}
#endif
@
\fimcodigo

E finalmente a especificação fica completa mostrando como usando o seu
estado interno o gerador obtém novos valores pseudo-aleatórios:

@<ChaCha20: Gera Novos Valores@>=
{
  int i;
  uint32_t padded_array[16], copied_array[16];
  chacha_padding(rng -> array, padded_array);
  for(i = 0; i < 16; i ++)
    copied_array[i] = padded_array[i];
  chacha_permutation(padded_array);
  for(i = 0; i < 16; i ++)
    rng -> generated_values[i] = padded_array[i] + copied_array[i];
}
@

\subsecao{2.6. Estrutura Final do Arquivo}

O arquivo com o código-fonte de nossas funções terá a seguinte forma:


\iniciocodigo
@(src/random.c@>=
@<Incluir Cabeçalhos Necessários@>
#include "random.h"
#include <string.h> // memcpy
@<Definição de \_Wrand@>
@<Definição de \_Wcreate\_rng@>
@<Definição de \_Wdestroy\_rng@>
@
\fimcodigo

\secao{3. Testes de Qualidade}

Como forma de testar a qualidade de todos os algoritmos fornecidos e
fazer comparações, fazemos alguns testes estatísticos empíricos
sugeridos em [Knuth, 1998]. Uma ressalva é que muitos dos testes
sugeridos lá presume o desejo de gerar números aleatórios entre 0 e um
valor $n$, ou então valores reais entre 0 e 1. Contudo, se assumirmos
isso acabaremos dando um peso maior para os bits mais significativos
gerados. Mas queremos garantir a qualidade mínima dele para todos os
bits. Desta forma, buscamos em diferentes testes tratar os valores
gerados como uma sequência de bits (não de números de 64 bits), e
escolhemos de maneira arbitrária, mas diferente em cada um dos testes,
escolher um tamanho em bits diferente para cada valor gerado.

Em todos os testes, aplicamos um teste como o Chi-Quadrado como
sugerido na referência [Knuth, 1998]. Repetimos as medidas 3 vezes, se
uma delas um valor aleatório tiver uma chance inferior a 1\% de ser
tão distante ou de ser tão próximo do esperado, interrompemos os
testes e consideramos uma falha. Se por duas das três vezes testadas o
valor obtido for algo com apenas 5\% de chance de ser tão distante ou
tão próximo do esperado, também consideramos uma falha. Repetimos os
três testes mil vezes e geramos assim uma porcentagem de sucesso. Como
referência, a menos que seja dito o contrário, o valor esperado de
sucesso é de 92,34\%.

Todos os testes são feitos também para a mesma semente, escolhida de
maneira elatória e uniforme. No caso, ela é a sequência de 5 valores:
\monoespaco{0x32147198b5436569}, \monoespaco{0x260287febfeb34e9},
\monoespaco{0x0b6cc94a91a265e4},
\monoespaco{0xc6a109c50dd52f1b} e \monoespaco{0x8298497f3992d73a}.

\subsecao{3.1. Teste de Equidistribuição}

Basicamente neste teste geramos uma sequência de dez mil bits
aleatórios e avaliamos o quão bem distribuído o valor está entre zeros
e uns. A taxa de sucesso para as implementações apresentadas foi:

\vbox{%A forma mais gseral de remover espaçamento entre linhas:
\baselineskip-1000pt\lineskip0pt\lineskiplimit16383.99999pt\tabskip0pt
\def\linha{\noalign{\hrule}}
\def\hidewidth{\hskip-1000pt plus 1fill}
\def\col{\hbox{\vrule height12pt depth3.5pt width0pt}}
\halign to15cm{\col#& \vrule#\tabskip=1em plus2em&
\hfil#& \vrule#& \hfil#\hfil& \vrule#&
\hfil#& \vrule#&\hfil#& \vrule#\tabskip=0pt\cr\linha
&&\omit\hidewidth Gerador\hidewidth&&\omit\hidewidth
Sucesso\hidewidth&&
\omit\hidewidth Gerador\hidewidth&&Sucesso&\cr\linha
&&SFMT&&92\%&&Xorshiro**&&91\%&\cr\linha
&&PCG&&93\%&&\monoespaco{LCG}&&93\%&\cr\linha
&&ChaCha20&&91\%&&&&&\cr\linha}}

\subsecao{3.2. Teste Serial}

O teste serial consiste em gerar sequências de $d$ bits e verificar se
cada combinação possível ocorre com uma proporção semelhante a
$1/2^d$. No nosso caso escolhemos $d=15$. O resultado obtido foi:

\vbox{%A forma mais gseral de remover espaçamento entre linhas:
\baselineskip-1000pt\lineskip0pt\lineskiplimit16383.99999pt\tabskip0pt
\def\linha{\noalign{\hrule}}
\def\hidewidth{\hskip-1000pt plus 1fill}
\def\col{\hbox{\vrule height12pt depth3.5pt width0pt}}
\halign to15cm{\col#& \vrule#\tabskip=1em plus2em&
\hfil#& \vrule#& \hfil#\hfil& \vrule#&
\hfil#& \vrule#&\hfil#& \vrule#\tabskip=0pt\cr\linha
&&\omit\hidewidth Gerador\hidewidth&&\omit\hidewidth
Sucesso\hidewidth&&
\omit\hidewidth Gerador\hidewidth&&Sucesso&\cr\linha
&&SFMT&&92\%&&Xorshiro**&&92\%&\cr\linha
&&PCG&&93\%&&\monoespaco{LCG}&&91\%&\cr\linha
&&ChaCha20&&93\%&&&&&\cr\linha}}

\subsecao{3.3. Teste de lacuna}

Este teste também foi sugerido em [Knuth, 1998]. Mas ao invés de fazer
a medida de lacuna interpretando os números obtidos como entre 0 e 1
como sugerido (o que talvez desse um peso alto para os bits mais
significativos), nós medimos a lacuna de bits: interpretamos os
valores retornados pelo gerados como uma sequência contínua de bits. E
contamos o tamanho das lacunas encontradas (sequências de bits 0
terminadas em bits 1).

Por exemplo, na sequência ``100110010100001'' obtemos as lacunas ``1''
(tamanho 0), ``001'' (tamanho 2), ``1'' (tamanho 0), ``001'' (tamanho
2), ``01'' (tamanho 1) e ``00001'' (tamanho 4). Em cada gerador
obtemos uma amostragem de $5\cdot2^{20}$ lacunas e avaliamos a
quantidade de cada tamanho de lacunas com um teste de
chi-quadrado. Aplicamos então a mesma avaliação dos outros testes e
medimos a taxa de sucesso. O resultado obtido foi:

\vbox{%A forma mais gseral de remover espaçamento entre linhas:
\baselineskip-1000pt\lineskip0pt\lineskiplimit16383.99999pt\tabskip0pt
\def\linha{\noalign{\hrule}}
\def\hidewidth{\hskip-1000pt plus 1fill}
\def\col{\hbox{\vrule height12pt depth3.5pt width0pt}}
\halign to15cm{\col#& \vrule#\tabskip=1em plus2em&
\hfil#& \vrule#& \hfil#\hfil& \vrule#&
\hfil#& \vrule#&\hfil#& \vrule#\tabskip=0pt\cr\linha
&&\omit\hidewidth Gerador\hidewidth&&\omit\hidewidth
Sucesso\hidewidth&&
\omit\hidewidth Gerador\hidewidth&&Sucesso&\cr\linha
&&SFMT&&92\%&&Xorshiro**&&92\%&\cr\linha
&&PCG&&91\%&&\monoespaco{LCG}&&86\%&\cr\linha
&&ChaCha20&&91\%&&&&&\cr\linha}}

\subsecao{3.4. Teste de Pôquer}

Neste teste interpretamos o retorno de nossos geradores como
sequências de números de 0 a 15. Geramos então sucessivas tuplas com 5
elementos. E fazemos a contagem de fenômenos como tuplas em que todos
os valores são diferentes, todos são iguais, há quatro valores iguais,
há uma tripla e um par, há uma tripla, há dois pares ou há só um
par. Medimos o quanto os valores obtidos divergem do que seria
esperado com valores completamente aleatórios.

O resultado obtido foi:

\vbox{%A forma mais gseral de remover espaçamento entre linhas:
\baselineskip-1000pt\lineskip0pt\lineskiplimit16383.99999pt\tabskip0pt
\def\linha{\noalign{\hrule}}
\def\hidewidth{\hskip-1000pt plus 1fill}
\def\col{\hbox{\vrule height12pt depth3.5pt width0pt}}
\halign to15cm{\col#& \vrule#\tabskip=1em plus2em&
\hfil#& \vrule#& \hfil#\hfil& \vrule#&
\hfil#& \vrule#&\hfil#& \vrule#\tabskip=0pt\cr\linha
&&\omit\hidewidth Gerador\hidewidth&&\omit\hidewidth
Sucesso\hidewidth&&
\omit\hidewidth Gerador\hidewidth&&Sucesso&\cr\linha
&&SFMT&&92\%&&Xorshiro**&&93\%&\cr\linha
&&PCG&&91\%&&\monoespaco{LCG}&&92\%&\cr\linha
&&ChaCha20&&92\%&&&&&\cr\linha}}

\subsecao{3.5. Teste do Colecionador}

Este teste é um complemento do anterior. Enquanto o anterior mostrava
a equidistributividade de uma mão de pôquer, este tenta medir o quão
aleatória é a ordem na qual obtemos cada valor de 4 bits. Geramos
valores novos até que tenhamos gerado os 16 valores diferentes, quando
encerramos. Fazemos isso um total de 4408394 vezes. E notamos em
quantas dessas tentativas paramos de gerar novos valores por já termos
obtido todos após 16, 17, $\ldots$, 115 ou 116+ tentativas. O teste
compara então a quantidade que cada caso ocorreu comparando com o
esperado.

O resultado do teste foi:

\vbox{%A forma mais gseral de remover espaçamento entre linhas:
\baselineskip-1000pt\lineskip0pt\lineskiplimit16383.99999pt\tabskip0pt
\def\linha{\noalign{\hrule}}
\def\hidewidth{\hskip-1000pt plus 1fill}
\def\col{\hbox{\vrule height12pt depth3.5pt width0pt}}
\halign to15cm{\col#& \vrule#\tabskip=1em plus2em&
\hfil#& \vrule#& \hfil#\hfil& \vrule#&
\hfil#& \vrule#&\hfil#& \vrule#\tabskip=0pt\cr\linha
&&\omit\hidewidth Gerador\hidewidth&&\omit\hidewidth
Sucesso\hidewidth&&
\omit\hidewidth Gerador\hidewidth&&Sucesso&\cr\linha
&&SFMT&&91\%&&Xorshiro**&&92\%&\cr\linha
&&PCG&&92\%&&\monoespaco{LCG}&&00\%&\cr\linha
&&ChaCha20&&92\%&&&&&\cr\linha}}

\subsecao{3.6. Teste da Permutação}

Este teste consiste em gerar números entre 0 e 7, ignorando valores
repetidos, até que os oito valores diferentes apareçam. Em seguida,
repetimos o teste e vamos contando todas as diferentes permutações
destes oito valores assim obtida, verificando se a distribuição das
$8!$ diferentes permutações está de acordo com o esperado.

O nosso resultado foi:

\vbox{%A forma mais gseral de remover espaçamento entre linhas:
\baselineskip-1000pt\lineskip0pt\lineskiplimit16383.99999pt\tabskip0pt
\def\linha{\noalign{\hrule}}
\def\hidewidth{\hskip-1000pt plus 1fill}
\def\col{\hbox{\vrule height12pt depth3.5pt width0pt}}
\halign to15cm{\col#& \vrule#\tabskip=1em plus2em&
\hfil#& \vrule#& \hfil#\hfil& \vrule#&
\hfil#& \vrule#&\hfil#& \vrule#\tabskip=0pt\cr\linha
&&\omit\hidewidth Gerador\hidewidth&&\omit\hidewidth
Sucesso\hidewidth&&
\omit\hidewidth Gerador\hidewidth&&Sucesso&\cr\linha
&&SFMT&&93\%&&Xorshiro**&&92\%&\cr\linha
&&PCG&&93\%&&\monoespaco{LCG}&&02\%&\cr\linha
&&ChaCha20&&92\%&&&&&\cr\linha}}

\subsecao{3.7. Teste de Tamanho de Sequêncas Ascendentes}

Este teste, como o anterior, consiste em gerar permutações de
elementos. Mas ao invés de fazer uma contagem de todas as permutações,
ele avalia elas por meio da quantidade de elementos ascendentes, onde
o elemento gerado é maior que o anterior.

Para este teste escolhemos gerar permutações de números entre 0 e
8192. E fizemos uma análise estatística da quantidade de sequências
ascendentes de 1, 2, 3, 4, 5 ou maiores ou iguais a 6 elementos.

O resultado obtido foi:

\vbox{%A forma mais gseral de remover espaçamento entre linhas:
\baselineskip-1000pt\lineskip0pt\lineskiplimit16383.99999pt\tabskip0pt
\def\linha{\noalign{\hrule}}
\def\hidewidth{\hskip-1000pt plus 1fill}
\def\col{\hbox{\vrule height12pt depth3.5pt width0pt}}
\halign to15cm{\col#& \vrule#\tabskip=1em plus2em&
\hfil#& \vrule#& \hfil#\hfil& \vrule#&
\hfil#& \vrule#&\hfil#& \vrule#\tabskip=0pt\cr\linha
&&\omit\hidewidth Gerador\hidewidth&&\omit\hidewidth
Sucesso\hidewidth&&
\omit\hidewidth Gerador\hidewidth&&Sucesso&\cr\linha
&&SFMT&&92\%&&Xorshiro**&&91\%&\cr\linha
&&PCG&&90\%&&\monoespaco{LCG}&&90\%&\cr\linha
&&ChaCha20&&90\%&&&&&\cr\linha}}

\subsecao{3.8. Teste do Maior de $t$}

Este teste consiste em gerar $t$ sequências de números entre 0 e algum
valor (no nosso caso foi $2^6-1$) e armazenar o maior dos valores. No
nosso caso foi escolhido $t=3$. Em seguida, compara-se a distribuição
dos maiores valores com o que é estatisticamente esperado.

O resultado foi:

\vbox{%A forma mais gseral de remover espaçamento entre linhas:
\baselineskip-1000pt\lineskip0pt\lineskiplimit16383.99999pt\tabskip0pt
\def\linha{\noalign{\hrule}}
\def\hidewidth{\hskip-1000pt plus 1fill}
\def\col{\hbox{\vrule height12pt depth3.5pt width0pt}}
\halign to15cm{\col#& \vrule#\tabskip=1em plus2em&
\hfil#& \vrule#& \hfil#\hfil& \vrule#&
\hfil#& \vrule#&\hfil#& \vrule#\tabskip=0pt\cr\linha
&&\omit\hidewidth Gerador\hidewidth&&\omit\hidewidth
Sucesso\hidewidth&&
\omit\hidewidth Gerador\hidewidth&&Sucesso&\cr\linha
&&SFMT&&92\%&&Xorshiro**&&92\%&\cr\linha
&&PCG&&91\%&&\monoespaco{LCG}&&86\%&\cr\linha
&&ChaCha20&&92\%&&&&&\cr\linha}}

\subsecao{3.9. Teste de Colisões}

Este teste, ao contrário dos demais, não usa o teste do
Chi-Quadrado. Isso porque ele tenta medir de maneira aproximada a
ocorrência de eventos que tem uma probabilidade muito baixa para o
Chi-Quadrado. Se a saída do nosso gerador de números aleatórios fôsse
a saída de uma função hash, com que probabilidade ocorreria uma
colisão?

Este teste assume uma saída de 20 bits para o nosso gerador e gera um
total de $2^{14}$ saídas diferentes. Em seguida, o número de colisões
que ocorreram é testada para ver se ficou muito acima ou muito abaixo
do esperado. Tal teste é repetido um total de cerca de 3000 vezes,
onde o número de colisões esperada é ajustada para que a saída deste
teste tenha uma porcentagem de sucesso esperada semelhante aos testes
anteriores com o Chi-Quadrado (que é próximo de 92\%).

O resultado do teste foi:

\vbox{%A forma mais gseral de remover espaçamento entre linhas:
\baselineskip-1000pt\lineskip0pt\lineskiplimit16383.99999pt\tabskip0pt
\def\linha{\noalign{\hrule}}
\def\hidewidth{\hskip-1000pt plus 1fill}
\def\col{\hbox{\vrule height12pt depth3.5pt width0pt}}
\halign to15cm{\col#& \vrule#\tabskip=1em plus2em&
\hfil#& \vrule#& \hfil#\hfil& \vrule#&
\hfil#& \vrule#&\hfil#& \vrule#\tabskip=0pt\cr\linha
&&\omit\hidewidth Gerador\hidewidth&&\omit\hidewidth
Sucesso\hidewidth&&
\omit\hidewidth Gerador\hidewidth&&Sucesso&\cr\linha
&&SFMT&&94\%&&Xorshiro**&&93\%&\cr\linha
&&PCG&&94\%&&\monoespaco{LCG}&&92\%&\cr\linha
&&ChaCha20&&93\%&&&&&\cr\linha}}

\subsecao{3.10. Teste de Espaçamento de Aniversário}

Neste teste nós geramos um número $n=512$ de valores de 25 bits. Em
seguida, os ordenamos e medimos a diferença entre valores
sucessivos. E contamos o número de distâncias iguais que
encontramos. Este teste foi sugerido por George Marsaglia e quando
proposto encontrou falhas em geradores que passavam nos testes
estatísticos existentes na época.

Nós fazemos o teste acima um total de mil vezes e comparamos o a
distribuição de espaçaments repetidos com o que seria
esperado. Aplicamos então o teste do Chi-Quadrado e por fim repetimos
tudo isso cerca de 3000 vezes para obter a porcentagem de sucesso.

Nosso resultado foi:

\vbox{%A forma mais gseral de remover espaçamento entre linhas:
\baselineskip-1000pt\lineskip0pt\lineskiplimit16383.99999pt\tabskip0pt
\def\linha{\noalign{\hrule}}
\def\hidewidth{\hskip-1000pt plus 1fill}
\def\col{\hbox{\vrule height12pt depth3.5pt width0pt}}
\halign to15cm{\col#& \vrule#\tabskip=1em plus2em&
\hfil#& \vrule#& \hfil#\hfil& \vrule#&
\hfil#& \vrule#&\hfil#& \vrule#\tabskip=0pt\cr\linha
&&\omit\hidewidth Gerador\hidewidth&&\omit\hidewidth
Sucesso\hidewidth&&
\omit\hidewidth Gerador\hidewidth&&Sucesso&\cr\linha
&&SFMT&&92\%&&Xorshiro**&&92\%&\cr\linha
&&PCG&&92\%&&\monoespaco{LCG}&92\%&\cr\linha
&&ChaCha20&&91\%&&&&&\cr\linha}}

\subsecao{3.11. Teste da Correlação Serial}

Este teste gera uma sequência de $n$ elementos (onde $n=1000$) de 64
bits, que aqui vamos interpretar como elementos entre 0 e 1. Ou seja,
dividimos o resultado por $2^{61}-1$ para obter um número em ponto
flutuante. Calculamos então o coeficiente de correlação serial entre
cada um dos números gerados quando comparado com cada número anterior
e medimos se ele está dentro de um intervalo considerado
aceitável.

Em seguida, repetimos esta mesma comparação não apenas entre cada
elemento e seu sucessor, mas também entre cada elemento e o elemento
duas posições à frente. Depois três posições. Até chegar nas 500
posições de deslocamento, tentando encontrar qual nos dá o pior índice
de correlação serial. A medida do pior índice é usada e comparada com
valores esperados de forma que o esperado seja passar neste teste
derca de 93\% das vezes. E repetimos isso mil vezes para ver se a
porcentagem de testes qu passou é próxima desta.

Assim como o anterior, este teste não é uma medida de Chi-Quadrado.

O resultado obtido foi:

\vbox{%A forma mais gseral de remover espaçamento entre linhas:
\baselineskip-1000pt\lineskip0pt\lineskiplimit16383.99999pt\tabskip0pt
\def\linha{\noalign{\hrule}}
\def\hidewidth{\hskip-1000pt plus 1fill}
\def\col{\hbox{\vrule height12pt depth3.5pt width0pt}}
\halign to15cm{\col#& \vrule#\tabskip=1em plus2em&
\hfil#& \vrule#& \hfil#\hfil& \vrule#&
\hfil#& \vrule#&\hfil#& \vrule#\tabskip=0pt\cr\linha
&&\omit\hidewidth Gerador\hidewidth&&\omit\hidewidth
Sucesso\hidewidth&&
\omit\hidewidth Gerador\hidewidth&&Sucesso&\cr\linha
&&SFMT&&93\%&&Xorshiro**&&93\%&\cr\linha
&&PCG&&93\%&&\monoespaco{LCG}&93\%&\cr\linha
&&ChaCha20&&93\%&&&&&\cr\linha}}

\subsecao{3.12. Conclusão dos Testes}

O único algoritmo que notavelmente falha em um dos testes é o Gerador
Linear Congruente. Mesmo sendo construído por parâmetros testados como
tendo uma quaidade maior, ele falha no Teste do Colecionador. Para um
algoritmo a ser usado em jogos, isso é bastante ruim. É possível que
existam itens colecionáveis em um jogo que possam ser obtidos de
maneira aleatória. Mas usando o LCG, os itens são obtidos com um
número de tentativas diferente do que seria esperado se fôsse algo
aleatório.

Entretanto, analizando o teste, o LCG falha não por apresentar
resultados muito distorcidos neste teste, mas sim por apresentar
resultado idealizado demais. Dependendo da intenção desejada, isso
pode não ser um impeditivo para usá-lo.


\secao{4. Medida de Desempenho}


Para medir o desempenho de cada gerador, pedimos para cada um deles
gerar sequencialmente um total de 100 milhões de números aleatórios de
64 bits. O teste foi feito em um computador com processador Intel
Pentium B980 dual core com 2,40 GHz e 4 GB de memória RAM. O Sistema
Operacional foi o Open BSD 6.7. O teste foi compilado com o Clang
8.0.1.

O resultado em segundos foi:

\vbox{%A forma mais gseral de remover espaçamento entre linhas:
\baselineskip-1000pt\lineskip0pt\lineskiplimit16383.99999pt\tabskip0pt
\def\linha{\noalign{\hrule}}
\def\hidewidth{\hskip-1000pt plus 1fill}
\def\col{\hbox{\vrule height12pt depth3.5pt width0pt}}
\halign to15cm{\col#& \vrule#\tabskip=1em plus2em&
\hfil#& \vrule#& \hfil#\hfil& \vrule#&
\hfil#& \vrule#&\hfil#& \vrule#\tabskip=0pt\cr\linha
&&\omit\hidewidth Gerador\hidewidth&&\omit\hidewidth
Sucesso\hidewidth&&
\omit\hidewidth Gerador\hidewidth&&Sucesso&\cr\linha
&&SFMT&&9.386125&&Xorshiro**&&7.703922&\cr\linha
&&PCG&&7.589271&&\monoespaco{LCG}&7.328776&\cr\linha
&&ChaCha20&&10.278860&&&&&\cr\linha}}


\secao{5. Conclusão}

Apesar de gastar muito mais memória que as outras implementações, a
versão utilizada do Mersenne Twister (o SFMT) ainda é muito mais
rápida que o dos algoritmos mais recentes utilizados, mesmo que não
utilizemos algumas otimizações comuns (tornar a função ``inline'' ou
gerar antecipadamente muitos valores em um buffer). Isso possivelmente
se deve às suas operações feitas para serem rápidas em processadores
com registradores e operações de 128 bits. Contudo, é importante notar
que nós só implementamos uma versão do Mersenne Twister que usa tais
recursos e também usa uma extensão da linguagem C com variáveis de 128
bits. Caso estas opções não estejam disponíveis, temos que recorrer a
outro algoritmo.

Por causa disso, e como não há uma grande diferença de qualidade entre
os algoritmos, em nossa biblioteca usaremos o SFMT se possível, e se
não for possível, usemos o PCG por se mostrar ligeiramente mais rápido
e menor que o Xorshiro:

\iniciocodigo
@<Escolhe Algoritmo Padrão do RNG@>=
#if !defined(W_RNG_MERSENNE_TWISTER) && !defined(W_RNG_XORSHIRO) && \
!defined(W_RNG_PCG) && !defined(W_RNG_LCG) && !defined(W_RNG_CHACHA20)
#ifdef __SIZEOF_INT128__
#define W_RNG_MERSENNE_TWISTER
#else
#define W_RNG_PCG
#endif
#endif
@
\fimcodigo

\secao{Referências}

\referencia{Knuth, D. E. (1984) ``Literate Programming'', The Computer Journal,
volume 27, edição 2, p. 97--111}

\referencia{Knuth, D. (1998) ``The Art of Computer Programming, v. 2:
Seminumerical Algorithms'', Addison-Wesley Professional, terceira
edição.}

\referencia{Saito, M.; Matsumoto M. (2006) ``SIMD-oriented fast Mersenne Twister: a 128-bit pseudorandom number generator'', Monte Carlo and Quasi-Monte Carlo Methods, Springer, p. 607--622}.

\referencia{Steele, G.; Vigna S. (2021) ``Computationally Easy, Spectrally Good Multipliers for Congruential Pseudorandom Number Generators'', arXiv preprint arXiv:2001.05304}.

\fim
